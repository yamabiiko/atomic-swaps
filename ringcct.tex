% TeX root = atomic-swaps.tex

\section{RingCCT: Ring confidential commit transaction}
\begin{todobox}
\begin{itemize}
\item need to swap to a PRF scheme, using only two master keys
\end{itemize}
\end{todobox}

We present an extension of RingCT called ring confidential commit transactions (RingCCT), 
which introduces an additional account abstraction enabling timeout-dependent ownership logic for accounts jointly created by two parties. The system state now mantains, in addition to the the list of all existing accounts and associated tags, an integer-valued epoch counter which is incremented upon each succesful verification. Similar to RingCT, each account is associated with a commitment encoding an amount and the corresponding set of public keys. However, the commited data now additionaly encodes a bit indicating the account type and a (possibly zero-valued) timeout epoch parameter, which governs conditional control over the account. \\ 
If the account type is standard, the timeout is set to zero and the system behaves analogously as before. Otherwise, the account type is said to be \textit{commit-based}, and its ownership is jointly held by two distinct users until the system epoch reaches the account-specified timeout. Upon expiration, the ownership reverts to a single designated user. 

\subsection{Syntax}
\paragraph*{System setup and state.} A RingCCT system is initialized by running the setup algorithm $\mathsf{Setup}$ to generate public parameters and the initial state $\st$. Users may invoke the key genation algorithm $\mathsf{KGen}$ to obtain a master key pair $\mathsf{mpk}, \mathsf{msk}$. 
The set of all accounts in the systems, denoted by $\mathsf{AC_U} = \{ \ac_i \}_{i\in U}$ (where U stands for universe), can be extracted from the system state $\st$ via the algorithm $\mathsf{StExt}$. The state also encodes an integer-valued epoch counter which can be extracted with the algorithm $\mathsf{TimeExt}$.

\paragraph*{Accounts.}
Each account $\ac$ comprises a commitment $\mathsf{co}$ to some account data $\accd$, along with a tuple of public keys $\mathsf{pks} = ( \spk, \tpk, \rpk )$, each corresponding to the secret keys  $\mathsf{sks} = ( \ssk, \tsk, \rsk )$. The account data $\accd$ encodes an amount $a \in \mathbb{Z}$, a timeout epoch $\tout \in \mathbb{N}$, and the account type identified by the bit $b \in \{ 0, 1 \}$. If $b = 0$, the account is said to be \textit{standard} and is exclusively controlled by a single master secret key $\mathsf{msk}$. In this case, only the key pair $(\spk, \ssk)$ is populated, with the remaining keys in $\mathsf{pks}$ set to $\perp$. If $b = 1$, the account is classified as \textit{commit-based} and is associated to two distinct master keys, possibily belonging to distinct users. Prior to epoch $\tout$, the account is jointly controlled by the secret keys $\ssk$ and $\tsk$, where $\tsk$ is associated to a distinct master secret key. After the $\tout$, control over the account transitions to  $\rsk$ exclusively, which is bound to the same master key pair as $\ssk$.

\paragraph*{Transaction generation.} Transactions are generated via the algorithm $\mathsf{TxGen}$, which takes as input the system state $\mathsf{st}$, an index set $R \subseteq U$ called the ring, a set of source account information  $\mathcal{S} = \{ \mathsf{sks}_i, r_i, \accd_i \}_{i \in S}$ for some $S \subseteq R$, a set of target account data $\mathcal{T} = \{ \mathsf{mpks}_i, \accd'_i \}_{i \in T}$, and a predicate $P$ over source and target amounts (e.g.,enforcing asset conservation). Each source account $\ac_i := (\accd_i, \mathsf{co}_i)$, for $i \in S$, belongs to the account collection $\mathsf{AC}U$ and holds an amount $a_i$ encoded in $\accd_i = (a_i, \tout_i, b_i)$. The transaction specifies that the amounts ${ a_i }{i \in S}$ are to be transferred to a set of target accounts ${ \ac'i }{i \in T}$, where each $\ac'_i$ is associated with a received amount $a'_i$ stored in $\accd'i$. Correctness of the transaction requires that the predicate $P({ a_i }{i \in S}, { a'i }{i \in T}) = 1$ holds, ensuring the intended relation between source and target assets. The set of source and target accounts may contain a mix of standard and commit-based types. However, to maintain consistency, all commit-based source accounts must share the same epoch timeout $\tout$, since a transaction may only encode a single such timeout.
 The $\mathsf{TxGen}$ additionally generates the set of tokens $\mathsf{TK} = \{ \tx_i \}_{i \in T}$ corresponding to the target accounts $\{ \ac'_i \}_{i \in T}$. These tokens may be used in the key derivation algorithm $\mathsf{KDer}$ to derive the associated secret keys and account data tuple necessary for future transactions. Tokens are assumed to be securely exchanged to the respective owners of the target accounts through a secure channel. Alternatively each token $tk_i$ may be encrypted under the recipient's master public key and  appended to the corresponding $\ac'_i$ to enable future recovery. \\

\paragraph*{Verification.}
Once a transaction is published, each user may run the verification algorithm $\mathsf{TimeVf}$ deciding whether the transaction is valid with respect to the current state $\st$ and the current epoch $\time = \mathsf{TimeExt}(\st)$. If the verification succeds, the verification outputs and updated state $\st'$. \\
An auxiliary verification algorithm $\mathsf{Vf}$ may be used to check the validity of a transaction with respect to an arbitrary epoch $\time$.
The correctness and integrity of account usage within a transaction are verified using the auxiliary algorithms $\mathsf{SrcChk}$ and $\mathsf{TgtChk}$.

	\begin{definition}[Ring  Confidential Commit Transactions (RingCCT)]
    A RingCCT scheme consists of the PPT algorithms $\mathsf{Setup},\mathsf{KGen},\mathsf{KDer}, \mathsf{TxGen},\mathsf{Vf},\mathsf{TimeVf}, \mathsf{StExt},\mathsf{TxExt}, \mathsf{TimeExt}, \mathsf{ExtAccTout}, \mathsf{EvalTags}, \mathsf{TimedOut},$ $\mathsf{SrcChk}, \mathsf{TgtChk}$
    whose interfaces are defined as follows.
    \begin{itemize}
        \item $(\mathsf{pp,st}) \gets \mathsf{Setup}(1^\lambda)$: the setup algorithm generates the public parameters $\mathsf{st}$ and an initial global state $\mathsf{st}$.
        \item $(\mathsf{mpk},\mathsf{msk}) \gets \mathsf{KGen}(\pp)$: the key generation algorithm generates a master public key $\mathsf{mpk}$ and a matching secret key $\mathsf{msk}$.
        \item $(\mathsf{sk},\accd) \gets \mathsf{KDer}(\mathsf{msks, \tk})$: the key derivation algorithm generates derives the keys-account data tuple given the master keys set $\mathsf{msks}$ owning the account and a token $\tk$ of the account.
	\item $(\mathsf{tx,TK}) \gets \mathsf{TxGen}(\mathsf{st},P,R,\mathcal{S},\mathcal{T})$: the transaction algorithm inputs a state $\mathsf{st}$, a predicate $P: \mathbb{Z}^S \times \mathbb{Z}^T \rightarrow \{0,1\}$, an index set R called the ring, a set of source accounts information $\mathcal{S} = \{(\mathsf{sks}_i, r_i \accd_i)\}_{i\in S}$ and a set of target account information $\mathcal{T} = \{\mathsf{mpks}_i, \accd'_i\}_{i\in T}$. Each account is associated with data of the form $\accd := (a, \tout, b)$, where $a$ denotes the account's asset balance, $\tout \in \mathbb{N}$ specifies an epoch-based timeout for ownership logic, and the bit $b \in \{ 0, 1 \}$ indicates the account type - where 0 corresponds to a standard account, and 1 to a commit-based account. The algorithm outputs a transaction $\tx$ and the set of tokens $\mathsf{TK} = \{ \tx_i \}_{i \in T}$ of the target accounts. The transaction $\tx$ is intended to be published to all users of the system, while the token $\tk_i$ is meant to be securely exchanged to the owner of $\mathsf{mpk}$ for $i \in T$.
        \item $(b) \gets \mathsf{Vf}(\st,\tx, \time)$: The verification algorithm outputs a bit $b \in \{ 0, 1 \}$ deciding whether the transaction $\tx$ is valid relative to the state $\st$ and the given input epoch $\time \in \mathbb{N}$.
	\item $(b,\mathsf{st}') \gets \mathsf{TimeVf}(\st,\tx)$: The time verification algorithm extracts the current epoch from the given state and inputs it to the $\mathsf{Vf}$ algorithm,  outputing the result bit $b \in \{ 0, 1 \}$ and a possibly updated state $\st'$.
        \item $\mathsf{AC}_U \gets \mathsf{StExt}(\st)$: The state extraction algorithm extracts the set of universe accounts $\mathsf{AC}_U = \{\ac_i\}_{i \in U}$ encoded in the state $\st$.
        \item $\mathsf{AC}_T \gets \mathsf{TxExt}(\tx)$: The transaction extraction algorithm extracts the set of target accounts $\mathsf{AC}_T = \{\ac_i\}_{i \in T}$ encoded in the state $\st$.
	\item $\time \gets \mathsf{TimeExt}(\tx)$: The time extraction algorithm extracts the epoch $\time \in \mathbb{N}$ encoded in the state $\mathsf{st}$.
	\item $\tout \gets \mathsf{ExtAccTout}(\mathcal{A})$: The account timeout extraction algorithm takes as input set of account data $\mathcal{A} = \{ \accd_i \}_{i \in \mathcal{A}}$ and performs a consistency check to verify that the timeout values $\tout$ are identical across all commit-based and standard accounts in the set. If this condition holds, the algorithm returns the common timeout epoch $\tout \in \mathbb{N}$ associated with the commit-based accounts (if any exist), or returns $0$ if there are no commit-based accounts. If the timeout values are inconsistent, the algorithm returns $\perp$.
	\item $\expired \gets \mathsf{TimedOut}(\time, \tout)$: The timed out algorithm outputs a bit $\expired \in \{0, 1\}$ determining whether $\tout$ is non-zero and $\tout < \time$, with $\tout, \time in \mathbb{N}$.
	\item $\mathcal{Z} \gets \mathsf{EvalTags}(\mathsf{sks})$: The tag evaluation algorithm outputs the set of tags $\mathcal{Z} = \{ \zeta_i \}_{i \in \mathsf{sks}}$ evaluated on the given tuple of secret keys $\mathsf{sks}$.
	\item $b \gets \mathsf{SrcChk}(\ac,\mathsf{sc},\mathcal{Z}, \tout, \expired)$: The source checking takes as input the account $\ac$, along with the related keys and account data encaspulated in $\mathsf{sc} := (\sks, r, \accd)$, a set of tags $\mathcal{Z}$, an epoch timeout $\tout \in \mathbb{N}$, and an expiration bit $\expired \in \{ 0, 1 \}$. The algorithm outputs a bit $b \in \{ 0, 1 \}$ deciding whether to accept or reject that the account $\ac$ is associated to the provided secret keys $\sks$ and tags $\mathcal{Z}$, given  the epoch $\tout$ and the expiration status $\expired$, and that the account data $\accd$ has been commited with randomness $r$.
        \item $b \gets \mathsf{TgtChk}(\mathsf{ac, tk, accd})$: The target checking algorithms outputs a bit $b \in \{ 0, 1 \}$ deciding whether to accept or reject that $\accd$ has been commited in $\ac$. 
    \end{itemize}
\end{definition}

\subsection{Correctness}
\begin{definition}[Correctness] 
    Let $\mathcal{P}$ be a family of predicates. A RingCCT scheme $\Omega$ is $\mathcal{P}$-correct if all of the following holds for any $\lambda \in \mathbb{N}$ and any $(pp, *) \in \mathsf{Setup}(1^\lambda)$.
\end{definition}

\paragraph*{Derivation correctness.} For any $(\mathsf{mpk}_i, \mathsf{msk}_i) \in \mathsf{KGen}(\pp)$ with $i \in \{0, 1\}$ let $(\mathsf{sks}, \ac, \mathsf{tk}, \accd, \accd')$ be a tuple satisfying  $(\mathsf{sks}, \accd') \in \mathsf{KDer}(\{\mathsf{msk}_i\}, \mathsf{tk})$ and $\mathsf{TgtChk}(\ac, \tk, \accd) = 1$. \\ Then it follows $\accd = \accd'$ and  that for every $\pk \in \mathsf{pks}$ there exists an $\sk \in \mathsf{sks}$ such that $\pk = \Delta.\mathsf{KGen}(\pk)$.

\paragraph*{TxGeneration correctness.} Define the set $V_\mathsf{pp}$ to be the collection of all tuples (\st, P, R, S, T), satisfying the following properties:
\begin{itemize}
\item $\mathsf{P} \in \mathcal{P}$
\item $\mathsf{P}(\mathsf{a}_S, \mathsf{a}'_T) = 1$
\item $S \subseteq R \subseteq U$
\item $\mathsf{SrcChk}(\mathsf{StExt}(\mathsf{st})[i], (\mathsf{sks}_{i}, r_i, \accd_i), \mathsf{EvalTags}(\mathsf{sks}_i), \tout, \expired)$ for all $i \in S$.
\end{itemize}

where $\mathcal{S} = \{ \mathsf{sks}_i, \accd_i \}$, $\mathcal{T} = \{ \mathsf{mpks}'_i, \accd'_i \}$, $\tout = \mathsf{ExtAccTout}(\{\accd_i \}_{i \in S})$ and $\expired = \mathsf{TimedOut}(\mathsf{TimeExt}(\st), \tout)$. \\ 
Then for any $(\st, P, R, \mathcal{S}, \mathcal{T}) \in V_\mathsf{pp}$, and any $(\tx, \mathsf{tk}) \in \mathsf{TxGen}(\st,  P, R, \mathcal{S}, \mathcal{T})$, if $(b, \mathsf{st'}) = \mathsf{TimeVf}(\st, \tx)$, the following holds:
\begin{itemize}
\item $b = 1$
\item $\mathsf{TxExt}(\tx) \subseteq \mathsf{StExt}(\st')$
\item $\mathsf{TgtChk}(\mathsf{TxExt}(\mathsf{tx}[i], \tk_i, \accd_i) = 1$  for all i $\in T$
\end{itemize}

\subsection{Security}
We here define the security properties of RingCCT.
\paragraph*{Balance.} Balance ensures correct account ownership and prevents of double-spending and overspeding. Specifically, an account may only spend assets that it owns and have not been previously spent. We first require that the source checking algorithms $\mathsf{SrcChk}$ is computationally binding to a set of secret keys and an associated amount, while the target checking algorithm $\mathsf{TgtChk}$ is binding solely to an amount. We then formalise the balance property via the security experiment $\mathsf{Balance}_{\Omega,\mathcal{P},\adv,\mathcal{E}_\adv}$, where $\adv$ denotes an adversary and $\mathcal{E}_\adv$ a knowledge extractor. The adversary $\adv$ generates a sequence of valid transactions $\tx_i$ for $i \in \mathbb{Z}_l$. The knowledge extractor $\mathcal{E}_\adv$ subsequently extracts the information associated with the source and target accounts for every transaction. The experiments returns 1 if some source or target account is ill-formed or if there exist distinct $i < i'$ such that the sets of source accounts used in the transactions $tx_i$ and $tx_i$ overlap, which indicates a double-spending event.


\begin{definition}[Balance] A RingCCT scheme is balanced if: \\
1. The source checking algorithm $\mathsf{SrcChk}$ computationally binds an account to the stored amount and a set of secret keys determined by the account type $b$ and epoch timeout $\tout$, that is for any PPT adversary $\adv$ it holds that 
\vspace{0.3cm} \\
$\mathsf{Pr}\left[
    \begin{cases} 
	\mathsf{SrcChk}(\ac, \mathsf{sc}, \mathcal{Z}_{S}, \tout, \expired) = 1 \tabularnewline
	\mathsf{SrcChk}(\ac, \mathsf{sc}', \mathcal{Z}_{S}', \tout, \expired) = 1 \tabularnewline
	\accd \neq \accd' \: \lor \ssk \neq \ssk' \: \lor (b = 1  \: \land \tabularnewline
        (e = 1 \land \rsk \neq \rsk') \: \lor (e = 0 \land \tsk \neq \tsk'))) \: \tabularnewline
    \end{cases} 
    \middle|
    \begin{aligned}
	(\pp, \st) \gets \mathsf{Setup}(1^\lambda) \\
	(\ac,  \mathsf{sc}, \mathsf{sc}', \mathcal{Z}_{S}, \mathcal{Z}_{S}', \tout, \expired) \gets \mathcal{A}(\pp) \\
    \end{aligned}
\right]
\leq \negl
\vspace{0.2cm} \\
$ where $\mathsf{sc} = (\mathsf{sks}, r, \accd)$ and $\mathsf{sks} = (\ssk, \rsk, \tsk)$. \vspace{0.3cm} \\
2. The target checking algorithm $\mathsf{TgtChk}$ computationally binds an account to the stored amount and account data, that is for any PPT adversary $\adv$ it holds that
\vspace{0.3cm} \\
$\mathsf{Pr}\left[
    \begin{cases} 
	\mathsf{TgtChk}(\ac, \mathsf{tk}, (a, \tout, b)) = 1 \tabularnewline
	\mathsf{TgtChk}(\ac, \mathsf{tk}', (a', \tout', b')) = 1 \tabularnewline
	(a, \tout, b) \neq (a', \tout', b')
    \end{cases} 
    \middle|
    \begin{aligned}
	(\pp, \st) \gets \mathsf{Setup}(1^\lambda) \\
	(\ac, \mathsf{tk}, \tout, a, b) \gets \mathcal{A}(\pp) \\
	(\ac, \mathsf{tk}', \tout', a', b') \gets \mathcal{A}(\pp) \\
    \end{aligned}
\right]
\leq \negl
$ 
\vspace{0.3cm} \\
3. For any PPT adversary $\adv$ there exists an expected polynomial-time extractor such that
\begin{equation*}
\mathsf{Pr}\left[\mathsf{Balance}_{\Omega,\mathcal{P},\adv,\mathcal{E}_\adv}(1^\lambda) = 1\right] \leq \negl
\end{equation*}
where $\mathsf{Pr}\left[\mathsf{Balance}_{\Omega,\mathcal{P},\adv,\mathcal{E}_\adv}\right]$ is defined in \cref{fig:balance}.\\
\end{definition}

The transaction verification is performed by extracting the current epoch from the state via the time extraction algorithm $\mathsf{TimeExt}$ and then by running the verification algorithm $\mathsf{TimeVf}$. This implies that the balance security properties covers both standard and commit accounts. \\
\begin{figure}[H]
\begin{pchstack}[center, boxed]
\pseudocode{
     \mathsf{Balance}_{\Omega,\mathcal{P},\adv,\mathcal{E}_\adv} \\[0.1\baselineskip ][\hline] 
    (\pp, \mathsf{st}_0) \gets \mathsf{Setup}(1^\lambda) \\
    (\mathsf{tx}_i)_{i \in \mathbb{Z}_l} \gets \mathcal{A}(\pp, \mathsf{st}_0) \\
    (P_i, R_i, S_i, T_i)_{i \in \mathbb{Z}_{l}} \gets \mathcal{E}_{\mathsf{A}} (\pp, \mathsf{st}_0, (\mathsf{tx}_i)_{i \in \mathbb{Z}_{l}}) \\
    \{ \mathsf{sc}_{i, j} := (\mathsf{sks}_{i,j}, r_{i, j}, \accd_{i,j}) \}_{j \in S_i} := \parse \: (S_i)_{i \in \mathbb{Z}_{l}} \\
    \{ \mathsf{mpks}_{i,j},\mathsf{tk}_{i,j}, \accd'_{i,j} \}_{j \in T_i} := \parse \: (T_i)_{i \in \mathbb{Z}_{l}} \\
    \mathbf{for} \: t \in \mathbb{Z}_l \: \mathbf{do} \: (b_t, \mathsf{st}_{t+1}) := \mathsf{TimeVf}(\mathsf{st_t}, \mathsf{tx_t}) \\
    \mathbf{for} \: i \in \mathbb{Z}_l \: \mathbf{do} \\
    \t \accd_{i, S_i} := (\accd_{i,j})_{j \in S_i}, \tout_i := \mathsf{ExtAccTout}(\{\accd_{i, j} \}_{j \in S}) \\
    \t \expired_i := \mathsf{TimedOut}(\mathsf{TimeExt}(\st_i), \tout), \accd'_{i, T_i} := (\accd'_{i,j})_{j \in T_i} \\
    \t \{ \ac_{i,j} \}_{j \in U_i} := \mathsf{StExt}(\mathsf{st}_i), \{ \mathsf{ac'}_{i,j} \}_{j \in T_i} := \mathsf{TxExt}(\mathsf{tx}_i) \\
    \t b'_i := 
    \begin{cases}
	\mathsf{TxExt} \subseteq \mathsf{StExt}(\mathsf{st}_{i+1}) \vspace{0.3em} \tabularnewline
	P_i \in \mathcal{P} \tabularnewline
	P_i(\accd_{i, S_i}, \accd'_{i, T_i}) \vspace{0.3em} \tabularnewline
	S_i \subseteq R_i \subseteq U_i \vspace{0.3em} \tabularnewline
	\mathsf{SrcChk}(\mathsf{StExt}(\mathsf{st}_i)[j], \mathsf{sc}_{i, j}, \mathsf{EvalTags}(\mathsf{sks}_{i, j}), \tout_i, \expired_i) = 1 \:\:\: \forall j \in \mathsf{S}_i \vspace{0.3em} \tabularnewline
	\mathsf{TgtChk}(\mathsf{TxExt}(\mathsf{tx}_i)[j], \tk_{i, j}, \accd_{i,j}) = 1 \:\:\: \forall j \in \mathsf{T}_i \vspace{0.3em} \tabularnewline
    \end{cases} \\
    b'' := (\exists i_0 < i_1, S_{i_0} \cap S_{i_1} = \emptyset) \\
    \pcreturn \bigwedge_{i \in \mathbb{Z}_l} b_i \land \neg (\bigwedge_{i \in \mathbb{Z}_l} b_i' \land b_i'')
}
\end{pchstack}
\caption{Balance experiment definition}
\label{fig:balance}
\end{figure}

\paragraph*{Privacy.} Privacy captures spender and receiver anonymity as well as assets confidentiality. The property is modeled by the security experiment $\mathsf{Privacy}_{\Omega,\adv}^b$, which is parameterised by the bit $b$. The experiments begins by initialising the RingCCT system state $\st$ through the $\mathsf{Setup}$ algorithm. The adversary is then gives access to oracles for account generation, corruption, transaction and verification, as defined in \cref{fig:oracles}. \\
Commit and standard accounts can be generated by calling $\mathsf{CommAccGen}\mathcal{O}$ and $\mathsf{AccGen}\mathcal{O}$ respectively, both of which return the set of associated public keys. Existing accounts can be corrupted via $\mathsf{Corr}\mathcal{O}$, which returns the corresponding secret keys unless the account belongs to the set $\mathsf{ID}^*$. The verification oracle $\mathsf{TimeVf}\mathcal{O}$ takes a transaction $\tx$ as input and updates the system state $\st$ if the given transaction is sucessfully verified with the verification algorithm $\mathsf{TimeVf}$. \\
The transaction oracle $\mathsf{TxGen}\mathcal{O}$ takes as input the tuple $(P, R, \mathcal{S}', \mathcal{S}^*, \mathcal{T}', \mathcal{T}^*)$ provided $\adv$, where $\mathcal{S}'$ and $\mathcal{T}'$ are sets of source and target accounts with keypairs generated by the adversary, and $\mathcal{S}^*$, $\mathcal{T}^*$ instructs the oracle to retrieve keypairs of uncorrupted accounts. The oracle combines the given sets, generates the transaction and returns it alongside the tokens associated with the target accounts. \\ 
The adversary then generates a pair of transactions with inputs $(P, R, \mathcal{S}'_i, \mathcal{S}^*, \mathcal{T}'_i, \mathcal{T}^*)$, where $i \in \{0,1\}$, differing only in the selected set of honest source and target accounts. Honest accounts that are involved in verified transactions are recorded and blocked by including them in the set $\mathsf{AC}^*$. \\
The bit-parametised transaction is then returned to the adversary $\adv$, who may further interact with oracles before outputting a bit that represents the outcome of the experiment.

\begin{definition}[Privacy] A RingCCT scheme is private if for all PPT adversaries $\adv$ it holds that
\begin{equation*}
\mid \mathsf{Pr}\left[\mathsf{Privacy}_{\Omega,\adv}^0(1^\lambda) = 1\right] -  \mathsf{Pr}\left[\mathsf{Privacy}_{\Omega,\adv}^1(1^\lambda) = 1\right] \mid \:\: \leq \negl
\end{equation*}
Where $\mathsf{Privacy}_{\Omega,\adv}^b$ is defined in \cref{fig:privacy}. 
\end{definition}

\begin{figure}
\begin{minipage}[t]{\textwidth}
\begin{pchstack}[center,boxed]
\begin{pcvstack}
\pseudocode{
    \mathsf{AccGen}\mathcal{O}(\mathsf{id}) \\[0.1\baselineskip ][\hline]
    \pcif \mathsf{id} \notin \mathsf{ID} \\
    \t (\mathsf{mpk},\mathsf{msk}) \gets \mathsf{KGen}(\pp) \\
    \t (\mathsf{MPK}, \mathsf{MSK})[\mathsf{id}] := ( \{ \mathsf{mpk} \}, \{ \mathsf{msk} \}) \\
    \mathsf{ID} := \mathsf{ID} \cup \{\mathsf{id}\} \\
    \pcreturn \: \mathsf{MPK}[\mathsf{id}]
}
\vspace{1em}
\pseudocode{
    \mathsf{ComAccGen}\mathcal{O}(\mathsf{id}) \\[0.1\baselineskip ][\hline]
    \pcif \mathsf{id} \notin \mathsf{ID} \\
    \t (\mathsf{smpk},\mathsf{smsk}) \gets \mathsf{KGen}(\pp) \\
    \t (\mathsf{tmpk},\mathsf{tmsk}) \gets \mathsf{KGen}(\pp) \\
    \t (\mathsf{rmpk},\mathsf{rmsk}) \gets \mathsf{KGen}(\pp) \\
    \t \mathsf{mpks} := \{ \mathsf{smpk}, \mathsf{tmpk}, \mathsf{rmpk} \} \\
    \t \mathsf{msks} := \{ \mathsf{smsk}, \mathsf{tmsk}, \mathsf{rmsk} \} \\
    \t (\mathsf{MPK}, \mathsf{MSK})[\mathsf{id}] := (\mathsf{mpks},\mathsf{msks}) \\
    \mathsf{ID} := \mathsf{ID} \cup \{\mathsf{id}\} \\
    \pcreturn \: \mathsf{MPK}[\mathsf{id}]
}
\vspace{1em}
\pseudocode{
	\mathsf{Vf}\mathcal{O}(\mathsf{tx}) \\[0.1\baselineskip ][\hline]
        \pcreturn \: \mathsf{Vf}(\mathsf{st}, \mathsf{tx})
}
\vspace{1em}
\pseudocode{
	\mathsf{TimeVf}\mathcal{O}(\mathsf{tx}) \\[0.1\baselineskip ][\hline]
        \pcreturn \: \mathsf{TimeVf}(\mathsf{st}, \mathsf{tx})
}

\end{pcvstack}
\qquad
\begin{pcvstack}
\pseudocode{
	\mathsf{TxGen}\mathcal{O}(P, R, \mathcal{S'}, \mathcal{T'}, \mathcal{S}^*, \mathcal{T}^*) \\[0.1\baselineskip ][\hline]
	 \{\mathsf{sks}_i, r_i, \accd_i\}_{i \in S'} := \parse \: \mathcal{S'}  \\
	 \{\mathsf{id}_i, \mathsf{tk}_i\}_{i \in S^*} := \parse \: \mathcal{S}^*  \\
	 \{\mathsf{mpks}_i, \accd'_i\}_{i \in T'} := \parse \: \mathcal{T'} \\
	 \{\mathsf{id}'_i, \accd'_i\}_{i \in T^*} := \parse \: \mathcal{T}^*  \\
         \pcif \mathcal{S}^* \cap \mathcal{S}' \neq \emptyset \lor \mathcal{T}' \cap \mathcal{T}^* \: \pcreturn \perp \\
         \pcif (\{\mathsf{id}_{i\in S^*}\} \cup \{\mathsf{id}_{i\in T^*}\}) \cap \mathsf{ID}^* \neq \emptyset \: \pcreturn \perp \\
         \pcif \mathsf{StExt}(\mathsf{st})[\mathcal{S}^*] \cap \mathsf{AC}^* \neq \emptyset \: \pcreturn \perp \\
         \mathcal{S} := \mathcal{S}' \cup \{\mathsf{KDer}(\mathsf{MSK}[\mathsf{id}_i], \mathsf{tk}_i)\}_{i \in S^*} \\
         \mathcal{T} := \mathcal{T}' \cup \{\mathsf{MPK}[\mathsf{id}'_i], \accd'_i\}_{i\in T^*} \\
         (\mathsf{tx}, \mathsf{tks}) \gets \mathsf{TxGen}(\mathsf{st}, P, R, \mathcal{S}, \mathcal{T}) \\
         \mathsf{AC}[\mathsf{id'}_i] := \mathsf{AC}[\mathsf{id'}_i] \cup \mathsf{TxExt}(\mathsf{tx})[i], \forall i \in T \\
         \pcreturn \: (\mathsf{tx}, \mathsf{TK})
}
\vspace{1em}
\pseudocode{
	\mathsf{Corr}\mathcal{O}(\mathsf{id}) \\[0.1\baselineskip ][\hline]
        \pcif \mathsf{id} \notin \mathsf{ID} \pcthen \pcreturn \bot \\
        * \gets \mathsf{AccGen}\mathcal{O}(\mathsf{id}) \\
        \mathsf{ID}^* := \mathsf{ID}^* \cup \{\mathsf{id}\} \\
        \mathsf{AC}^* := \bigcup_{\mathsf{id} \in \mathsf{ID}^*} \mathsf{AC}[\mathsf{id}] \\
        \pcreturn \: \mathsf{MSK}[\mathsf{id}]
}
\end{pcvstack}
\end{pchstack}
\end{minipage}%
\caption{Oracles for the privacy and availability experiments.}
\label{fig:oracles}
\end{figure}

\begin{figure}[H]
\begin{pchstack}[center, boxed]
\pseudocode{
    \mathsf{Privacy}^b_{\mathcal{A}} \\[0.1\baselineskip ][\hline] 
    (\pp, \mathsf{st}) \gets \mathsf{Setup}(1^\lambda) \\
    \mathcal{O} := \{\mathsf{AccGen}\mathcal{O}, \mathsf{CommAccGen}\mathcal{O},\mathsf{Corr}\mathcal{O}, \mathsf{TxGen}\mathcal{O}, \mathsf{TimeVf}\mathcal{O}\} \\
    (P,R, \mathcal{S}', \mathcal{T}', (\mathcal{S}^*_i, \mathcal{T}^*_i)_{i \in \{0,1\}}) \gets \mathcal{A}^\mathcal{O}(\pp) \\
    \mathbf{for} \: i \in \{0, 1\} \\
    \t (\mathsf{tx}_i, *) \gets \mathsf{TxGen}\mathcal{O}(P, R, \mathcal{S}', \mathcal{T}', \mathcal{S}^*_i, \mathcal{T}^*_i) \\
    \t (b_i, \mathsf{st'_i}) := \mathsf{TimeVf}(\mathsf{st}, \mathsf{tx}_i) \\
    \t \pcif b_i = 0 \: \pcreturn \: 0 \\
    \{ \mathsf{id}_{i,j},\mathsf{tk}_{i,j} \}_{j\in S^*_i} := \parse \: \mathcal{S}^*_i \\
    \{ \mathsf{id}'_{i,j},\accd_{i,j} \}_{j\in S^*_i} := \parse \: \mathcal{T}^*_i \\
    \mathsf{ID}^* := \mathsf{ID}^* \cup \{\mathsf{id}_{i,j}\}_{j\in S^*_i} \cup \{\mathsf{id;}_{i,j}\}_{j\in T^*_i} \\
    \mathsf{AC}^* := \mathsf{AC}^* \cup \mathsf{StExt}(\mathsf{st})[\mathcal{S}^*_i] \cup \mathsf{TxExt}(\mathsf{tx}_i)[\mathcal{T}^*_i] \\
    \mathbf{if} (|\mathcal{S^*_0}| \neq |\mathcal{S^*_1}|) \lor (|\mathsf{StExt}(\mathsf{st}'_0) \ \mathsf{StExt}(\mathsf{st})| \neq |\mathsf{StExt}(\mathsf{st}'_1) \ \mathsf{StExt}(\mathsf{st})|) \:\: \pcreturn \: 0 \\
    b' \gets \mathcal{A}^\mathcal{O}(\mathsf{tx}_b) \\
    \pcreturn \: b'
}
\end{pchstack}
\caption{Privacy experiment with oracles defined in \cref{fig:oracles}.}
\label{fig:privacy}
\end{figure}

\begin{definition}[Availability] A RingCCT scheme is available if for all PPT adversaries $\adv$ it holds that
\begin{equation*}
\mathsf{Pr}\left[\mathsf{Availability}_{\Omega,\adv}(1^\lambda) = 1\right] \leq \negl
\end{equation*}
Where $\mathsf{Availability}_{\Omega,\adv}$ is defined in \cref{fig:availability}. 
\end{definition}

\begin{figure}[H]
\begin{pchstack}[center, boxed]
\pseudocode{
    \mathsf{Availability}_{\Omega, \mathcal{A}} \\[0.1\baselineskip ][\hline] 
    (\pp, \mathsf{st}) \gets \mathsf{Setup}(1^\lambda) \\
    \mathcal{O} := \{\mathsf{KGen}\mathcal{O}, \mathsf{Corr}\mathcal{O}, \mathsf{TxGen}\mathcal{O}, \mathsf{Vf}\mathcal{O}\} \\
    (P,R, \mathcal{S}', \mathcal{T}', (\mathcal{S}^*_i, \mathcal{T}^*_i)_{i \in \{0,1\}}) \gets \mathcal{A}^\mathcal{O}(\pp) \\
    (\mathsf{tx}, \mathsf{TK}) \gets \mathsf{TxGen}\mathcal{O}(P, R, \mathcal{S}', \mathcal{T}', \mathcal{S}^*_i, \mathcal{T}^*_i)
    \{ \mathsf{id}_j,\mathsf{tk}_j \}_{j\in S^*} := \parse \: \mathcal{S}^* \\
    \pcif \mathcal{S}^* \not\subseteq U\: \pcreturn \: 0 \\
    (\mathsf{ID}^*, \mathsf{AC}^*) := (\{id_j\}_{j \in S^*}, \mathsf{StExt}(\mathsf{st})[\mathcal{S}^*]) \\
    \mathsf{time} := \mathsf{TimeExt}(\mathsf{st}) \\
    (b, \perp) := \mathsf{Vf}(\mathsf{st}, \mathsf{tx}, \mathsf{time}) \\
    \perp \gets \mathcal{A}^\mathcal{O}(\mathsf{tx}, \mathsf{TK}) \\
    (b', \perp) := \mathsf{Vf}(\mathsf{st}, \mathsf{tx}, \mathsf{time}) \\
    \pcreturn \: b \land b'
}
\end{pchstack}
\caption{Availability experiment with oracles defined in \cref{fig:oracles}.}
\label{fig:availability}
\end{figure}

\paragraph*{Availability.} Availability captures the notion that a valid transaction created at a given point in time should remain valid in the future, unless the associated output has been spent.This provides security against \textit{denial-of-spending} attacks, wherein an adversary attempts to prevent a specific user or account from spending their assets.  We note that in the RingCCT schemes incorporates two verification algorithms, $\mathsf{TimeVf}$ and $\mathsf{Vf}$. The former verifies against the current epoch of the given state, while the latter takes an input an arbitrary epoch. Given that commit-type accounts inherently violate future availability by design, we only require that the validity of a transaction with respect to some epoch is valid at any point in time. Consequently, the experiment $\mathsf{Availability}_{\Omega,\adv}$ is defined with respect to $\mathsf{Vf}$ \\
The security experiment $\mathsf{Availability}_{\Omega,\adv}$ is described in \cref{fig:availability}. Similarly to the privacy experiments, we initialise the RingCCT system and provide the adversary $\adv$ access to the same set of oracles. Following oracle interaction, the adversary $\adv$ specifies the input for the oracle $\mathsf{TxGen}\mathcal{O}$, which computes $\tx$. The set of source accounts used by the $\adv$ must be honest and are subsequently blocked once included in the challenge transaction $\tx$. The adversary $\adv$ receive the transaction $\tx$ and further interacts with the oracles, which may update the system state. After $\adv$ halts, the transaction $\tx$ is verified against the previous state's time with the result denoted by $b'$. The experiment returns $b' \land b$, capturing whether the transaction is valid for some state $\st$ but rejected for $\st'$. \\
\newpage

\subsection{Construction}
In this section we construct a RingCCT scheme $\Omega$ from a tagging scheme $\Delta$, a commitment scheme $\Gamma$ and a non-interactive argument system $\Pi$. The global system state $\st$ consists of the set of all universe accounts $\mathsf{AC}_U$, the set $\mathcal{Z}_U$ containing all valid tags generated by $\Delta$ in previously accepted transactions, and the variable $\time \in \mathbb{N}$ representing the epoch of the system.  \\
Each spent account in the system will be associated with one or more tags recorded in  $\mathcal{Z}_U$, with each tag computationally binding to the account's corresponding secret key. The master public and secret keys of the RingCCT scheme are public and secret keys of $\Delta$ respectively. \\
An account $\ac$ is characterized by the tuple of public keys $\mathsf{pks} := (\mathsf{spk}, \mathsf{tpk}, \mathsf{rpk})$ and a commitment $\mathsf{co} := \mathsf{Com}(\accd, r)$, where $\accd := (a, \tout, b)$ represents the commited account data. Here, $a$ denotes the asset amount owned by the account, $\tout$ defines a timeout timeout, and $b$ is a bit specifying the type of the account (i.e., standard or commit-type). The account is associated to the corresponding tuple of secret keys $\mathsf{sks} := (\ssk, \tsk, \rsk)$, where for each secret key in the tuple $\mathsf{sks}$ we have the corresponding public key in $\mathsf{pks}$ computed with  $\Delta.\mathsf{Eval}$. Each of the secret key in $\mathsf{sks}$ the account data are derivable from the master secret key $\mathsf{msk}$ and the token $\tk := (r, \delta, \accd)$ by computing $\mathsf{sk} := \mathsf{msk} + \delta$ \\
Fix any predicate family $\mathcal{P}$ and let $P \in \mathcal{P}$. To create a transaction, the $\mathsf{TxGen}$ algorithm computs the tags of the corresponding secret keys $\mathsf{sk}_i + \delta_i$ for all source accounts $i \in S$ and creates the target accounts $\ac$. It then generates a non-interactive argument of following relation:
\begin{equation*}
\mathcal{R}(\mathsf{stmnt}, \mathsf{wit}) := \begin{cases} 
    S \subseteq R \\ 
    \mathsf{SrcChk}(\ac_i, \mathsf{sc}_i, \mathcal{Z}_i, \tout, \expired) = 1 \qquad \forall i \in S \\ 
    \mathsf{TgtChk}(\ac'_i, \tk_i, \accd'_i) = 1 \qquad \forall i \in T \\ 
    P(a_S, a'_T) = 1
\end{cases}
\end{equation*}
where the statement $\mathsf{stmnt}$ and witness $\mathsf{wit}$ are of the form
\begin{equation*}
\mathsf{stmnt} := (P,\mathsf{AC}_R,\mathcal{Z}_{\bar{S}}, \mathsf{AC}_T, \tout, \expired) \\
\end{equation*}
\begin{equation*}
	\mathsf{wit} := ((\mathsf{sc}_i)_{i\in S}, (\tk_i, \accd'_i)_{i\in T}) \\
\end{equation*}
respectively, where $\mathsf{sc}_i = (\mathsf{sks}_i, r_i, \accd_i)$. To verify a transaction, the verifier checks that the predicate $P$ is admissible, the ring $R$ is a subset of the universe accounts $R \subseteq U$, the proof for the above relation is sound, and that no tags $\zeta$ given in the transaction appeared before, i.e. $\mathcal{Z}_S \cup \mathcal{Z}_U = \emptyset$. 
If these checks are passed, the verified would agree to advance the state from $\st$ to $\st' := (\mathsf{AC}_T \cup \mathsf{AC}_T ...)$\rlai{incomplete}

\begin{figure}
\begin{minipage}[t]{\textwidth}
\begin{pchstack}[center,boxed]
\begin{pcvstack}
\pseudocode{
    \Setup(1^\lambda) \\ [0.1\baselineskip ][\hline]
    \mathsf{crs} \gets \Pi.\Setup(1^\lambda) \\
    \mathsf{ck} \gets \Gamma.\mathsf{Gen}(1^\lambda) \\
    \pp_\Delta \gets \Delta.\Setup(1^\lambda) \\
    \pcreturn (\pp, \mathsf{st})
}
\vspace{1em}
\pseudocode{
    \mathsf{TimeExt}(\mathsf{st}) \\[0.1\baselineskip ][\hline]
    (\mathsf{AC}_U, \mathcal{Z}_U, \time) := \parse \mathsf{st} \\
    \pcreturn \time
}
\vspace{1em}
\pseudocode{
    \mathsf{StExt}(\mathsf{st}) \\[0.1\baselineskip ][\hline]
    (\mathsf{AC}_U, \mathcal{Z}_U, \time) := \parse \mathsf{st} \\
    \pcreturn \mathsf{AC}_U
}
\vspace{1em}
\pseudocode{
    \mathsf{TimedOut}(\tout, \time) \\[0.1\baselineskip ][\hline]
    \pcreturn \tout \neq 0 \land \tout \overset{?}{<} \time \\
}
\end{pcvstack}
\qquad
\begin{pcvstack}
\pseudocode{
    \mathsf{KGen}(\pp) \\[0.1\baselineskip ][\hline]
    \mathsf{msk} \sample\mathcal{K} \\
    \mathsf{mpk} := \Delta.\mathsf{KGen(msk)} \\
    \pcreturn (\mathsf{mpk}, \mathsf{msk})
}
\vspace{1em}
\pseudocode{
    \mathsf{TxExt}(\mathsf{tx}) \\[0.1\baselineskip ][\hline]
    (P,R,\mathsf{AC}_T, \mathcal{Z}_{\bar{S}}) := \parse \mathsf{tx} \\
    \pcreturn \mathsf{AC}_T
}\vspace{1em}
\pseudocode{
    \mathsf{ExtAccTout}(\mathcal{A}) \\[0.1\baselineskip ][\hline]
    \{(a_i, \tout_i, b_i)\}_{i \in \mathcal{A}} := \parse \mathcal{A} \\
    \mathbf{assert} \not\exists \: i,j \in \mathcal{A} \mid  (\tout_i, b_i) \neq (\tout_j, b_j) \\
    \tout := a \in \{ \tout_i \}_{i \in \mathcal{A}} \mid a \neq 0 \\
    \pcif \tout = \: \perp \\
    \t \pcreturn 0 \\
    \pcreturn \tout
}\vspace{1em}
\pseudocode{
    \mathsf{EvalTags}(\mathsf{sks}) \\[0.1\baselineskip ][\hline]
    \pcreturn \{ \Delta.\mathsf{Eval}(\mathsf{sk}_i) \}_{i \in \mathsf{sks}}  \\
}
\end{pcvstack}
\end{pchstack}
\end{minipage}%
\end{figure}
\begin{figure}
\begin{minipage}[t]{\textwidth}
\begin{pchstack}[center,boxed]

\begin{pcvstack}
\pseudocode{
    \mathsf{Vf}(\mathsf{st},\mathsf{tx}, \time) \\[0.1\baselineskip ][\hline]
    (\mathsf{AC}_U, \mathcal{Z}_U, \perp) := \parse \mathsf{st} \\
    \{\ac_i\}_{i \in U} := \parse \mathsf{AC}_U \\
    (P,R,\mathsf{AC}_T, \mathcal{Z}_{\bar{S}}, \tout, \pi) := \parse \mathsf{tx} \\
    \mathsf{AC}_R := \{\ac_i\}_{i \in R} \\
    \expired := \mathsf{TimedOut}(\tout, \time) \\
    \mathsf{stmnt} := (P, \mathsf{AC}_R, \mathsf{AC}_T,\mathcal{Z}_{\bar{S}}, \tout, \expired) \\
    \pcif \begin{cases}
        P \in \mathcal{P} \tabularnewline
        R \subseteq U \tabularnewline
        \Pi.\mathsf{Vf}(\mathsf{crs}, \mathsf{stmnt}, \pi) = 1 \tabularnewline
        \mathcal{Z}_{\bar{S}} \cap \mathcal{Z}_{\bar{U}} = \emptyset
    \end{cases} \: \mathbf{then} \\
    \t \pcreturn (1, \mathsf{st}') \\
    \pcelse \pcreturn \: (0, \mathsf{st})
}
\vspace{1em}
\pseudocode{
    \mathsf{KDer}(\mathsf{msks},\tau) \\[0.1\baselineskip ][\hline]
    (r, \delta, \accd) := \parse \tau \\
    \mathsf{sks} := \{ \mathsf{msk}_i +\delta \}_{i\in \mathsf{msks}}\\
    \pcreturn (\mathsf{sks}, r, \accd)
}
\vspace{1em}
\pseudocode{
    \mathsf{TgtChk}(\ac, \tk, \accd) \\[0.1\baselineskip ][\hline]
    (\mathsf{pks}, \mathsf{co}) := \parse \ac \\
    (r, \delta, \accd') := \parse \mathsf{tk} \\
    \pcreturn \begin{cases}
        \accd' \overset{?}{=} \accd \tabularnewline
        \mathsf{co} \overset{?}{=} \Gamma.\mathsf{Com}(\accd,r)
    \end{cases} 
}
\vspace{1em}
\pseudocode{
    \mathsf{SrcChk}(\ac, \mathsf{sc}, \mathcal{Z}, \tout, \expired) \\[0.1\baselineskip ][\hline]
    (\mathsf{sks}, r, \accd) := \parse \mathsf{sc} \\
    (a, \tout', b) := \parse \accd \\
    (\mathsf{pks}, \mathsf{co}) := \parse \ac \\
    (\mathsf{ssk}, \mathsf{tsk}, \mathsf{rsk}) := \parse \mathsf{sks} \\
    (\mathsf{spk}, \mathsf{tpk}, \mathsf{rpk}) := \parse \mathsf{pks} \\
    \mathbf{assert} \: \mathsf{co} = \Gamma.\mathsf{Com}(\accd, r) \\
    \mathbf{assert} \: b = 0 \lor (\expired \neq 0 \land \tout' = \tout) \\
    \pcif b = 0 \\
    \t \pcreturn \begin{cases} 
    	\mathsf{spk} \overset{?}{=} \Delta.\mathsf{KGen}(\mathsf{ssk}) \tabularnewline
	\Delta.\mathsf{Eval}(\mathsf{ssk}) \overset{?}{\in} \mathcal{Z}
    \end{cases} \\
    \pcelse \pcif \expired = 1  \\
    \t \pcreturn  \begin{cases}
        \mathsf{tpk} \overset{?}{=} \Delta.\mathsf{KGen}(\mathsf{tsk}) \tabularnewline
        \mathsf{spk} \overset{?}{=} \Delta.\mathsf{KGen}(\mathsf{ssk}) \tabularnewline
	\Delta.\mathsf{Eval}(\mathsf{ssk}), \Delta.\mathsf{Eval}(\mathsf{tsk}) \overset{?}{\in} \mathcal{Z}
    \end{cases} \\
    \pcelse \\
    \t \pcreturn \begin{cases} 
	\mathsf{rpk} \overset{?}{=} \Delta.\mathsf{KGen}(\mathsf{rsk}) \tabularnewline
	\Delta.\mathsf{Eval}(\mathsf{ssk}), \Delta.\mathsf{Eval}(\mathsf{rsk}) \overset{?}{\in} \mathcal{Z}
    \end{cases} \\
}
\end{pcvstack}
\qquad
\begin{pcvstack}
\pseudocode{
    \mathsf{TimeVf}(\mathsf{st}, \mathsf{tx}) \\[0.1\baselineskip ][\hline]
    \time := \parse \mathsf{TimeExt} \\
    \pcreturn \mathsf{Vf}(\mathsf{st}, \mathsf{tx}, \time) \\
}
\vspace{1em}
\pseudocode{
    \mathsf{TxGen}(\mathsf{st}, P, R, \mathcal{S}, \mathcal{T}) \\[0.1\baselineskip ][\hline]
    \{\mathsf{sc}_i := (\mathsf{sks}_i, r_i, \accd_i)\}_{i\in S} := \parse \mathcal{S} \\
    \{(\mathsf{mpks}_i, \accd'_i)\}_{i\in T} := \parse \mathcal{T} \\
    \{\ac_i\}_{i \in U} := \mathsf{StExt(st)} \\
    \time := \mathsf{TimeExt(st)} \\
    \tout \gets \mathsf{ExtAccTout}(\{\accd_i\}_{i\in S}) \\
    \mathbf{assert} \: \tout \neq \: \perp \\
    \expired \gets \mathsf{TimedOut}(\tout, \time) \\
    \mathbf{for} \: i \in T \: \mathbf{do} \\
    \t r'_i \sample \chi \\
    \t \delta_i \sample \mathcal{K} \\
    \t \mathsf{co}_i := \Gamma.\mathsf{Com}(\accd'_i, r'_i) \\
    \t (a_i, \tout, b_i) := \parse \accd'_i \\
    \t (\mathsf{smpk}_i, \mathsf{tmpk}_i, \mathsf{rmpk}_i) := \parse \mathsf{mpks}_i \\
    \t \mathsf{spk}_i := \mathsf{smpk}_i + \Delta.\mathsf{Eval}(\delta_i) \\
    \t \mathsf{tk}_i := (r'_i, \delta_i, \accd'_i) \\
    \t \pcif b \neq 0 \\
    \t\t \mathsf{tpk}_i := \mathsf{tmpk}_i + \Delta.\mathsf{Eval}(\delta_i) \\
    \t\t \mathsf{rpk}_i := \mathsf{rmpk}_i + \Delta.\mathsf{Eval}(\delta_i) \\
    \t\t \mathsf{pks}_i := (\mathsf{spk}_i, \mathsf{tpk}_i, \mathsf{rpk}_i) \\
    \t \pcelse \\
    \t\t \mathsf{pks}_i := (\mathsf{spk}_i, \perp, \perp) \\
    \t \ac'_i := (\mathsf{pks}_i, \mathsf{co}'_i) \\
    \mathsf{AC}_R := \{\ac_i\}_{i \in R} \\
    \mathsf{AC}_T := \{\ac'_i\}_{i \in T} \\
    \mathcal{Z}_{S} := \{ \mathsf{EvalTags}(\mathsf{sks}_i) \}_{i \in S} \\
    \mathsf{stmnt} := (P,\mathsf{AC}_R,\mathsf{AC}_T,\mathcal{Z}_{S}, \tout, \expired) \\
    \mathsf{wit} := ((\mathsf{sc}_i)_{i\in S}), (\tk_i, \accd'_i)_{i\in T}) \\
    \pi \gets \Pi.\mathsf{Prove}(\mathsf{crs},\mathsf{stmnt},\mathsf{wit}) \\
    \mathsf{tx} := (P,\mathsf{AC}_R,\mathsf{AC}_T,\mathcal{Z}_{S}, \tout, \pi) \\
    \mathsf{TK} := \{ \mathsf{tk}_i \}_{i \in T} \\
    \pcreturn (\mathsf{tx}, \mathsf{TK})
}
\end{pcvstack}
\end{pchstack}
\end{minipage}%
\end{figure}
\newpage
