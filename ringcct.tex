% TeX root = atomic-swaps.tex

\section{RingCCT: Ring confidential commit transaction}
We present an extension of ring confidential transactions (RingCT), called ring confidential commit transactions (RingCCT).
RingCCT introduces an additional account abstraction that incorporates commitment-based ownership logic with epoch-based timeout semantics. More precisely, accounts in RingCCT are represented as commitments to account data, including both a token amount and a (possibly zero) timeout parameter, which governs conditional control over the committed asset.

At a high level, RingCCT abstracts the ledger into a set of accounts, each cryptographically encoded as a commitment $\mathsf{co}$ to underlying account data $\mathsf{accd} := (a, \mathsf{time})$, where $a$ represents the committed amount and $\mathsf{time}$ optionally specifies an epoch-based timeout. Each account is associated with a tuple of public keys $(\mathsf{spk}, \mathsf{tpk}, \mathsf{rpk})$ that respectively define:

\begin{itemize}
	\item a primary owner key $\mathsf{spk}$,

	\item an optional joint-control timeout key $\mathsf{tpk}$, and

	\item a recovery key $\mathsf{rpk}$ which gains control after timeout.
\end{itemize}

Bfore the timeout epoch, spending from the account requires joint authorization from the primary and timeout keys; after the timeout, spending transitions to the recovery key alone. When no timeout is defined, the account behaves identically to a standard RingCT output, where only $\mathsf{spk}$ is required to authorize transactions.

\paragraph*{Account Types} We distinguish between two types of accounts:

\begin{itemize}
	\item Standard RingCT accounts: encoded as commitments to $(a, 0)$ with $\mathsf{spk}$ defined, and no meaningful $\mathsf{tpk}$ or $\mathsf{rpk}$. These replicate classic RingCT behavior.

	\item Commit accounts: commit to $(a, \mathsf{time})$ with a complete triplet $(\mathsf{spk}, \mathsf{tpk}, \mathsf{rpk})$. These implement time-based joint ownership and recovery.
\end{itemize}

\paragraph*{Transaction Generation}
Transactions in RingCCT are generated via the algorithm $\mathsf{TxGen}$, which accepts a global state $\mathsf{st}$, a set of source account information $\mathcal{S}$, each including a tuple of secret keys $(\mathsf{ssk}, \mathsf{tsk}, \mathsf{rsk})$ and the committed data $\mathsf{accd}$, a set of target account data $\mathcal{T}$ (including public keys and updated $\mathsf{accd}'$), a ring of public accounts used to obfuscate the true input, and a predicate $P$ over source/target amounts (e.g., sum conservation).
The transaction enforces predicate correctness and zero-knowledge ownership proof via ring signatures, commitments, and zero-knowledge proofs that validates that knowledge of keys consistent with timeout logic, the conservation of committed amounts, and correct embedding of public keys and account data.

\paragraph*{Timeout-Aware Ownership Checks}
The algorithms $\mathsf{SrcChk}$ and $\mathsf{TgtChk}$ verify ownership and integrity of account data based on time:

$\mathsf{SrcChk}$ ensures that the provided secret keys correctly match the account’s public keys and timeout logic. If the account is a commit account with epoch timeout $\mathsf{time}$, it checks:

$(\mathsf{ssk}, \mathsf{tsk})$ are valid when the transaction clock $\mathsf{clock} \leq \mathsf{time}$,

$(\mathsf{rsk})$ is valid when $\mathsf{clock} > \mathsf{time}$.

$\mathsf{TgtChk}$ ensures that the target account includes a valid commitment to the new account data $\mathsf{accd}'$.

\paragraph*{State and Transaction Extraction}
The ledger state and transactions are abstracted into sets of committed accounts via $\mathsf{StExt}$ and $\mathsf{TxExt}$, enabling stateless verification, auditability, and off-chain analysis without leaking sensitive linkage information.

\subsection{Syntax}

\begin{definition}[Ring  Confidential Commit Transactions (RingCCT)]
    A RingCCT scheme consists of the PPT algorithms 
    \[\mathsf{Setup},\mathsf{KGen,Tx,Vf,StExt,TxExt,SrcChk,TgtChk}\] 
    whose interfaces are defined as follows.
    \begin{itemize}
        \item $(\mathsf{pp,st}) \gets \mathsf{Setup}(1^\lambda)$: the setup algorithm generates the public parameters $\mathsf{st}$ and an initial global state $\mathsf{st}$.
        \item $(\mathsf{mpk},\mathsf{msk}) \gets \mathsf{KGen}(\mathsf{pp})$: the key generation algorithm generates a master public key $\mathsf{mpk}$ and a matching secret key $\mathsf{msk}$.
        \item $(\mathsf{sk},\mathsf{accd}) \gets \mathsf{KDer}(\mathsf{msk, \tau})$: the key derivation algorithm generates derives the keys-account data tuple given the master key $\mathsf{msk}$ owning the account and the token $\tau$ of the account.
        \item $(\mathsf{tx,TK}) \gets \mathsf{TxGen}(\mathsf{st},P,R,\mathcal{S},\mathcal{T})$: the transaction algorithm inputs a state $\mathsf{st}$, a predicate $P: \mathbb{Z}^S \times \mathbb{Z}^T \rightarrow \{0,1\}$, an index set R called the ring, a set of source accouts information $\mathcal{S} = \{\mathsf{ssk}_i, \mathsf{tsk}_i, \mathsf{rsk}_i, \mathsf{accd}_i\}_{i\in S}$ and some targets account information $\mathcal{T} = \{\mathsf{smpk}_i, \mathsf{tmpk}_i, \mathsf{rmpk}_i, \mathsf{accd}'_i\}_{i\in T}$; where $\mathsf{ssk,tsk,rsk}$ and $\mathsf{spk,tpk,rpk}$ are the source, target and recovery secret and public keys respectively. If source or target account is not commit based, only the source key pair is defined. Each account has some $\mathsf{accd} := (a,\mathsf{time})$ defined, where $a$ represents some amount and $\mathsf{time}$ sets a specific epoch timeout of the ownership of commit account by the target key pair, and empty otherwise. 
        \item $(b,\mathsf{st}') \gets \mathsf{Vf}(\mathsf{st,tx})$: The verification algorithm outputs a bit b deciding whether to accept or reject that the transaction $\mathsf{tx}$ is a valid relative to the state $\mathsf{st}$, outputting an updated state $\mathsf{st}'$ if the verification is sucessful.
    \item $\mathsf{AC}_U \gets \mathsf{StExt}(\mathsf{st})$: The state extraction algorithm
    extracts the set of universe accounts $\mathsf{AC}_U = \{\mathsf{ac}_i\}_{i \in U}$ encoded in the state $\mathsf{st}$.
    \item $\mathsf{AC}_T \gets \mathsf{TxExt}(\mathsf{tx})$: The transaction extraction algorithm
    extracts the set of universe accounts $\mathsf{AC}_T = \{\mathsf{ac}_i\}_{i \in T}$ encoded in the state $\mathsf{st}$.
    \item $b \gets \mathsf{SrcChk}(\mathsf{ac,r,ssk,tsk,rsk,accd,clock})$: The source checking algorithm outputs a bit $b$ deciding whether to accept or reject that the account $\mathsf{ac}$ is associated to the provided secret keys and that $\mathsf{accd}$ has been commited with randomness $r$. If the account is commit based, it checks validity of $\mathsf{ssk,tsk}$ when $\mathsf{clock} <= \mathsf{time}$ and of $\mathsf{rsk}$ otherwise; if the account is standard only $\mathsf{ssk}$ is required.
    \item $b \gets \mathsf{TgtChk}(\mathsf{ac,accd})$: The target checking algorithms outputs a bit $b$ deciding whether to accept or reject that the $\mathsf{accd}$ has been commited in $\mathsf{ac}$. 
    \end{itemize}
\end{definition}


\subsection{Correctness}

\begin{todobox}
    Decide whether to go with correctness or integrity, i.e. correctness in presence of malicious inputs.    
\end{todobox}


\subsection{Security}
We here define the security properties of RingCCT.

\begin{todobox}
    The balance experiment should also run the time verification.    
\end{todobox}


\begin{definition}[Balance] A RingCCT scheme is balanced if: \\
	1. A commit transaction account ownership changes from $\mathsf{ssk}, \mathsf{tsk}$ to $\mathsf{rsk}$ based on some epoch $\mathsf{time}$, i.e. for any PPT adversary $\mathcal{A}$ it holds that \\
$\mathsf{Pr}\left[
    \begin{cases} 
	\mathsf{SrcChk}(\mathsf{ac}, r, (\mathsf{ssk}, \mathsf{tsk}, \perp), \mathsf{accd}, \mathsf{time}+1) \tabularnewline 
	\mathsf{SrcChk}(\mathsf{ac}, r, (\perp, \perp, \mathsf{rsk}), \mathsf{accd}, \mathsf{time}-1) \tabularnewline
        \Gamma.\mathsf{Com}(\mathsf{accd}, r) = \mathsf{co} 
    \end{cases} 
    \middle|(
    \begin{aligned}
	(\mathsf{pp}, \mathsf{st}) \gets \mathsf{Setup}(1^\lambda) \\
	\mathsf{ac}, \mathsf{ssk}, \mathsf{tsk}, \mathsf{rsk}, \mathsf{accd}) \gets \mathcal{A}(\mathsf{pp}) \\
    	\mathsf{time} := \mathbf{parse} \: \mathsf{accd} \\
    \end{aligned}
\right]
\leq \negl
$
\end{definition}

\begin{figure}[H]
\begin{pchstack}[center, boxed]
\pseudocode{
    \text{Balance} \\[0.1\baselineskip ][\hline] 
    (\mathsf{pp}, \mathsf{st}_0) \gets \mathsf{Setup}(1^\lambda) \\
    (\mathsf{tx}_i)_{i \in \mathbb{Z}_l} \gets \mathcal{A}(\mathsf{pp}, \mathsf{st}_0) \\
    (P_i, R_i, S_i, T_i)_{i \in \mathbb{Z}_{l}} \gets \mathcal{E}_{\mathsf{A}} (\mathsf{pp}, \mathsf{st}_0, (\mathsf{tx}_i)_{i \in \mathbb{Z}_{l}}) \\
    \{ \mathsf{sks}_{i,j}, \mathsf{accd}_{i,j} \}_{j \in S_i} := \mathbf{parse} \: (S_i)_{i \in \mathbb{Z}_{l}} \\
    \{ \mathsf{mpks}_{i,j},\mathsf{tk}_{i,j}, \mathsf{accd}'_{i,j} \}_{j \in T_i}) := \mathbf{parse} \: (T_i)_{i \in \mathbb{Z}_{l}} \\
    \mathbf{for} \: t \in \mathbb{Z}_l \: \mathbf{do} \: (b_t, \mathsf{st}_{t+1}) := \mathsf{Vf}(\mathsf{st_t}, \mathsf{tx_t}) \\
    \mathbf{for} \: i \in \mathbb{Z}_l \: \mathbf{do} \\
    \qquad \mathsf{accd}_{i, S_i} := (\mathsf{accd}_{i,j})_{j \in S_i},
    \mathsf{accd}'_{i, T_i} := (\mathsf{accd}'_{i,j})_{j \in T_i} \\
    \qquad \{ \mathsf{ac}_{i,j} \}_{j \in U_i} := \mathsf{StExt}(\mathsf{st}_i), \{ \mathsf{ac'}_{i,j} \}_{j \in T_i} := \mathsf{TxExt}(\mathsf{tx}_i) \\
    \qquad b'_i := 
    \begin{cases}
	\mathsf{TxExt} \subseteq \mathsf{StExt}(\mathsf{st}_{i+1}) \vspace{0.3em} \tabularnewline
	P_i \in \mathcal{P} \tabularnewline
	P_i(\mathsf{accd}_{i, S_i}, \mathsf{accd}'_{i, T_i}) \vspace{0.3em} \tabularnewline
	S_i \subseteq R_i \subseteq U_i \vspace{0.3em} \tabularnewline
	\mathsf{SrcChk}(\mathsf{StExt}(\mathsf{st}_i)[j], \mathsf{sks}_{i,j}, \mathsf{accd}_{i,j}) = 1 \:\:\: \forall j \in \mathsf{S}_i \vspace{0.3em} \tabularnewline
	\mathsf{TgtChk}(\mathsf{TxExt}(\mathsf{tx}_i)[j], \mathsf{mpks}_{i,j}, \mathsf{accd}_{i,j}) = 1 \:\:\: \forall j \in \mathsf{S}_i \vspace{0.3em} \tabularnewline
    \end{cases} \\
    b'' := (\exists i_0 < i_1, S_{i_0} \cap S_{i_1} = \emptyset) \\
    \mathbf{return} \bigwedge_{i \in \mathbb{Z}_l} b_i \land \neg (\bigwedge_{i \in \mathbb{Z}_l} b_i' \land b_i'')
}
\end{pchstack}
\caption{Balance experiment definition}
\end{figure}

\begin{figure}
\begin{minipage}[t]{\textwidth}
\begin{pchstack}[boxed]
\begin{pcvstack}
\pseudocode{
    \mathsf{KGen}\mathcal{O}(\mathsf{id}) \\[0.1\baselineskip ][\hline]
    \mathbf{if} \: \mathsf{id} \notin \mathsf{ID} \\
    \qquad (\mathsf{mpk},\mathsf{msk}) \gets \mathsf{KGen}(\mathsf{pp}) \\
    \qquad (\mathsf{MPK}, \mathsf{MSK})[\mathsf{id}] := (\mathsf{mpk},\mathsf{msk}) \\
    \mathsf{ID} := \mathsf{ID} \cup \{\mathsf{id}\} \\
    \mathbf{return} \: \mathsf{MPK}[\mathsf{id}]
}
\vspace{1em}
\pseudocode{
	\mathsf{Corr}\mathcal{O}(\mathsf{id}) \\[0.1\baselineskip ][\hline]
        \mathbf{if} \: \mathsf{id} \notin \mathsf{ID}^* \: \mathbf{return} \: \perp \\
        * \gets \mathsf{KGen}\mathcal{O}(\mathsf{id}) \\
        \mathsf{ID}^* := \mathsf{ID}^* \cup \{\mathsf{id}\} \\
        \mathsf{AC}^* := \bigcup_{\mathsf{id} \in \mathsf{ID}^*} \mathsf{AC}[\mathsf{id}] \\
        \mathbf{return} \: \mathsf{MSK}[\mathsf{id}]
}
\vspace{1em}
\pseudocode{
	\mathsf{Vf}\mathcal{O}(\mathsf{tx}) \\[0.1\baselineskip ][\hline]
        \mathbf{return} \: \mathsf{Vf}(\mathsf{st}, \mathsf{tx})
}
\end{pcvstack}
\qquad
\begin{pcvstack}
\pseudocode{
	\mathsf{Trans}\mathcal{O}(P, R, \mathcal{S'}, \mathcal{T'}, \mathcal{S}^*, \mathcal{T}^*) \\[0.1\baselineskip ][\hline]
	 \{\mathsf{ssk}_i, \mathsf{accd}_i\}_{i \in S'} := \mathbf{parse} \: \mathcal{S'}  \\
	 \{\mathsf{id}_i, \mathsf{tk}_i\}_{i \in S^*} := \mathbf{parse} \: \mathcal{S}^*  \\
	 \{\mathsf{mpk}_i, \mathsf{accd}'_i\}_{i \in T'} := \mathbf{parse} \: \mathcal{T'} \\
	 \{\mathsf{id}'_i, \mathsf{accd}'_i\}_{i \in T^*} := \mathbf{parse} \: \mathcal{T}^*  \\
         \mathbf{if} \: \mathcal{S}^* \cap \mathcal{S}' \neq \emptyset \lor \mathcal{T}' \cap \mathcal{T}^* \: \mathbf{return} \perp \\
         \mathbf{if} \: (\{\mathsf{id_i}_{i\in S^*}\} \cup \{\mathsf{id_i}_{i\in T^*}\}) \cap \mathsf{ID}^* \neq \emptyset \: \mathbf{return} \perp \\
         \mathbf{if} \: \mathsf{StExt}(\mathsf{st})[\mathcal{S}^*] \cap \mathsf{AC}^* \neq \emptyset \: \mathbf{return} \perp \\
         \mathcal{S} := \mathcal{S}' \cup \{\mathsf{KDer}(\mathsf{MSK}[\mathsf{id}_i], \mathsf{tk}_i)\}_{i \in S^*} \\
         \mathcal{T} := \mathcal{T}' \cup \{\mathsf{MPK}[\mathsf{id}'_i], \mathsf{accd}'_i\}_{i\in T^*} \\
         (\mathsf{tx}, \mathsf{TK}) \gets \mathsf{Trans}(\mathsf{st}, P, R, \mathcal{S}, \mathcal{T}) \\
         \mathsf{AC}[\mathsf{id'}_i] := \mathsf{AC}[\mathsf{id'}_i] \cup \mathsf{TxExt}(\mathsf{tx})[i], \forall i \in T \\
         \mathbf{return} \: (\mathsf{tx}, \mathsf{TK})
}
\end{pcvstack}
\end{pchstack}
\end{minipage}%
\end{figure}

\begin{definition}[Privacy] A RingCCT scheme is private if: \\
\end{definition}

\begin{todobox}
    The privacy experiment should extract the account times from both sets of inputs and check that they are the same.    
\end{todobox}


\begin{figure}[H]
\begin{pchstack}[center, boxed]
\pseudocode{
    \mathsf{Privacy}^b_{\mathcal{A}} \\[0.1\baselineskip ][\hline] 
    (\mathsf{pp}, \mathsf{st}) \gets \mathsf{Setup}(1^\lambda) \\
    \mathcal{O} := \{\mathsf{KGen}\mathcal{O}, \mathsf{Corr}\mathcal{O}, \mathsf{Trans}\mathcal{O}, \mathsf{Vf}\mathcal{O}\} \\
    (P,R, \mathcal{S}', \mathcal{T}', (\mathcal{S}^*_i, \mathcal{T}^*_i)_{i \in \{0,1\}}) \gets \mathcal{A}^\mathcal{O}(\mathsf{pp}) \\
    \mathbf{for} i \in \{0, 1\} \\
    \:\: (\mathsf{tx}_i, *) \gets \mathsf{Trans}\mathcal{O}(P, R, \mathcal{S}', \mathcal{T}', \mathcal{S}^*_i, \mathcal{T}^*_i)
    \:\: (b_i, \mathsf{st'_i}) := \mathsf{Vf}(\mathsf{st}, \mathsf{tx}_i) \\
    \:\: \mathbf{if} b_i = 0 \mathbf{return} \: 0 \\
    \{ \mathsf{id}_{i,j},\mathsf{tk}_{i,j} \}_{j\in S^*_i} := \mathbf{parse} \: \mathcal{S}^*_i \\
    \{ \mathsf{id}'_{i,j},\mathsf{accd}_{i,j} \}_{j\in S^*_i} := \mathbf{parse} \: \mathcal{T}^*_i \\
    \mathsf{ID}^* := \mathsf{ID}^* \cup \{\mathsf{id}_{i,j}\}_{j\in S^*_i} \cup \{\mathsf{id;}_{i,j}\}_{j\in T^*_i} \\
    \mathsf{AC}^* := \mathsf{AC}^* \cup \mathsf{StExt}(\mathsf{st})[\mathcal{S}^*_i] \cup \mathsf{TxExt}(\mathsf{tx}_i)[\mathcal{T}^*_i] \\
    \mathbf{if} (|\mathcal{S^*_0}| \neq |\mathcal{S^*_1}|) \lor (|\mathsf{StExt}(\mathsf{st}'_0) \ \mathsf{StExt}(\mathsf{st})| \neq |\mathsf{StExt}(\mathsf{st}'_1) \ \mathsf{StExt}(\mathsf{st})|) \:\: \mathbf{return} \: 0 \\
    b' \gets \mathcal{A}^\mathcal{O}(\mathsf{tx}_b) \\
    \mathbf{return} \: b'
}
\end{pchstack}
%\caption{Privacy experiment definition}
\end{figure}



\begin{definition}[Availability] A RingCCT scheme is available if: \\
\end{definition}

\begin{figure*}[H]
\begin{pchstack}[center, boxed]
\pseudocode{
    \mathsf{Available}_{\mathcal{A}} \\[0.1\baselineskip ][\hline] 
    (\mathsf{pp}, \mathsf{st}) \gets \mathsf{Setup}(1^\lambda) \\
    \mathcal{O} := \{\mathsf{KGen}\mathcal{O}, \mathsf{Corr}\mathcal{O}, \mathsf{Trans}\mathcal{O}, \mathsf{Vf}\mathcal{O}\} \\
    (P,R, \mathcal{S}', \mathcal{T}', (\mathcal{S}^*_i, \mathcal{T}^*_i)_{i \in \{0,1\}}) \gets \mathcal{A}^\mathcal{O}(\mathsf{pp}) \\
    (\mathsf{tx}, \mathsf{TK}) \gets \mathsf{Trans}\mathcal{O}(P, R, \mathcal{S}', \mathcal{T}', \mathcal{S}^*_i, \mathcal{T}^*_i)
    \{ \mathsf{id}_j,\mathsf{tk}_j \}_{j\in S^*} := \mathbf{parse} \: \mathcal{S}^* \\
    \mathbf{if} \mathcal{S}^* \not\subseteq U \mathbf{return} \: 0 \\
    (\mathsf{ID}^*, \mathsf{AC}^*) := (\{id_j\}_{j \in S^*}, \mathsf{StExt}(\mathsf{st})[\mathcal{S}^*])
    (b, \perp) := \mathsf{Vf}(\mathsf{st}, \mathsf{tx}) \\
    \perp \gets \mathcal{A}\mathcal{O}(\mathsf{tx}, \mathsf{TK}) \\
    (b', \perp) := \mathsf{Vf}(\mathsf{st}, \mathsf{tx}) \\
    \mathbf{return} \: b' \land b'
}
\end{pchstack}
%\caption{Privacy experiment definition}
\end{figure*}

\newpage

\subsection*{Construction}

\begin{equation*}
\mathcal{R}(\mathsf{stmnt}, \mathsf{wit}) := \begin{cases} 
    S \subseteq R \\ 
    \xi_{\phi S(i)} = \Delta.\mathsf{Eval}(s_i) \qquad \forall i \in S \\
    \mathsf{SrcChk}(\mathsf{ac}_i, r, \mathsf{sks}_i, \mathsf{accd}_i, \mathsf{time}, \mathsf{type}) = 1 \qquad \forall i \in S \\ 
    \mathsf{TgtChk}(\mathsf{ac'}_i, \mathsf{accd}'_i) = 1 \qquad \forall i \in T \\ 
    P(a_S, a'_T) = 1
\end{cases}
\end{equation*}

\begin{equation*}
\mathsf{stmnt} := (P,\mathsf{AC}_R,\mathcal{Z}_{\bar{S}}, \mathsf{AC}_T, \mathsf{time}, \mathsf{type}) \\
\end{equation*}
\begin{equation*}
\mathsf{wit} := ((r,\mathsf{sks}_i, \mathsf{accd}_i)_{i\in S}), (\mathsf{mpks}_i, \mathsf{accd}'_i)_{i\in T}) \\
\end{equation*}



\begin{figure}
\begin{minipage}[t]{\textwidth}
\begin{pchstack}[boxed]

\begin{pcvstack}
\pseudocode{
    \mathsf{Setup}(1^\lambda) \\ [0.1\baselineskip ][\hline]
    \mathsf{crs} \gets \Pi.\mathsf{Setup}(1^\lambda) \\
    \mathsf{ck} \gets \Gamma.\mathsf{Gen}(1^\lambda) \\
    \mathsf{pp_\Delta} \gets \Delta.\mathsf{Setup}(1\lambda) \\
    \mathbf{return} \: (\mathsf{pp}, \mathsf{st})
}
\vspace{1em}
\pseudocode{
    \mathsf{KGen}(\mathsf{pp}) \\[0.1\baselineskip ][\hline]
    \mathsf{msk} \sample\mathcal{K} \\
    \mathsf{mpk} := \Delta.\mathsf{KGen(msk)} \\
    \mathbf{return} \: (\mathsf{mpk}, \mathsf{msk})
}
\vspace{1em}
\pseudocode{
    \mathsf{TimeExt}(\mathsf{st}) \\[0.1\baselineskip ][\hline]
    \mathsf{time} := \mathbf{parse} st \\
    \mathbf{return} \: \mathsf{time}
}
\vspace{1em}
\pseudocode{
    \mathsf{Vf}(\mathsf{st},\mathsf{tx}) \\[0.1\baselineskip ][\hline]
    (\mathsf{AC}_U, \mathcal{Z}_U) := \mathbf{parse} \: \mathsf{st} \\
    \{\mathsf{ac}_i\}_{i \in U} := \mathbf{parse} \: \mathsf{AC}_U \\
    (P,R,\mathsf{AC}_T, \mathcal{Z}_{\bar{S}}) := \mathbf{parse} \: \mathsf{tx} \\
    \mathsf{AC}_R := \{\mathsf{ac}_i\}_{i \in R} \\
    \mathsf{stmnt} := (P,\mathsf{AC}_R,\mathsf{AC}_T,\mathcal{Z}_{\bar{S}}, \mathsf{type}, \mathsf{time}) \\
    \mathbf{if} \: \begin{cases}
        P \in \mathcal{P} \tabularnewline
        R \subseteq U \tabularnewline
        \Pi.\mathsf{Vf}(\mathsf{crs}, \mathsf{stmnt}, \pi) = 1 \tabularnewline
        \mathcal{Z}_{\bar{S}} \cap \mathcal{Z}_{\bar{U}} = \emptyset
    \end{cases} \: \mathbf{then} \\
    \:\: \mathbf{return} \: (1, \mathsf{st}') \\
    \mathbf{else} \: \mathbf{return} \: (0, \mathsf{st})
}
\vspace{1em}
\pseudocode{
    \mathsf{TimeVf}(\mathsf{st}, \mathsf{tx}) \\[0.1\baselineskip ][\hline]
    \mathsf{time} := \mathbf{parse} \: \mathsf{st} \\
    \mathsf{type}, \mathsf{time}' := \mathbf{parse} \: \mathsf{tx} \\
    (\mathsf{ret}, \mathsf{st}') := \mathsf{Vf}(\mathsf{st}, \mathsf{tx}) \\
    \mathbf{if} \: \mathsf{ret} = 1 \\
    \:\: \mathbf{if} \: \mathsf{type} = 0 \: \lor \\
    \:\: \: (\mathsf{type} = 1 \land  \mathsf{time} \leq \mathsf{time}') \: \lor\\
    \:\: \: (\mathsf{type} = -1 \land  \mathsf{time} > \mathsf{time}') \\
    \:\:\:\:\:\: \mathbf{return} \: (1, \mathsf{st}') \\
    \mathbf{return} \: (0, \mathsf{st})
}
\vspace{1em}
\pseudocode{
    \mathsf{StExt}(\mathsf{st}) \\[0.1\baselineskip ][\hline]
    (\mathsf{AC}_U, \mathcal{Z}_U) := \mathbf{parse} \: \mathsf{st} \\
    \mathbf{return} \: \mathsf{AC}_U
}
\vspace{1em}
\pseudocode{
    \mathsf{TxExt}(\mathsf{tx}) \\[0.1\baselineskip ][\hline]
    (P,R,\mathsf{AC}_T, \mathcal{Z}_{\bar{S}}) := \mathbf{parse} \: \mathsf{tx} \\
    \mathbf{return} \: \mathsf{AC}_T
}
\vspace{1em}
\pseudocode{
    \mathsf{TgtChk}(\mathsf{ac}, \mathsf{accd}) \\[0.1\baselineskip ][\hline]
    \mathsf{co} := \mathbf{parse} \: \mathsf{ac} \\
    (r, \mathsf{accd}') := \mathbf{parse} \: \mathsf{tk} \\
    \mathbf{return} \: \begin{cases}
        %\mathsf{accd}' \overset{?}{=} \mathsf{accd} \tabularnewline
        \mathsf{co} \overset{?}{=} \Gamma.\mathsf{Com}(\mathsf{accd},r)
    \end{cases} 
}
\end{pcvstack}
\qquad
\begin{pcvstack}
\pseudocode{
    \mathsf{KDer}(\mathsf{msk},\tau) \\[0.1\baselineskip ][\hline]
    (r, \delta, \mathsf{accd} := (a, \mathsf{time}, \mathsf{type})) := \mathbf{parse} \: \tau \\
    \mathsf{sk} := \mathsf{msk}+\delta \\
    \mathbf{return} \: (\mathsf{sk}, r, \mathsf{accd})
}
\vspace{1em}
\pseudocode{
    \mathsf{TxGen}(\mathsf{st}, P, R, \mathcal{S}, \mathcal{T}) \\[0.1\baselineskip ][\hline]
    \{\mathsf{sks}_i, \mathsf{accd}_i\}_{i\in S} := \mathbf{parse} \: \mathcal{S} \\
    \{\mathsf{mpks}_i, \mathsf{accd}'_i\}_{i\in T} := \mathbf{parse} \: \mathcal{T} \\
    \mathbf{for} \; i \in T \: \mathbf{do} \\
    \:\: r'_i \sample \chi \\
    \:\: \delta'_i \sample \mathcal{K} \\
    \:\: \mathsf{co}'_i := \Gamma.\mathsf{Com}(\mathsf{accd}'_i, r'_i) \\
    \:\: (\mathsf{smpk}_i, \mathsf{tmpk}_i, \mathsf{rmpk}_i) := \mathbf{parse} \: \mathsf{mpks}_i \\
    \:\: \mathsf{spk}_i := \mathsf{smpk}_i + \Delta.\mathsf{Eval}(\delta''_i) \\
    \:\: \mathbf{if} \: \mathsf{rmpk}_i \neq \: \perp \land \: \mathsf{tmpk}_i \neq \: \perp \\
    \:\:\:\: \delta''_i \sample \mathcal{K} \\
    \:\:\:\: \delta'''_i \sample \mathcal{K} \\
    \:\:\:\: \mathsf{tpk}_i := \mathsf{tmpk}_i + \Delta.\mathsf{Eval}(\delta'_i) \\
    \:\:\:\: \mathsf{rpk}_i := \mathsf{rmpk}_i + \Delta.\mathsf{Eval}(\delta''_i) \\
    \:\:\:\: \mathsf{pks}_i := (\mathsf{spk}_i, \mathsf{tpk}_i, \mathsf{rpk}_i) \\
    \:\:\;\: \mathsf{tk}_i := (r'_i, \delta'_i, \delta''_i, \delta'''_i, \mathsf{accd}'_i) \\
    \;\: \mathbf{else} \\
    \:\:\:\: \mathsf{pks}_i := (\mathsf{spk}_i, \perp, \perp) \\
    \:\:\;\: \mathsf{tk}_i := (r'_i, \delta'_i, \mathsf{accd}'_i) \\
    \mathsf{ac}'_i := (\mathsf{pks}_i, \mathsf{co}'_i) \\
    \{\mathsf{ac}_i\}_{i \in U} := \mathsf{StExt(st)} \\
    \mathsf{AC}_R := \{\mathsf{ac}_i\}_{i \in R} \\
    \mathsf{AC}_T := \{\mathsf{ac}_i\}_{i \in T} \\
    \mathcal{Z}_{\bar{S}} := \{\Delta.\mathsf{Eval}(\mathsf{ssk}_i)\}_{\phi S(i)\in \bar{S}} \\
    \mathsf{stmnt} := (P,\mathsf{AC}_R,\mathsf{AC}_T,\mathcal{Z}_{\bar{S}}, \mathsf{time}, \mathsf{type}) \\
    \mathsf{wit} := ((r,\mathsf{sks}_i, \mathsf{accd}_i)_{i\in S}), (\mathsf{mpks}_i, \mathsf{accd}'_i)_{i\in T}) \\
    \pi \gets \Pi.\mathsf{Prove}(\mathsf{crs},\mathsf{stmnt},\mathsf{wit}) \\
    \mathsf{tx} := (P,\mathsf{AC}_R,\mathsf{AC}_T,\mathcal{Z}_{\bar{S}}, \pi) \\
    \mathsf{TK} := {\mathsf{tk}_i}_{i \in T} \\
    \mathbf{return} \: (\mathsf{tx}, \mathsf{TK})
}
\vspace{1em}
\pseudocode{
    \mathsf{SrcChk}(\mathsf{ac}, r, \mathsf{sks}, \mathsf{accd}, \mathsf{time}, \mathsf{type}) \\[0.1\baselineskip ][\hline]
    (\mathsf{ssk}, \mathsf{tsk}, \mathsf{rsk}) := \mathbf{parse} \: \mathsf{sks} \\
    (\mathsf{pks}, \mathsf{co}) := \mathbf{parse} \: \mathsf{ac} \\
    (\mathsf{spk}, \mathsf{tpk}, \mathsf{rpk}) := \mathbf{parse} \: \mathsf{pks} \\
    \mathsf{accd} := (a, \mathsf{time'}, \mathsf{type'}) \\
    \mathbf{if} \: \mathsf{co} \neq \Gamma.\mathsf{Com}(\mathsf{accd}, r) \\
    \:\: \mathbf{return} \: 0 \\
    \mathbf{if} \: \mathsf{type} = \: 0 \\
    \:\: \mathbf{return} \: \mathsf{spk} \overset{?}{=} \Delta.\mathsf{KGen}(\mathsf{ssk}) \\
    \mathbf{else} \: \mathbf{if} \: \mathsf{type} = 1  \\
    \:\: \mathbf{return} \: \begin{cases}
        \mathsf{tpk} \overset{?}{=} \Delta.\mathsf{KGen}(\mathsf{tsk}) \tabularnewline
        \mathsf{spk} \overset{?}{=} \Delta.\mathsf{KGen}(\mathsf{ssk}) \tabularnewline
    \end{cases} \\
    \mathbf{else} \\
    \:\: \mathbf{return} \: \mathsf{rpk} \overset{?}{=} \Delta.\mathsf{KGen}(\mathsf{rsk}) \\
}
\end{pcvstack}
\end{pchstack}
\end{minipage}%
\end{figure}
