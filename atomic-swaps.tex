\documentclass{article}      	% Style of the document                     

%% CONFIG

\IfFileExists{./.config.tex}{
  \input{.config}
}
{
  \def\isanonymous{1}
  \def\isfullversion{1}
  \def\buildexternal{1}
}

%% There is a package that you should include first: nag. Why first? Because it warns you about
%% obsolete commands and package. https://daniel-j-h.github.io/post/latex-a-modern-approach/
\usepackage[l2tabu,orthodox]{nag}

% Full submissions, submitted by the full paper submission deadline, should contain the abstract and the complete paper. The abstract should summarize the paper’s contributions. There is no page limit and authors are encouraged to use the “full version” of their paper as the submission. The submission should contain, within the initial ten pages following the title page, a clear presentation of the merits of the paper, including a discussion of the paper’s importance within the context of prior work and a description of the key technical and conceptual ideas used to achieve its main claims. The submission should be addressed to a broad spectrum of theoretical computer science researchers. Proofs must be provided which can enable the main mathematical claims of the paper to be fully verified. Although there is no bound on the length of a submission, material other than the abstract, references, and the first ten pages will be read at the committee’s discretion. Authors are encouraged to put the references at the very end of the submission. The submission should be typeset using 11-point or larger fonts, in a single-column, single-space (between lines) format with ample spacing throughout and 1-inch margins all around, on letter-size (8 1/2 x 11 inch) paper. Submissions deviating significantly from these guidelines risk rejection without consideration of their merits.

%% ANONYMOUS SUBMISSIONS

\usepackage{ifthen}
\newcommand{\anonymous}[2]{%
\ifthenelse{\equal{\isanonymous}{1}}%
{#1}%
{#2}%
}%

\newcommand{\fullversion}[2]{%
\ifthenelse{\equal{\isfullversion}{1}}%
{#1}%
{#2}%
}%

\newcommand{\submissionresizebox}[1]{
  \fullversion{
    #1
  }{
    \resizebox{.95\textwidth}{!}{#1}
  }
}

\fullversion{
    \usepackage{fullpage}
}{}

%% MARGINS

\fullversion{
  \usepackage[margin=1in]{geometry}
  \setlength{\marginparwidth}{1in}
  \setlength{\marginparsep}{0pt}
}{
  \usepackage[pass]{geometry}
}

%% MISC PACKAGES

\usepackage{microtype}
\usepackage[bookmarksdepth=2,backref]{hyperref}
\hypersetup{colorlinks=true,linkcolor={red!50!black},citecolor=darkgray,linkcolor=darkgray}
\usepackage{booktabs}  %% tables
\usepackage{comment}
\usepackage[inline]{enumitem}
\usepackage{nicematrix}
\usepackage{thm-restate}
\usepackage{tabularx}
\usepackage{breakcites}

\usepackage{amsmath}
\usepackage{amssymb}
\usepackage{xspace}
\usepackage[
    lambda,
    operators,
    advantage,
    sets,
    adversary,
    landau,
    probability,
    notions,
    logic,
    ff,
    mm,
    primitives,
    events,
    complexity,
    asymptotics,
    keys]{cryptocode}


\usepackage{ifluatex}
\ifluatex
\usepackage{luatex85}
\fi
% \usepackage{navigator} % Somehow incompatible?


\usepackage{amsthm}     	   	% Maths    
\newtheorem{definition}{Definition}                                      
\usepackage[utf8]{inputenc}	% UTF-8 characters                                               
\usepackage[T1]{fontenc}    	
% \usepackage{todonotes}
\usepackage{tcolorbox}

\usepackage{amsfonts,graphicx}


%% PDF SETUP

\pagestyle{plain}
\hypersetup{colorlinks=true,citecolor=darkgray,linkcolor=darkgray}
\allowdisplaybreaks

%% REFERENCES
\usepackage[capitalize]{cleveref}
\AddToHook{cmd/appendix/before}{\crefalias{section}{appendix}}
\AddToHook{cmd/appendix/before}{\crefalias{subsection}{appendix}}
\DeclareRobustCommand{\cshref}[1]{{\abbrevcrefs\cref{#1}}}

%% COMMENTS, BUGS, TODOS
\usepackage[draft]{comments-bugs-todos} % use "draft" option to show comments, bugs, todos

\input{macro}



\usepackage{tikz}
\usetikzlibrary{decorations.pathreplacing}
\usetikzlibrary{decorations.pathmorphing}


\definecolor{cgreen}{RGB}{0, 153, 51}
\definecolor{cblue}{RGB}{0, 102, 204}
\definecolor{cyellow}{RGB}{255, 204, 0} 
\definecolor{cred}{RGB}{204, 51, 0} 

\newtcolorbox{todobox}{colback=yellow!3!white, colframe=white!75!black}

\newcommand{\commentline}[2]{%
    \tikz[remember picture, overlay]{
        \node [black,anchor=west,xshift=10pt] at (#1) {#2};
    }
}

\newcommand{\blockcomment}[3]{%
    \tikz[remember picture, overlay]{
        \draw [decorate,decoration={lineto,amplitude=10pt,mirror,raise=4pt},yshift=0pt,very thick,{#3}] 
        (#1) -- (#2) node [black,midway,xshift=10pt] {};
    }
}

\begin{document}         
\author{Russell W. F. Lai \and Lorenzo Tucci}
\title{Ring Confidential Commit Transactions (RingCCT) for Private Atomic Swaps}

\maketitle

\tableofcontents
\newpage


\begin{todobox}
\textbf{Narrative}
\begin{itemize}
\item Universal atomic swaps allow atomic swaps across arbitrary pairs of chains which support ordinary transactions, in particular without requiring support of scripting, time-lock contracts, etc.
\item This is appealing because most privacy chains (e.g. ZCash, Monero, MimbleWimble) do not support scripting.
\item However, UAS requires both parties to solve TLPs, which are computationally intensive especially for lightweight clients.
\item Moreover, it is tricky from an implementation perspective to properly set the difficulty level TLPs in UAS. For example, we identify a minor flaw ... and propose a fix.
\end{itemize}

\textbf{Contributions}
\begin{itemize}
\item We identify that a minimal chain functionality -- commit transactions -- suffices for achieving atomic swaps. Concretely, we propose a generic construction of an atomic swap protocol using only commit transactions and other basic functionalities of the chains. (To avoid dealing with UC, maybe we write this as an informal theorem?)
\item We propose an extension of RingCT, the underlying transaction scheme of Monero, which allows to realise commit transactions in privacy-preserving cryptocurrencies. We propose a generic construction and show how to efficiently instantiate it over groups where discrete logarithm and other related problems are hard. 
\item We provide a prototype implementation of CommitTx-based atomic swaps
\end{itemize}
\end{todobox}

% TeX root = main.tex

\section{Introduction}

\begin{todobox}
    Background of atomic swaps -> bring out the problem -> our contributions
\end{todobox}

\subsection{Atomic Swaps}

\subsection{Our Contributions}

\subsection{Related Work}
% TeX root = atomic-swaps.tex

\section{Preliminaries}

\begin{todobox}
    General notation, e.g. security parameter, $[n]$, PPT, negligible, ...
\end{todobox}

\subsection{Basic primitives}

\begin{todobox}
    commitments, ZKP, ...
\end{todobox}

\subsection{Computational Assumptions}

\begin{todobox}
    General setting of group-based crypto, implicit notation, assumptions that we need e.g. DLOG 
\end{todobox}
% TeX root = main.tex

\section{Atmoic Swaps and Existing Solutions}\label{sec:atomic_swap_overview}

\rlai*{Define (formally or informally) atomic swaps, overview existing constructions from HTLCs and TLPs.}
% TeX root = atomic-swaps.tex

\section{Commit Transactions and Atomic Swaps}

\subsection{Commit Transactions}
The commit transaction primitive offers a lightweight and expressive mechanism for implementing conditional, time-sensitive asset transfers directly on-chain, without requiring a general-purpose scripting environment. This functionality enables users to lock funds under epoch-dependent spending conditions, supporting contract patterns such as atomic swaps, escrows, and delayed claims—all of which rely on time-based control over asset ownership. \\
We note that providing a fully universal definition of a commit transaction primitive is challenging due to the diversity of transaction models across blockchain systems. In particular, fundamental differences between UTXO-based (e.g., Bitcoin, MimbleWimble) and account-based (e.g., Ethereum, ZCash Sapling) systems lead to distinct transaction semantics and interface requirements. As a result, a single formalization of commit transactions that universaly applies to all transaction schemes would either be overly abstract or would need to be tailored closely to each specific model. Nevertheless, the underlying concept of commit transactions - namely, timeout conditional asset ownership — remains simple and expressive. In practice, this functionality can be readily instantiated in various ledger models with modest modifications to fit the specifics of the transaction scheme and primitives in use. \vspace{0.3em} \\
The functionality of a commit transaction can be logically divided into two phases: Commit Phase and Reveal Phase, described as follows: \\
\textbf{Commit Phase}: The user locks a given amount of coins into a special account defined by a commitment to:
    \begin{itemize}
        \item A main secret key $\mathsf{sk}_0$
	\item A set of auxiliary secret keys $\mathsf{sk}_1, \dots, \mathsf{sk}_k$
	\item Two (or more) index sets $I_0, I_1 \subseteq [k]$ which define the required authorized key combinations for spending the funds
	\item A time threshold $T$, denoting an epoch after which the ownership of the accounts transitions.
    \end{itemize}
\textbf{Reveal Phase}: Depending on whether a transaction is issued before or after the epoch threshold $T$, different subsets of auxiliary keys are required:
\begin{itemize}
    \item Before time $T$: The account can be spent using $\mathsf{sk}_0$ and the auxiliary keys indexed by $I_0$
    \item After time $T$: The spending requires $\mathsf{sk}_0$ and keys indexed by $I_1$
\end{itemize}
This time-dependent control structure enables the definition of disjoint execution paths, ensuring that only authorized parties can redeem the funds within their designated time windows. For example, in an atomic swap protocol, one party may claim the funds by providing the required keys before a deadline, while the other party regains control if the swap fails and the deadline elapses.

\subsection{Commit-Transaction-based Atomic Swaps}
In this section we construct a protocol that realizes atomic swaps on a generic blockchain interface supporting commit transactions.

\subsubsection{Blockchain syntax}
\begin{definition} A blockchain system supporting confidential transactions is defined by the following sets of the algorithms \[(\mathsf{TxPub}, \mathsf{TxGen}, \mathsf{CommitTx}, \mathsf{TxVf}, \mathsf{GetState}, \mathsf{TimeExt})\]
\begin{itemize}[topsep=0pt, itemsep=0pt, leftmargin=2em]
    \item $\mathbf{0/1} \gets \mathbf{TxPub}(\tx)$: publishes the transaction $\tx$ on the blockchain. Outputs 1 if the transaction is accepted, 0 otherwise.
    \item $\mathbf{tx}  \gets \mathbf{TxGen}(\st, \{ \pk_i, \sk_i \}_{i \in \mathsf{SC}}, \pk_\mathsf{TG}, \amnt)$: generates a signed transaction transferring an amount $\amnt$ from a source account associated to the keypairs in the set $\mathsf{SC} = \{\pk_i, \sk_i \}_{i \in \mathsf{SC}}$ to the target public key $\pk_\mathsf{TG}$, based on the current state $\st$.
    \item $\mathbf{tx} \gets \mathbf{CommitTx}(\st,  \{ \pk_i, \sk_i \}_{i \in \mathsf{SC}}, \mathsf{pk}_\mathsf{TG}, \{ \pk_i \}_{i \in I_0}, \{ \pk_i \}_{i \in I_11}, T, \amnt)$: produces a commit transaction transferring $\amnt$ from a source account associated to the keypairs in the set $\mathsf{SC} = \{\pk_i, \sk_i \}_{i \in \mathsf{SC}}$  to a commit-type account defined by a main public key $\pk_\mathsf{TG}$, auxiliary key sets $I_0 = \{ \pk_i \}_{i \in I_0}$ and $I_1 = \{ \pk_i \}_{i \in I_1}$, and a timeout parameter $T$.
    \item $\mathbf{0/1} \gets \mathbf{TxVf}(\st, \tx)$: Verifies whether a transaction $\tx$ is valid under the given blockchain state $\st$ and its encoded epoch, which may be extracted using $\mathsf{TimeExt}$. If the source account of the transaction is a commit-type account, it requires that the secret keys corresponding to $\pk_T$ and either one of the auxiliary sets, $\{ \pk_i \}_{i \in I_0}$ or  $I_1 = \{ \pk_i \}_{i \in I_1}$ are included in the set $\mathsf{SC}$ based on defined timeout $T$. Returns 1 if the transaction is valid, 0 otherwise.
    \item $\mathsf{st} \gets \mathbf{GetState}$: Returns the current state $\mathsf{st}$ from the global system state (consensus), including account records, verified transactions, and the current epoch.
    \item $\mathsf{time} \gets \mathbf{TimeExt}$: extracts the epoch from the state.
\end{itemize}
\end{definition}
\subsubsection{Construction}
\paragraph*{Notation.} In order to encode the concurrent execution of routines in an asynchronous setting, we use the following notation in the protocol's pseudocode definition.
\begin{itemize}[nosep, noitemsep]
    \item $\mathbf{wait} \:\: \mathsf{fn}(...)$ - Suspends execution until the execution of the algoritm $\mathsf{fn}(...)$ terminates. If $\mathsf{fn}(...)$ returns $\perp$, the current execution block aborts and returns $\perp$.
    \item $\mathbf{wait} \: \{...\}$ - Enforces the $\mathbf{wait}$ operation to all asynchonous routines inside the execution block.
    \item $\mathbf{assert} \: \{...\}$ - Verifies that enclosed expression or routine returns 1. If not, the current block terminates and returns $\perp$.
    \item $\mathbf{select} \: \{...\}$ - Concurrently runs multiple asynchronous $\mathbf{wait}$ routines in the block and returns the value of the first routine that completes successfully.
\end{itemize}
All routines that interact with network channels—such as accessing the blockchain interface or executing a two-party computation (2PC) with another participant—are treated as asynchronous. This reflects the fact that such operations may incur unpredictable delays and cannot be assumed to complete within a fixed timeframe. \\
In the protocol specification, variables and routines that are specific to a particular blockchain $\mathbb{B}$ are annotated with a subscript to distinguish their context, unless otherwise clear. For example, a public key belonging to chain $\mathbb{B}$ is denoted by $\mathsf{pk_{(\mathbb{B})}}$. \\
Parties may communicate over an unreliable and unauthenticated channel, with no assumptions regarding security or message delivery guarantees. Communication is modeled using the primitives $\mathsf{send}$ and $\mathsf{receive}$, which respectively transmit and await data over the channel.
\paragraph*{Setup.} Let $\mathbb{A}$ and $\mathbb{B}$ denote two blockchains that support confidential constructions, as defined previously. Consider two parties, $P_0$ and $P_1$, who wish to perform a cross-chain asset exchange: $P_0$ aims to swap $\amnt_\mathbb{A}$ units of currency on $\mathbb{A}$ in exchange for $\amnt_\mathbb{B}$ units on $\mathbb{B}$ from $P_1$, and vice versa. \\
Each party generates keypairs for both chains. Specifically, $P_0$ generates $(\pk_i, \sk_i)$ on chain $\mathbb{A}$ and $(\pk_s, \sk_s)$ on chain $\mathbb{B}$, while $P_1$ generates $(\pk_i, \sk_i)$ on $\mathbb{B}$ and $(\pk_s, \sk_s)$ on $\mathbb{A}$. \\
The parties agree on a common set of global parameters: timeouts $T_0$ and $T_1$, and transfer amounts $\amnt_\mathbb{A}$ and $\amnt_\mathbb{B}$. Given this setup, both parties proceed to execute the protocol described in \cref{generic_atomic_protocol}, using the global parameters as public input and their respective keypairs as private input.
\paragraph*{Key generation and commit phase.} 
Each party begins by generating three cryptographic key pairs, each serving a distinct role in the protocol. The pair $(\sk_m, \pk_m)$ is designated as the main keypair, associated with the control of a commit-type account that will eventually hold the locked assets. The second pair, $(\sk_r, \pk_r)$ functions as a recovery keypair, enabling the party to reclaim funds in case the protocol fails to complete within the allotted timeout. The third pair, $(\sk_c, \pk_c)$ is the commitment keypair, which is owned by the respective counterparty in order to allow for a joint ownership of the commit account. Party $P_0$ generates its main and recovery keypairs on blockchain $\mathbb{B}$. Party $P_1$ follows the symmetric strategy: it generates its main and recovery keypairs on blockchain $\mathbb{B}$, and its commitment keypair on blockchain $\mathbb{A}$. Following key generation, the parties exchange their respective public commitment keys over an off-chain channel. \\
Using this setup, $P_0$ locally executes, with respect to chain $\mathbb{A}$ \[\mathsf{CommitTx}(\st, \{ (\pki, \ski) \}, \pkm , \{ \pkc \}, \{ \pkr \}, T_0, \amnt)\], where $(\pki, \ski)$ is the party's source keypair, $\amnt$ is the amount to be exchanged with respect to $\mathbb{A}$, $T_0$ is the agreed timeout parameter, $\pkm$ and $\pkr$ are the main and recovery public key, and $\pkc$ is the commitment public key received by the counterparty. $P_1$ then symmetrically executes $\mathsf{CommitTx}$ with respect to chain $\mathbb{B}$ and using timeout $T_1$, and both parties publish the transactions $\mathsf{tx}_r$ through the $\mathbf{TxPub}$.
\paragraph*{Swap or refund phase.}
The parties now proceed to concurrently write two routines corresponding to the swap and refund mechanism. The refund routine checks whether $T_0$ or $T_1$ (respectively for $P_0$ and $P_1$ are timed out, and if so generate and publish the refund transaction using the master and recovery key pair as source $\mathsf{SC} := \{ (\pkm, \skm), (\pkr, \skr) \}$ paying back $\amnt$ to $\pki$. \\
The swap routine send the previously published transactions initialized the commit accounts $\mathsf{tx}_r$ to the respective counterparty, and they both verify that the transactions are valid through the algorithm $\mathsf{TxVf}$ and that the the $\pkc$ and $\amnt$ are set correctly. The parties now proceed to run the 2PC protocol $\Gamma_\mathsf{Swap}$ as defined in \cref{generic_2pc}. 
After publishing the commit transactions, both parties proceed to concurrently execute two protocol routines that implement the refund and swap mechanisms. These routines are designed to ensure liveness and atomicity of the asset exchange in the presence of network delays or adversarial behavior.
The refund routine serves as a timeout-based recovery path. Each party locally monitors the blockchain state and checks whether the corresponding timeout parameter  or  has expired. Upon detecting a timeout, the party generates and publishes a refund transaction using its main and recovery keypairs as signing authorities, i.e., it sets as source keypairs $\mathsf{SC} := \{ (\pkm, \skm), (\pkr, \skr) \}$. This transaction transfers the originally committed amount $\amnt$ back to the inital public key pair $\pk_i$, ensuring that funds are reclaimed securely if the protocol fails to complete. \\
In parallel, each party initiates the swap routine, which coordinates the completion of the atomic asset exchange. The parties begin by sending the previously published commit transactions $\tx_r$, which initialize the commit-type accounts, to their respective counterparties over an off-chain channel. Upon receipt, each party verifies the validity of the received transaction using the verification algorithm. Additionally, they confirm that the embedded commitment public key $\pkc$ and the committed amount $\amnt$ match the expected values agreed upon during setup. If the validation succeeds, the parties then execute the two-party computation protocol $\Gamma_{\mathsf{Swap}}$, as defined in \cref{generic_2pc}.

\begin{figure}[H]
    \begin{pchstack}[center, boxed]
    \pseudocode{
	    P_0((\pkm, \skm)_\bca,  (\pkc, \skc)_\bcb, \mathsf{\pk_{\mathsf{s}, \bcb}}) \qquad \qquad P_1((\pkm, \skm)_\bcb,  (\pkc, \skc)_\bca, \mathsf{\pk_{\mathsf{s}, \bca}}) \\[0.1\baselineskip ][\hline] 
        \<\< \\[-0.4\baselineskip ]
	\mathbf{assert} \: (\st_\bcb, \st_\bca, \amnt_\bcb,  \amnt_\bca)^0 = (\st_\bcb, \st_\bca, \amnt_\bcb,  \amnt_\bca)^1 \\
	\tx_{\mathsf{s}, \bcb} := \mathsf{TxGen_\bcb}(\st, \{ (\pkm^1, \skm^1), (\pkc^0, \skc^0) \}, \pks^0, \amnt) \\
	\mathbf{assert} \: \mathsf{TxVf}_\bcb(\st, \tx_{\mathsf{s}}) \\
	\tx_{\mathsf{s}, \bca} := \mathsf{TxGen_\bca}(\st, \{ (\pkm^0, \skm^0), (\pkc^1, \skc^1) \}, \pks^1, \amnt) \\
	\mathbf{assert} \: \mathsf{TxVf}_\bca(\st, \tx_{\mathsf{s}}) \\
	\mathsf{lk} \gets \mathcal{H}(\tx_{\mathsf{s}, \bcb}) \oplus \tx_{\mathsf{s}, \bca} \\
        \mathbf{output} \: \mathsf{lk} \: \mathbf{to} \: P_1 \\
        \mathbf{output} \: \tx_{\mathsf{s}, \bcb} \: \mathbf{to} \: P_0
    }
    \end{pchstack}
    \caption{Protocol definition of 2PC $\Gamma_{\mathsf{CommitTx}}$}
    \label{fig:generic_2pc}
    \end{figure}
    \begin{figure}[H]
    \begin{minipage}[t]{0.5\textwidth}
    \begin{pchstack}[boxed]
    \pseudocode{
	\text{Party input} \:\: (\pki, \ski)_{\bca}, (\pks, \sks)_{\bcb} \\[0.1\baselineskip ][\hline]
	(\skm, \pkm)^0 \gets \mathsf{KGen}_\bca(\pp) \\
	(\skr, \pkr)^0 \gets \mathsf{KGen}_\bca(\pp) \\
	(\skc, \pkc)^0 \gets \mathsf{KGen}_\bcb(\pp) \\
	\mathsf{send}(\pk_{\mathsf{c}, \bcb}^0) \\
	\pk_{\mathsf{c}, \bca}^1 \gets \mathsf{receive} \\
	\tx_{\mathsf{r}, \bca} \gets \mathsf{CommitTx}_\bca(\st, (\pki, \ski)^0, \pkm^0, \{ \pkc^1 \}, \{ \pkr^1 \}, T_0, \amnt) \\
	\mathsf{TxPub}_\bca(\tx_\mathsf{r}) \\
        \mathsf{\textbf{select}} \: \{ \\
        \quad \mathsf{\textbf{wait}} \: \{ \\
	\qquad \textbf{do} \: \st \gets \mathsf{GetState}_\bca \\ 
	\qquad \textbf{while} \: \mathsf{TimeExt}_\bca(\st) < T_0 \\
	\qquad \tx_{\mathsf{r}} \gets \mathsf{TxGen_\bca}(\st, \{ (\pkm, \skm)^0, (\pkr, \skr)^0 \}, \pki^0, \amnt) \\
	\qquad \mathsf{TxPub}_\bca(\tx_{\mathsf{r}}) \\
        \quad \} \\
        \quad \mathsf{\textbf{wait}} \:\: \{ \\
	\qquad \mathsf{send}(\tx_{\mathsf{r}, \bca}) \\
	\qquad \mathsf{tx_{\mathsf{r}, \bcb}} \gets \mathsf{receive} \\
	\qquad \textbf{assert} \: \mathsf{TxVf}_\bcb(\mathsf{tx}_\bcb) \\
	\qquad \textbf{assert} \: (\pk_{\mathsf{c}, \bcb}^0, \amnt_\bcb) \in \mathsf{tx_{\mathsf{r}, \bcb}} \\
	\qquad \tx_{\mathsf{s}, \bcb} \gets \Gamma.\mathsf{Swap}(\mathsf{\pk_{\mathsf{m}, \bca}^0}, \pk_{\mathsf{r}, \bca}^0, \pk_{\mathsf{c}, \bcb}^0) \\
	\qquad \textbf{assert} \: \mathsf{TxPub}_\bcb(\tx_{\mathsf{s}}) \\
        \quad \} \\
        \} \\
    }
    \end{pchstack}
    \end{minipage}%
    \hspace{0.7cm}
    \begin{minipage}[t]{0.5\textwidth}
    \begin{pchstack}[boxed]
    \pseudocode{
	\text{Party input} \:\: (\pki, \ski)_{\bcb}, (\pks, \sks)_{\bca} \\[0.1\baselineskip ][\hline] 
	(\skm, \pkm)^1,  \gets \mathsf{KGen}_\bcb(\pp) \\
	(\skr, \pkr)^1 \gets \mathsf{KGen}_\bcb(\pp) \\
	(\skc, \pkc)^1\gets \mathsf{KGen}_\bca(\pp) \\
	\mathsf{send}(\pk_{\mathsf{c}, \bca}^1) \\
	\pk_{\mathsf{c}, \bcb}^0 \gets \mathsf{receive} \\
	\tx_\bcb \gets \mathsf{CommitTx}_\bcb(\st, (\pki, \ski)^1, \pkm^1 , \{ \pkc^0 \}, \{ \pkr^1 \}, T_1, \amnt) \\
	\mathsf{TxPub}_\bcb(\tx_\bcb) \\
        \mathsf{\textbf{select}} \: \{ \\
        \quad \mathsf{\textbf{wait}} \: \{ \\
	\qquad \textbf{do} \: \st \gets \mathsf{GetState}_\bcb \\ 
	\qquad \textbf{while} \: \mathsf{TimeExt}_\bcb(\st) < T_1 \\
	\qquad \tx_{\mathsf{r}} \gets \mathsf{TxGen_\bcb}(\st, \{ (\pkm, \skm)^1, (\pkr, \skc)^1 \}, \pki^1, \amnt) \\
	\qquad \mathsf{TxPub}_\bcb(\tx_{\mathsf{r}}) \\
        \quad \} \\
        \quad \mathsf{\textbf{wait}} \:\: \{ \\
	\qquad \mathsf{send}(\tx_{\mathsf{r}, \bcb}) \\
	\qquad \mathsf{tx_{\mathsf{r}, \bca}} \gets \mathsf{receive} \\
	\qquad \textbf{assert} \: \mathsf{TxVf}_\bca(\mathsf{tx}_\mathsf{r}) \\
	\qquad \textbf{assert} \: (\pk_{\mathsf{c}, \bca}^1\, \amnt_\bca) \in \mathsf{tx_{\mathsf{r}, \bca}} \\
	\qquad \mathsf{lk} \gets \Gamma.\mathsf{Swap}(\mathsf{\pk_{\mathsf{m}, \bcb}^1}, \pk_{\mathsf{r}, \bcb}^1, \pk_{\mathsf{c}, \bca}^1) \\
	\qquad \textbf{do} \: \st \gets \mathsf{GetState}_\bcb \\
	\qquad \textbf{while} \: \not\exists \: \mathsf{tx} \in \st \mid (\pk_{\mathsf{c}, \bcb}^1, \pk_{\mathsf{m}, \bcb}^0) \in \mathsf{tx} \\
	\qquad \tx_{\mathsf{s}, \bcb} := \mathsf{tx} \in \st \mid (\pk_{\mathsf{c}, \bcb}^1, \pk_{\mathsf{m}, \bcb}^0) \in \mathsf{tx} \\
	\qquad \tx_{\mathsf{s}, \bca} := \mathsf{lk} \oplus  \tx_{\mathsf{s}, \bcb} \\
	\qquad \textbf{assert} \: \mathsf{TxPub}_\bca(\tx_{\mathsf{s}}) \\
        \quad \} \\
        \} \\
    }
    \end{pchstack}
    \end{minipage}%
    \caption{Full protocol execution for $P_0$ and $P_1$, respectively left and right}
    \label{fig:generic_atomic_protocol}
    \end{figure}

\subsection{Comparison with HTLC-based Atomic Swaps}

\newpage

% TeX root = atomic-swaps.tex

\section{RingCCT: Ring confidential commit transaction}
We present an extension of ring confidential transactions (RingCT), called ring confidential commit transactions (RingCCT).
RingCCT introduces an additional account abstraction that incorporates commitment-based ownership logic with epoch-based timeout semantics. More precisely, accounts in RingCCT are represented as commitments to account data, including both a token amount and a (possibly zero) timeout parameter, which governs conditional control over the committed asset.

At a high level, RingCCT abstracts the ledger into a set of accounts, each cryptographically encoded as a commitment $\mathsf{co}$ to underlying account data $\accd := (a, \time)$, where $a$ represents the committed amount and $\mathsf{time}$ optionally specifies an epoch-based timeout. Each account is associated with a tuple of public keys $(\mathsf{spk}, \mathsf{tpk}, \mathsf{rpk})$ that respectively define:

\begin{itemize}
	\item a primary owner key $\mathsf{spk}$,

	\item an optional joint-control timeout key $\mathsf{tpk}$, and

	\item a recovery key $\mathsf{rpk}$ which gains control after timeout.
\end{itemize}

Bfore the timeout epoch, spending from the account requires joint authorization from the primary and timeout keys; after the timeout, spending transitions to the recovery key alone. When no timeout is defined, the account behaves identically to a standard RingCT output, where only $\mathsf{spk}$ is required to authorize transactions.

\paragraph*{Account Types} We distinguish between two vfts of accounts:

\begin{itemize}
	\item Standard RingCT accounts: encoded as commitments to $(a, 0)$ with $\mathsf{spk}$ defined, and no meaningful $\mathsf{tpk}$ or $\mathsf{rpk}$. These replicate classic RingCT behavior.

	\item Commit accounts: commit to $(a, \time)$ with a complete triplet $(\mathsf{spk}, \mathsf{tpk}, \mathsf{rpk})$. These implement time-based joint ownership and recovery.
\end{itemize}

\paragraph*{TxGeneration Generation}
TxGenerations in RingCCT are generated via the algorithm $\mathsf{TxGen}$, which accepts a global state $\mathsf{st}$, a set of source account information $\mathcal{S}$, each including a tuple of secret keys $(\mathsf{ssk}, \mathsf{tsk}, \mathsf{rsk})$ and the committed data $\accd$, a set of target account data $\mathcal{T}$ (including public keys and updated $\accd'$), a ring of public accounts used to obfuscate the true input, and a predicate $P$ over source/target amounts (e.g., sum conservation).
The transaction enforces predicate correctness and zero-knowledge ownership proof via ring signatures, commitments, and zero-knowledge proofs that validates that knowledge of keys consistent with timeout logic, the conservation of committed amounts, and correct embedding of public keys and account data.

\paragraph*{Timeout-Aware Ownership Checks}
The algorithms $\mathsf{SrcChk}$ and $\mathsf{TgtChk}$ verify ownership and integrity of account data based on time:

$\mathsf{SrcChk}$ ensures that the provided secret keys correctly match the account’s public keys and timeout logic. If the account is a commit account with epoch timeout $\time$, it checks:

$(\mathsf{ssk}, \mathsf{tsk})$ are valid when the transaction clock $\mathsf{clock} \leq \time$,

$(\mathsf{rsk})$ is valid when $\mathsf{clock} > \time$.

$\mathsf{TgtChk}$ ensures that the target account includes a valid commitment to the new account data $\accd'$.

\paragraph*{State and TxGeneration Extraction}
The ledger state and transactions are abstracted into sets of committed accounts via $\mathsf{StExt}$ and $\mathsf{TxExt}$, enabling stateless verification, auditability, and off-chain analysis without leaking sensitive linkage information.

\subsection{Syntax}
\textbf{Definition} A RingCCT scheme (Ring Commit Confidential TxGenerations) scheme consists of the PPT algorithms ($\mathsf{Setup},\mathsf{KGen},\mathsf{KDer}, \mathsf{Tx},\mathsf{Vf},\mathsf{TimeVf}, \mathsf{StExt},\mathsf{TxExt}, \mathsf{TimeExt}, \mathsf{AccTimeExt}, \mathsf{SrcChk},\mathsf{TgtChk}$) whose interfaces are defined as follows.
\begin{itemize}
        \item $(\mathsf{pp,st}) \gets \mathsf{Setup}(1^\lambda)$: the setup algorithm generates the public parameters $\mathsf{st}$ and an initial global state $\mathsf{st}$.
        \item $(\mathsf{mpk},\mathsf{msk}) \gets \mathsf{KGen}(\pp)$: the key generation algorithm generates a master public key $\mathsf{mpk}$ and a matching secret key $\mathsf{msk}$.
        \item $(\mathsf{sk},\accd) \gets \mathsf{KDer}(\mathsf{msk, \tau})$: the key derivation algorithm generates derives the keys-account data tuple given the master key $\mathsf{msk}$ owning the account and a token $\tau$ of the account.
	\item $(\mathsf{tx,TK}) \gets \mathsf{TxGen}(\mathsf{st},P,R,\mathcal{S},\mathcal{T})$: the transaction algorithm inputs a state $\mathsf{st}$, a predicate $P: \mathbb{Z}^S \times \mathbb{Z}^T \rightarrow \{0,1\}$, an index set R called the ring, a set of source accouts information $\mathcal{S} = \{\mathsf{sks}_i, \accd_i\}_{i\in S}$ and some targets account information $\mathcal{T} = \{\mathsf{mpks}_i, \accd'_i\}_{i\in T}$; where $\mathsf{sks}$ and $\mathsf{mpks}$ may respectively contain the source, target and recovery secret and public keys. If a source or target account is of $\atype$ 0, only the source key pair $\ssk, \mathsf{smpk}$ are defined, respectively. Each account defines its own data as $\accd := (a, \tout, \atype)$, where $a$ represents the amount of assets held by the account, $\atype$ is a bit defining the type of the account (0 for standard and 1 for commit) and $\mathsf{tout}$ sets a specific epoch timeout of the ownership of the target key pair of commit-type accounts, and set to 0 otherwise. 
        \item $(b,\st') \gets \mathsf{Vf}(\st,\tx)$: The verification algorithm outputs a bit b deciding whether to accept or reject that the transaction $\tx$ is a valid relative to the state $\st$, outputting an updated state $\st'$ if the verification is sucessful. This verification is time-independent.
\item $(b,\mathsf{st}') \gets \mathsf{TimeVf}(\st,\tx)$: The time verification algorithm performs the same verification as in $\mathsf{Vf}(\st, \tx)$ with the further constraint of checking whether the declared transaction type $\txtype$ and timeout $\tout$ are consistent with respect to the verifier's epoch $\time$ encoded in $\st$.
    \item $\mathsf{AC}_U \gets \mathsf{StExt}(\st)$: The state extraction algorithm
    extracts the set of universe accounts $\mathsf{AC}_U = \{\mathsf{ac}_i\}_{i \in U}$ encoded in the state $\st$.
    \item $\mathsf{AC}_T \gets \mathsf{TxExt}(\tx)$: The transaction extraction algorithm
    extracts the set of universe accounts $\mathsf{AC}_T = \{\mathsf{ac}_i\}_{i \in T}$ encoded in the state $\st$.
    \item $\time \gets \mathsf{TimeExt}(\tx)$: The time extraction algorithm
    extracts the epoch $\time$ encoded in the state $\mathsf{st}$.
\item $(\tout, \atype) \gets \mathsf{AccTimeExt}(\{\accd\})$: The account time extraction algorithm takes a set of account data $\{\accd\}$ and extracts the epoch timeout $\tout$ and the bit $\atype$ if and only if they are all equal in the set. Returns $\perp$ otherwise.
    \item $b \gets \mathsf{SrcChk}(\mathsf{ac,r,sks,accd,tout,txtype})$: The source checking algorithm outputs a bit $b$ deciding whether to accept or reject that the account $\mathsf{ac}$ is associated to the provided secret keys and that $\accd$ has been commited with randomness $r$. It then checks that the provided secret keys are valid according to the type of transaction $\txtype$
    \item $b \gets \mathsf{TgtChk}(\mathsf{ac,accd})$: The target checking algorithms outputs a bit $b$ deciding whether to accept or reject that $\accd$ has been commited in $\mathsf{ac}$. 
\end{itemize}

\subsection{Correctness}
\begin{definition}[Correctness] Let $\mathcal{P}$ be a family of predicates. A RingCCT scheme $\Omega$ is $\mathcal{P}$-correct if all of the following holds for any $\lambda \in \mathbb{N}$ and any $(pp, *) \in \mathsf{Setup}(1^\lambda)$.
\end{definition}

\paragraph*{Derivation correctness.} For any $(\mathsf{mpk}, \mathsf{msk}) \in \mathsf{KGen}(\pp)$, with $\mathsf{msk} \in \mathsf{msks}$, $\mathsf{mpk} \in \mathsf{mpks}$ $(\mathsf{sks}, \mathsf{ac}, \mathsf{tk}, \accd, \accd')$ satisfying $(\mathsf{sks}, \accd') \in \mathsf{KDer}(\mathsf{msks}, \mathsf{tk})$ and $\mathsf{TgtChk}(\mathsf{ac}, \mathsf{mpks}, \accd_{i}) = 1$ we have $\accd = \accd'$ and $\forall\pk \in \accd \mid \exists \sk \in \mathsf{sks} : \pk = \Delta.\mathsf{KGen}(\pk)$.


\paragraph*{TxGeneration correctness.} Define the set $V_\mathsf{pp}$ to be the collection of all tuples (\st, P, R, S, T) satisfying the following properties: 
\begin{itemize}
\item $\mathsf{P} \in \mathcal{P}$
\item $\mathsf{P}(\mathsf{a}_S, \mathsf{a}'_T) = 1$
\item $S \subseteq R \subseteq U$
\item $\mathsf{SrcChk}(\mathsf{StExt}(\mathsf{st})[i], \mathsf{sks}_{i}, \accd, \atype_{i}, \mathsf{EvalTags}(sks_i))$
\end{itemize}

where $\mathcal{S} = \{ \mathsf{sks}_i, \accd_i \}$, $\mathcal{T} = \{ \mathsf{mpks}'_i, \accd'_i \}$, $(a_i, \tout_i, \atype_i) = \accd_i$. For any $(\st, P, R, \mathcal{S}, \mathcal{T}) \in V_\mathsf{pp}$, and any $(\tx, \mathsf{tks}) \in \mathsf{TxGen}(\st,  P, R, \mathcal{S}, \mathcal{T})$, if $(b, \mathsf{st'}) = \mathsf{Vf}(\st, \tx)$, then the following hold: \\
\begin{itemize}
\item $b = 1$
\item $\mathsf{TxExt}(\tx) \subseteq \mathsf{StExt}(\st')$
\item $\mathsf{TgtChk}(\mathsf{TxExt}(\mathsf{tx}[i], \mathsf{mpks}_{i}, \accd_{i}) = 1$  for all i $\in T$
\end{itemize}

\subsection{Security}
We here define the security properties of RingCCT.

\paragraph*{Balance.} Balance guarantees account ownership and prevention of doublespending and over-speding. That is, an account can only spend owned amounts that they have not already spent. We first require that the source checking algorithms $\mathsf{SrcChk}$ is computationally binding to a set of secret keys and some amount, and just an amount for the target checking algorithm $\mathsf{TgtChk}$. We then model the balance property via the security experiment $\mathsf{Balance}_{\Omega,\mathcal{P},\adv,\epsilon_\adv}$, with $\adv$ being an adversary and $\epsilon_\adv$ a knowledge extractor. The adversary $\adv$ generates a sequence of valid transactions $\tx_i$ for $i \in \mathbb{Z}_l$. The knowledge extractor then $\epsilon_\adv$ extracts the information about source and target accounts for every transaction. The experiments returns 1 if some source or target account is ill-formed or if there exist distinct $i < i'$ such that the source account sets for the transactions $tx_i$ and $tx_i$ overlap, which corresponds to a doublespend.


\begin{definition}[Balance] A RingCCT scheme is balanced if: \\
1. The source checking algorithm $\mathsf{SrcChk}$ computationally binds an account to the stored amount and a set of secret keys determined by the account type $\atype$ and epoch timeout $\tout$, that is for any PPT adversary $\adv$ it holds that 
\vspace{0.3cm} \\
$\mathsf{Pr}\left[
    \begin{cases} 
	    \mathsf{SrcChk}(\mathsf{ac}, r, \ssk, \tsk, \rsk, \mathsf{accd}, \txtype, \mathcal{Z}_{S}) = 1 \tabularnewline
	\mathsf{SrcChk}(\mathsf{ac}, r, \ssk', \tsk', \rsk', \mathsf{accd}',\txtype, \mathcal{Z}_{S}') = 1 \tabularnewline
	(\txtype = 0 \land (\ssk, a) \neq (\ssk', a')) \: \lor \tabularnewline 
	(\txtype = 1 \land (\ssk, \tsk, a) \neq (\ssk', \tsk', a')) \: \lor \tabularnewline
	(\txtype = 2 \land (\ssk, \rsk, a) \neq (\ssk', \rsk', a'))
    \end{cases} 
    \middle|
    \begin{aligned}
	(\pp, \st) \gets \mathsf{Setup}(1^\lambda) \\
	(\mathsf{ac}, \ssk, \tsk, \rsk, \accd, \mathcal{Z}_{S}, \txtype) \gets \mathcal{A}(\pp) \\
	(\mathsf{ac}, \ssk', \tsk', \rsk', \accd',  \mathcal{Z}_{S}') \gets \mathcal{A}(\pp) \\
    \end{aligned}
\right]
\leq \negl
$ 
\vspace{0.3cm} \\
2. The target checking algorithm $\mathsf{TgtChk}$ computationally binds an account to the stored amount and account data, that is for any PPT adversary $\adv$ it holds that
\vspace{0.3cm} \\
$\mathsf{Pr}\left[
    \begin{cases} 
	\mathsf{TgChk}(\mathsf{ac}, \mathsf{tk}, (a, \tout, \atype)) = 1 \tabularnewline
	\mathsf{TgChk}(\mathsf{ac}, \mathsf{tk}', (a', \tout', \atype')) = 1 \tabularnewline
	(a, \tout, \atype) \neq (a', \tout', \atype')
    \end{cases} 
    \middle|
    \begin{aligned}
	(\pp, \st) \gets \mathsf{Setup}(1^\lambda) \\
	(\mathsf{ac}, \mathsf{tk}, \tout, a, \atype) \gets \mathcal{A}(\pp) \\
	(\mathsf{ac}, \mathsf{tk}', \tout', a', \atype') \gets \mathcal{A}(\pp) \\
    \end{aligned}
\right]
\leq \negl
$ 
\vspace{0.3cm} \\
3. For any PPT adversary $\adv$ there exists an expected polynomial-time extractor such that
\begin{equation*}
\mathsf{Pr}\left[\mathsf{Balance}_{\Omega,\mathcal{P},\adv,\epsilon_\adv}(1^\lambda) = 1\right] \leq \negl
\end{equation*}
where $\mathsf{Pr}\left[\mathsf{Balance}_{\Omega,\mathcal{P},\adv,\epsilon_\adv}\right]$ is defined in.\\
\end{definition}
The transaction verification is performed by extracting the current epoch from the state via the time extraction algorithm $\mathsf{TimeExt}$ and then by running the verification algorithm $\mathsf{TimeVf}$. This implies that the balance security properties covers both standard and commit accounts. \\
\begin{figure}[H]
\begin{pchstack}[center, boxed]
\pseudocode{
    \text{Balance} \\[0.1\baselineskip ][\hline] 
    (\pp, \mathsf{st}_0) \gets \mathsf{Setup}(1^\lambda) \\
    (\mathsf{tx}_i)_{i \in \mathbb{Z}_l} \gets \mathcal{A}(\pp, \mathsf{st}_0) \\
    (P_i, R_i, S_i, T_i)_{i \in \mathbb{Z}_{l}} \gets \mathcal{E}_{\mathsf{A}} (\pp, \mathsf{st}_0, (\mathsf{tx}_i)_{i \in \mathbb{Z}_{l}}) \\
    \{ \mathsf{sks}_{i,j}, \accd_{i,j} \}_{j \in S_i} := \parse \: (S_i)_{i \in \mathbb{Z}_{l}} \\
    \{ \mathsf{mpks}_{i,j},\mathsf{tk}_{i,j}, \accd'_{i,j} \}_{j \in T_i}) := \parse \: (T_i)_{i \in \mathbb{Z}_{l}} \\
    \mathbf{for} \: t \in \mathbb{Z}_l \: \mathbf{do} \: (b_t, \mathsf{st}_{t+1}) := \mathsf{TimeVf}(\mathsf{st_t}, \mathsf{tx_t}) \\
    \mathbf{for} \: i \in \mathbb{Z}_l \: \mathbf{do} \\
    \t \accd_{i, S_i} := (\accd_{i,j})_{j \in S_i},
    \accd'_{i, T_i} := (\accd'_{i,j})_{j \in T_i} \\
    \t \{ \mathsf{ac}_{i,j} \}_{j \in U_i} := \mathsf{StExt}(\mathsf{st}_i), \time_i := \mathsf{TimeExt}(\st_i) \\
    \t (\{ \mathsf{ac'}_{i,j} \}, \txtype_{i, j})_{j \in T_i} := \mathsf{TxExt}(\mathsf{tx}_i) \\
    \t b'_i := 
    \begin{cases}
	\mathsf{TxExt} \subseteq \mathsf{StExt}(\mathsf{st}_{i+1}) \vspace{0.3em} \tabularnewline
	P_i \in \mathcal{P} \tabularnewline
	P_i(\accd_{i, S_i}, \accd'_{i, T_i}) \vspace{0.3em} \tabularnewline
	S_i \subseteq R_i \subseteq U_i \vspace{0.3em} \tabularnewline
	\mathsf{SrcChk}(\mathsf{StExt}(\mathsf{st}_i)[j], \mathsf{sks}_{i,j}, \accd_{i,j}, \txtype_{i,j}, \mathsf{EvalTags}(sks_i)) = 1 \:\:\: \forall j \in \mathsf{S}_i \vspace{0.3em} \tabularnewline
	\mathsf{TgtChk}(\mathsf{TxExt}(\mathsf{tx}_i)[j], \mathsf{mpks}_{i,j}, \accd_{i,j}) = 1 \:\:\: \forall j \in \mathsf{T}_i \vspace{0.3em} \tabularnewline
    \end{cases} \\
    b'' := (\exists i_0 < i_1, S_{i_0} \cap S_{i_1} = \emptyset) \\
    \pcreturn \bigwedge_{i \in \mathbb{Z}_l} b_i \land \neg (\bigwedge_{i \in \mathbb{Z}_l} b_i' \land b_i'')
}
\end{pchstack}
\caption{Balance experiment definition}
\end{figure}

\paragraph*{Privacy.} Privacy captures spender and receiver anonymity alongside assets confidentiality. The property is modeled by the security experiment $\mathsf{Privacy}_{\Omega,\adv}^b$ which is parameterised by the bit $b$. The experiments first initialises the RingCCT system state $\st$ through $\mathsf{Setup}$, the adversary is then gives access to oracles for account generation, corruption, transaction and verification. \\
Standard and commit accounts can be generated by calling $\mathsf{AccGen}\mathcal{O}$ and $\mathcal{CommAccGen}\mathcal{O}$ respectively, which returns the set of associated public keys. Existing users can be corrupted via $\mathsf{Corr}\mathcal{O}$, which returns the secret key unless they belong to the set $\mathsf{ID}^*$. The verification oracle $\mathsf{TimeVf}\mathcal{O}$ takes on input a transactions and updates the system state $\st$. \\
The transaction oracle $\mathsf{TxGen}\mathcal{O}$ takes as input the tuple $(P, R, \mathcal{S}', \mathcal{S}^*, \mathcal{T}', \mathcal{T}^*)$ from $\adv$, where $\mathcal{S}'$ and $\mathcal{T}'$ are sets of source and target accounts with keypairs provided by the adversary and $\mathcal{S}^*$ and $\mathcal{T}^*$ encode instruct the oracle to retrieve keypairs of uncorrupted accounts. Combining these sets, the oracle creates the transaction and returns alongside the target account associated tokens. \\
The adversary generates a pair of transactions with input $(P, R, \mathcal{S}'_i, \mathcal{S}^*, \mathcal{T}'_i, \mathcal{T}^*)$ where $i \in {0.1}$, which corresponds to the same input except for  different sets of honest accounts. Honest accounts that are involved in verified transactions are recorded and blocked by including them in the set $\mathsf{AC}^*$. \\
The bit-parametised transaction is then returned to the adversary $\adv$, who can further interact with oracles and finally output a bit corresponding to the output of the experiment.

\begin{definition}[Privacy] A RingCCT scheme is private if for all PPT adversaries $\adv$ it holds that
\begin{equation*}
\mid \mathsf{Pr}\left[\mathsf{Privacy}_{\Omega,\adv}^0(1^\lambda) = 1\right] -  \mathsf{Pr}\left[\mathsf{Privacy}_{\Omega,\adv}^1(1^\lambda) = 1\right] \mid \:\: \leq \negl
\end{equation*}
Where $\mathsf{Privacy}_{\Omega,\adv}^b$ is defined in. 
\end{definition}

\begin{figure}
\begin{minipage}[t]{\textwidth}
\begin{pchstack}[boxed]
\begin{pcvstack}
\pseudocode{
    \mathsf{AccGen}\mathcal{O}(\mathsf{id}) \\[0.1\baselineskip ][\hline]
    \pcif \mathsf{id} \notin \mathsf{ID} \\
    \t (\mathsf{mpk},\mathsf{msk}) \gets \mathsf{KGen}(\pp) \\
    \t (\mathsf{MPK}, \mathsf{MSK})[\mathsf{id}] := ( \{ \mathsf{mpk} \}, \{ \mathsf{msk} \}) \\
    \mathsf{ID} := \mathsf{ID} \cup \{\mathsf{id}\} \\
    \pcreturn \: \mathsf{MPK}[\mathsf{id}]
}
\vspace{1em}
\pseudocode{
    \mathsf{ComAccGen}\mathcal{O}(\mathsf{id}) \\[0.1\baselineskip ][\hline]
    \pcif \mathsf{id} \notin \mathsf{ID} \\
    \t (\mathsf{smpk},\mathsf{smsk}) \gets \mathsf{KGen}(\pp) \\
    \t (\mathsf{tmpk},\mathsf{tmsk}) \gets \mathsf{KGen}(\pp) \\
    \t (\mathsf{rmpk},\mathsf{rmsk}) \gets \mathsf{KGen}(\pp) \\
    \t \mathsf{mpks} := \{ \mathsf{smpk}, \mathsf{tmpk}, \mathsf{rmpk} \} \\
    \t \mathsf{msks} := \{ \mathsf{smsk}, \mathsf{tmsk}, \mathsf{rmsk} \} \\
    \t (\mathsf{MPK}, \mathsf{MSK})[\mathsf{id}] := (\mathsf{mpks},\mathsf{msks}) \\
    \mathsf{ID} := \mathsf{ID} \cup \{\mathsf{id}\} \\
    \pcreturn \: \mathsf{MPK}[\mathsf{id}]
}
\vspace{1em}
\pseudocode{
	\mathsf{Vf}\mathcal{O}(\mathsf{tx}) \\[0.1\baselineskip ][\hline]
        \pcreturn \: \mathsf{Vf}(\mathsf{st}, \mathsf{tx})
}
\vspace{1em}
\pseudocode{
	\mathsf{TimeVf}\mathcal{O}(\mathsf{tx}) \\[0.1\baselineskip ][\hline]
        \pcreturn \: \mathsf{TimeVf}(\mathsf{st}, \mathsf{tx})
}

\end{pcvstack}
\qquad
\begin{pcvstack}
\pseudocode{
	\mathsf{TxGen}\mathcal{O}(P, R, \mathcal{S'}, \mathcal{T'}, \mathcal{S}^*, \mathcal{T}^*) \\[0.1\baselineskip ][\hline]
	 \{\mathsf{sks}_i, \accd_i\}_{i \in S'} := \parse \: \mathcal{S'}  \\
	 \{\mathsf{id}_i, \mathsf{tk}_i\}_{i \in S^*} := \parse \: \mathcal{S}^*  \\
	 \{\mathsf{mpks}_i, \accd'_i\}_{i \in T'} := \parse \: \mathcal{T'} \\
	 \{\mathsf{id}'_i, \accd'_i\}_{i \in T^*} := \parse \: \mathcal{T}^*  \\
         \pcif \mathcal{S}^* \cap \mathcal{S}' \neq \emptyset \lor \mathcal{T}' \cap \mathcal{T}^* \: \pcreturn \perp \\
         \pcif (\{\mathsf{id}_{i\in S^*}\} \cup \{\mathsf{id}_{i\in T^*}\}) \cap \mathsf{ID}^* \neq \emptyset \: \pcreturn \perp \\
         \pcif \mathsf{StExt}(\mathsf{st})[\mathcal{S}^*] \cap \mathsf{AC}^* \neq \emptyset \: \pcreturn \perp \\
         \mathcal{S} := \mathcal{S}' \cup \{\mathsf{KDer}(\mathsf{MSK}[\mathsf{id}_i], \mathsf{tk}_i)\}_{i \in S^*} \\
         \mathcal{T} := \mathcal{T}' \cup \{\mathsf{MPK}[\mathsf{id}'_i], \accd'_i\}_{i\in T^*} \\
         (\mathsf{tx}, \mathsf{tks}) \gets \mathsf{TxGen}(\mathsf{st}, P, R, \mathcal{S}, \mathcal{T}) \\
         \mathsf{AC}[\mathsf{id'}_i] := \mathsf{AC}[\mathsf{id'}_i] \cup \mathsf{TxExt}(\mathsf{tx})[i], \forall i \in T \\
         \pcreturn \: (\mathsf{tx}, \mathsf{TK})
}
\vspace{1em}
\pseudocode{
	\mathsf{Corr}\mathcal{O}(\mathsf{id}) \\[0.1\baselineskip ][\hline]
        \pcif \mathsf{id} \notin \mathsf{ID}^* \: \pcreturn \: \perp \\
        * \gets \mathsf{AccGen}\mathcal{O}(\mathsf{id}) \\
        \mathsf{ID}^* := \mathsf{ID}^* \cup \{\mathsf{id}\} \\
        \mathsf{AC}^* := \bigcup_{\mathsf{id} \in \mathsf{ID}^*} \mathsf{AC}[\mathsf{id}] \\
        \pcreturn \: \mathsf{MSK}[\mathsf{id}]
}
\end{pcvstack}
\end{pchstack}
\end{minipage}%
\caption{Oracles for privacy and availability experiments}
\end{figure}

\begin{figure}[H]
\begin{pchstack}[center, boxed]
\pseudocode{
    \mathsf{Privacy}^b_{\mathcal{A}} \\[0.1\baselineskip ][\hline] 
    (\pp, \mathsf{st}) \gets \mathsf{Setup}(1^\lambda) \\
    \mathcal{O} := \{\mathsf{AccGen}\mathcal{O}, \mathsf{CommAccGen}\mathcal{O},\mathsf{Corr}\mathcal{O}, \mathsf{TxGen}\mathcal{O}, \mathsf{TimeVf}\mathcal{O}\} \\
    (P,R, \mathcal{S}', \mathcal{T}', (\mathcal{S}^*_i, \mathcal{T}^*_i)_{i \in \{0,1\}}) \gets \mathcal{A}^\mathcal{O}(\pp) \\
    \mathbf{for} \: i \in \{0, 1\} \\
    \t (\mathsf{tx}_i, *) \gets \mathsf{TxGen}\mathcal{O}(P, R, \mathcal{S}', \mathcal{T}', \mathcal{S}^*_i, \mathcal{T}^*_i) \\
    \t (b_i, \mathsf{st'_i}) := \mathsf{TimeVf}(\mathsf{st}, \mathsf{tx}_i) \\
    \t \pcif b_i = 0 \: \pcreturn \: 0 \\
    \{ \mathsf{id}_{i,j},\mathsf{tk}_{i,j} \}_{j\in S^*_i} := \parse \: \mathcal{S}^*_i \\
    \{ \mathsf{id}'_{i,j},\accd_{i,j} \}_{j\in S^*_i} := \parse \: \mathcal{T}^*_i \\
    \mathsf{ID}^* := \mathsf{ID}^* \cup \{\mathsf{id}_{i,j}\}_{j\in S^*_i} \cup \{\mathsf{id;}_{i,j}\}_{j\in T^*_i} \\
    \mathsf{AC}^* := \mathsf{AC}^* \cup \mathsf{StExt}(\mathsf{st})[\mathcal{S}^*_i] \cup \mathsf{TxExt}(\mathsf{tx}_i)[\mathcal{T}^*_i] \\
    \mathbf{if} (|\mathcal{S^*_0}| \neq |\mathcal{S^*_1}|) \lor (|\mathsf{StExt}(\mathsf{st}'_0) \ \mathsf{StExt}(\mathsf{st})| \neq |\mathsf{StExt}(\mathsf{st}'_1) \ \mathsf{StExt}(\mathsf{st})|) \:\: \pcreturn \: 0 \\
    b' \gets \mathcal{A}^\mathcal{O}(\mathsf{tx}_b) \\
    \pcreturn \: b'
}
\end{pchstack}
\caption{Privacy experiment definition}
\end{figure}

\begin{definition}[Privacy] A RingCCT scheme is available if for all PPT adversaries $\adv$ it holds that
\begin{equation*}
\mathsf{Pr}\left[\mathsf{Availability}_{\Omega,\adv}(1^\lambda) = 1\right] \leq \negl
\end{equation*}
Where $\mathsf{Availability}_{\Omega,\adv}$ is defined in. 
\end{definition}

\begin{figure}[H]
\begin{pchstack}[center, boxed]
\pseudocode{
    \mathsf{Available}_{\mathcal{A}} \\[0.1\baselineskip ][\hline] 
    (\pp, \mathsf{st}) \gets \mathsf{Setup}(1^\lambda) \\
    \mathcal{O} := \{\mathsf{KGen}\mathcal{O}, \mathsf{Corr}\mathcal{O}, \mathsf{TxGen}\mathcal{O}, \mathsf{Vf}\mathcal{O}\} \\
    (P,R, \mathcal{S}', \mathcal{T}', (\mathcal{S}^*_i, \mathcal{T}^*_i)_{i \in \{0,1\}}) \gets \mathcal{A}^\mathcal{O}(\pp) \\
    (\mathsf{tx}, \mathsf{TK}) \gets \mathsf{TxGen}\mathcal{O}(P, R, \mathcal{S}', \mathcal{T}', \mathcal{S}^*_i, \mathcal{T}^*_i)
    \{ \mathsf{id}_j,\mathsf{tk}_j \}_{j\in S^*} := \parse \: \mathcal{S}^* \\
    \mathbf{if} \mathcal{S}^* \not\subseteq U\: \pcreturn \: 0 \\
    (\mathsf{ID}^*, \mathsf{AC}^*) := (\{id_j\}_{j \in S^*}, \mathsf{StExt}(\mathsf{st})[\mathcal{S}^*])
    (b, \perp) := \mathsf{Vf}(\mathsf{st}, \mathsf{tx}) \\
    \perp \gets \mathcal{A}\mathcal{O}(\mathsf{tx}, \mathsf{TK}) \\
    (b', \perp) := \mathsf{Vf}(\mathsf{st}, \mathsf{tx}) \\
    \pcreturn \: b \land b'
}
\end{pchstack}
\caption{Availability experiment definition}
\end{figure}

\newpage

\subsection{Construction}

\begin{equation*}
\mathcal{R}(\mathsf{stmnt}, \mathsf{wit}) := \begin{cases} 
    S \subseteq R \\ 
    \mathsf{TagChk}(\mathcal{Z}_{S(i)}) \forall i \in S \\
    \mathsf{SrcChk}(\mathsf{ac}_i, r, \mathsf{sks}_i, \accd_i, \time, \txtype) = 1 \qquad \forall i \in S \\ 
    \mathsf{TgtChk}(\mathsf{ac'}_i, \accd'_i) = 1 \qquad \forall i \in T \\ 
    P(a_S, a'_T) = 1
\end{cases}
\end{equation*}

\begin{equation*}
\mathsf{stmnt} := (P,\mathsf{AC}_R,\mathcal{Z}_{\bar{S}}, \mathsf{AC}_T, \time, \txtype) \\
\end{equation*}
\begin{equation*}
\mathsf{wit} := ((r,\mathsf{sks}_i, \accd_i)_{i\in S}), (\mathsf{mpks}_i, \accd'_i)_{i\in T}) \\
\end{equation*}

\begin{figure}
\begin{minipage}[t]{\textwidth}
\begin{pchstack}[boxed]
\begin{pcvstack}
\pseudocode{
    \Setup(1^\lambda) \\ [0.1\baselineskip ][\hline]
    \mathsf{crs} \gets \Pi.\Setup(1^\lambda) \\
    \mathsf{ck} \gets \Gamma.\mathsf{Gen}(1^\lambda) \\
    \pp_\Delta \gets \Delta.\Setup(1^\lambda) \\
    \pcreturn (\pp, \mathsf{st})
}
\vspace{1em}
\pseudocode{
    \mathsf{TimeExt}(\mathsf{st}) \\[0.1\baselineskip ][\hline]
    (\mathsf{AC}_U, \mathcal{Z}_U, \time) := \parse \mathsf{st} \\
    \pcreturn \time
}
\vspace{1em}
\pseudocode{
    \mathsf{StExt}(\mathsf{st}) \\[0.1\baselineskip ][\hline]
    (\mathsf{AC}_U, \mathcal{Z}_U, \time) := \parse \mathsf{st} \\
    \pcreturn \mathsf{AC}_U
}
\end{pcvstack}
\qquad
\begin{pcvstack}
\pseudocode{
    \mathsf{KGen}(\pp) \\[0.1\baselineskip ][\hline]
    \mathsf{msk} \sample\mathcal{K} \\
    \mathsf{mpk} := \Delta.\mathsf{KGen(msk)} \\
    \pcreturn (\mathsf{mpk}, \mathsf{msk})
}
\vspace{1em}
\pseudocode{
    \mathsf{TxExt}(\mathsf{tx}) \\[0.1\baselineskip ][\hline]
    (P,R,\mathsf{AC}_T, \mathcal{Z}_{\bar{S}}) := \parse \mathsf{tx} \\
    \pcreturn \mathsf{AC}_T
}\vspace{1em}
\pseudocode{
    \mathsf{ExtAccType}(\mathsf{AC}) \\[0.1\baselineskip ][\hline]
    \mathbf{assert} \not\exists \: a,b \in \{ (\mathsf{tout}_i, \mathsf{atype}_i) \}_{i \in \mathsf{AC}} \mid  a \neq b \\
    \pcreturn (\tout, \atype)
}\vspace{1em}
\pseudocode{
    \mathsf{EvalTags}(\mathsf{sks}) \\[0.1\baselineskip ][\hline]
    \pcreturn \{ \Delta.\mathsf{Eval}(\mathsf{sk}_i) \}_{i \in \mathsf{sks}}  \\
}
\end{pcvstack}
\end{pchstack}
\end{minipage}%
\end{figure}
\begin{figure}
\begin{minipage}[t]{\textwidth}
\begin{pchstack}[boxed]

\begin{pcvstack}
\pseudocode{
    \mathsf{Vf}(\mathsf{st},\mathsf{tx}) \\[0.1\baselineskip ][\hline]
    (\mathsf{AC}_U, \mathcal{Z}_U, \time) := \parse \mathsf{st} \\
    \{\mathsf{ac}_i\}_{i \in U} := \parse \mathsf{AC}_U \\
    (P,R,\mathsf{AC}_T, \mathcal{Z}_{\bar{S}}, \tout, \txtype, \pi) := \parse \mathsf{tx} \\
    \mathsf{AC}_R := \{\mathsf{ac}_i\}_{i \in R} \\
    \mathsf{stmnt} := (P,\mathsf{AC}_R,\mathsf{AC}_T,\mathcal{Z}_{\bar{S}}, \txtype, \tout) \\
    \pcif \begin{cases}
        P \in \mathcal{P} \tabularnewline
        R \subseteq U \tabularnewline
        \Pi.\mathsf{Vf}(\mathsf{crs}, \mathsf{stmnt}, \pi) = 1 \tabularnewline
        \mathcal{Z}_{\bar{S}} \cap \mathcal{Z}_{\bar{U}} = \emptyset
    \end{cases} \: \mathbf{then} \\
    \t \pcreturn (1, \mathsf{st}') \\
    \pcelse \pcreturn \: (0, \mathsf{st})
}
\vspace{1em}
\pseudocode{
    \mathsf{KDer}(\mathsf{msks},\tau) \\[0.1\baselineskip ][\hline]
    (r, \delta, \accd) := \parse \tau \\
    \mathsf{sks} := \{ \mathsf{msk}_i +\delta \}_{i\in \mathsf{msks}}\\
    \pcreturn (\mathsf{sks}, r, \accd)
}
\vspace{1em}
\pseudocode{
    \mathsf{TgtChk}(\mathsf{ac}, \accd) \\[0.1\baselineskip ][\hline]
    (\mathsf{pks}, \mathsf{co}) := \parse \mathsf{ac} \\
    (r, \delta, \accd') := \parse \mathsf{tk} \\
    \pcreturn \begin{cases}
        %\accd' \overset{?}{=} \accd \tabularnewline
        \mathsf{co} \overset{?}{=} \Gamma.\mathsf{Com}(\accd,r)
    \end{cases} 
}
\vspace{1em}
\pseudocode{
    \mathsf{SrcChk}(\mathsf{ac}, r, \mathsf{sks}, \accd, \txtype, \mathcal{Z}) \\[0.1\baselineskip ][\hline]
    (\mathsf{ssk}, \mathsf{tsk}, \mathsf{rsk}) := \parse \mathsf{sks} \\
    (\mathsf{pks}, \mathsf{co}) := \parse \mathsf{ac} \\
    (\mathsf{spk}, \mathsf{tpk}, \mathsf{rpk}) := \parse \mathsf{pks} \\
    \mathbf{assert} \: \atype = \txtype \overset{?}{\neq} 0 \\
    \mathbf{assert} \: \mathsf{co} \neq \Gamma.\mathsf{Com}(\accd, r) \\
    \pcif  \txtype = \: 0 \\
    \t \pcreturn \begin{cases} 
    	\mathsf{spk} \overset{?}{=} \Delta.\mathsf{KGen}(\mathsf{ssk}) \tabularnewline
	\Delta.\mathsf{Eval}(\mathsf{ssk}) \overset{?}{=} \mathcal{Z}
    \end{cases} \\
    \pcelse \pcif  \txtype = 1  \\
    \t \pcreturn  \begin{cases}
        \mathsf{tpk} \overset{?}{=} \Delta.\mathsf{KGen}(\mathsf{tsk}) \tabularnewline
        \mathsf{spk} \overset{?}{=} \Delta.\mathsf{KGen}(\mathsf{ssk}) \tabularnewline
	\Delta.\mathsf{Eval}(\mathsf{ssk}), \Delta.\mathsf{Eval}(\mathsf{tsk}) \overset{?}{=} \mathcal{Z}
    \end{cases} \\
    \pcelse \\
    \t \pcreturn \begin{cases} 
	\mathsf{rpk} \overset{?}{=} \Delta.\mathsf{KGen}(\mathsf{rsk}) \tabularnewline
	\Delta.\mathsf{Eval}(\mathsf{ssk}), \Delta.\mathsf{Eval}(\mathsf{rsk}) \overset{?}{=} \mathcal{Z}
    \end{cases} \\
}
\end{pcvstack}
\qquad
\begin{pcvstack}
\pseudocode{
    \mathsf{TimeVf}(\mathsf{st}, \mathsf{tx}) \\[0.1\baselineskip ][\hline]
    (\mathsf{AC}_U, \mathcal{Z}_U, \time) := \parse \mathsf{st} \\
    (P,R,\mathsf{AC}_T, \mathcal{Z}_{\bar{S}}, \tout, \txtype, \pi) := \parse \mathsf{tx} \\
    \pcif \txtype = 0 \: \lor \\
    \t (\txtype = 1 \land  \time \leq \tout) \: \lor\\
    \t (\txtype = 2 \land  \time > \tout) \\
               \t\t \pcreturn \: \mathsf{Vf}(\mathsf{st}, \mathsf{tx}) \\
    \pcreturn (0, \mathsf{st})
}
\vspace{1em}
\pseudocode{
    \mathsf{TxGen}(\mathsf{st}, P, R, \mathcal{S}, \mathcal{T}) \\[0.1\baselineskip ][\hline]
    \{\mathsf{sks}_i, \accd_i\}_{i\in S} := \parse \mathcal{S} \\
    \{\mathsf{mpks}_i, \accd'_i\}_{i\in T} := \parse \mathcal{T} \\
    \{\mathsf{ac}_i\}_{i \in U} := \mathsf{StExt(st)} \\
    \time := \mathsf{TimeExt(st)} \\
    \tout, \atype \gets \mathsf{AccTimeExt}(\{\accd_i\}_{i\in S}) \\
    \txtype := \atype \cdot (1 + \tout \overset{?}{\leq} \time) \\
    \mathbf{for} \; i \in T \: \mathbf{do} \\
    \t r_i \sample \chi \\
    \t \delta_i \sample \mathcal{K} \\
    \t \mathsf{co}_i := \Gamma.\mathsf{Com}(\accd'_i, r_i) \\
    \t (a_i, \tout, b_i) := \parse \accd'_i \\
    \t (\mathsf{smpk}_i, \mathsf{tmpk}_i, \mathsf{rmpk}_i) := \parse \mathsf{mpks}_i \\
    \t \mathsf{spk}_i := \mathsf{smpk}_i + \Delta.\mathsf{Eval}(\delta_i) \\
    \t \mathsf{tk}_i := (r_i, \delta_i, \accd'_i) \\
    \t \pcif b \neq 0 \\
    \t\t \mathsf{tpk}_i := \mathsf{tmpk}_i + \Delta.\mathsf{Eval}(\delta_i) \\
    \t\t \mathsf{rpk}_i := \mathsf{rmpk}_i + \Delta.\mathsf{Eval}(\delta_i) \\
    \t\t \mathsf{pks}_i := (\mathsf{spk}_i, \mathsf{tpk}_i, \mathsf{rpk}_i) \\
    \t \pcelse \\
    \t\t \mathsf{pks}_i := (\mathsf{spk}_i, \perp, \perp) \\
    \t \mathsf{ac}'_i := (\mathsf{pks}_i, \mathsf{co}'_i) \\
    \mathsf{AC}_R := \{\mathsf{ac}_i\}_{i \in R} \\
    \mathsf{AC}_T := \{\mathsf{ac}'_i\}_{i \in T} \\
    \mathcal{Z}_{\bar{S}} := \mathsf{EvalTags}(\mathsf{sks}_i)_{i \in S} \\
    \mathsf{stmnt} := (P,\mathsf{AC}_R,\mathsf{AC}_T,\mathcal{Z}_{\bar{C}}, \tout, \txtype) \\
    \mathsf{wit} := ((r,\mathsf{sks}_i, \accd_i)_{i\in S}), (\mathsf{mpks}_i, \accd'_i)_{i\in T}) \\
    \pi \gets \Pi.\mathsf{Prove}(\mathsf{crs},\mathsf{stmnt},\mathsf{wit}) \\
    \mathsf{tx} := (P,\mathsf{AC}_R,\mathsf{AC}_T,\mathcal{Z}_{\bar{C}}, \tout, \txtype, \pi) \\
    \mathsf{TK} := {\mathsf{tk}_i}_{i \in T} \\
    \pcreturn (\mathsf{tx}, \mathsf{TK})
}
\end{pcvstack}
\end{pchstack}
\end{minipage}%
\end{figure}
\newpage

% TeX root = atomic-swaps.tex

\section{Instantiation and Performance Evaluation}

\subsection{Instantiation}

\paragraph{Commitment}

\paragraph{Tagging Scheme}

\paragraph{2PC}

\begin{todobox}
    Write down explicitly (i.e. in terms of $\GG$ and $\ZZ_q$ arithmetic) for which functionalities do we need 2PCs.     
\end{todobox}


$[M_0] := \begin{pmatrix}
[1] & 0 & 0 & -[s] \\
0 & [1] & 0 & -[t] \\
0 & 0 & [1] & -[r]
\end{pmatrix} $
$w_1 := \begin{bmatrix} s \\ t \\ r \\ 1 \\ \end{bmatrix}$
$[w_2] := \begin{pmatrix}
1-b & 0 & 0 \\
0 & c(1-b) & 0 \\
0 & 0 & b \\
\end{pmatrix}$
$[0] :=
\begin{pmatrix}
0 & 0 & 0 \\
0 & 0 & 0 \\
0 & 0 & 0 \\
\end{pmatrix} $
$([M_0]w_1)^T[w_2] = [0]$

\paragraph{Zero-Knowledge Proofs}

\subsection{Performance Evaluation}

\end{document}

\bibliographystyle{alpha}
\bibliography{cryptobib/abbrev3,cryptobib/crypto,references}

\end{document}