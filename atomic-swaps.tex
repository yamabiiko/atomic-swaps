\documentclass{article}      	% Style of the document                     
\usepackage{fullpage}
\usepackage{amsmath}     	   	% Maths                                          
\usepackage[utf8]{inputenc}	% UTF-8 characters                                               
\usepackage[T1]{fontenc}    	% Tuki ääkkösille (Finnish names don't cause problems)                                            
\usepackage{parskip}        		% Linebreak between paragraphs                
\usepackage{graphicx}       		% Graphics package for adding figures                        
\usepackage{epstopdf}       		% Possibility to add *.eps figures
 \usepackage{ dsfont }            % Symbol for real numbers
\usepackage{hyperref}
\usepackage{extarrows}
\usepackage{float}
\usepackage{makeidx}
\usepackage{enumitem}        % possibility to label list items by alphabet
\usepackage[a4paper, top=0.5in]{geometry}
\newcommand{\M}[1]{\ensuremath{\text{\texttt{#1}}}}
\usepackage[
    lambda,
    operators,
    advantage,
    sets,
    adversary,
    landau,
    probability,
    notions,
    logic,
    ff,
    mm,
    primitives,
    events,
    complexity,
    asymptotics,
    keys]{cryptocode}

\usepackage{todonotes}

 \usepackage{amsmath,amsfonts,graphicx,amssymb,amsthm}
\mathchardef\mhyphen="2D

 %% general
\mathchardef\mhyphen="2D
\newcommand{\fdv}{\mathcal{F}}
\newcommand{\tdv}{\mathcal{T}}
\newcommand{\vdv}{\mathcal{V}}
\newcommand{\cX}{\mathcal{X}}
\newcommand{\cF}{\mathcal{F}}
\newcommand{\cG}{\mathcal{G}}
\newcommand{\ID}{\mathcal{I}}
\newcommand{\bits}[1][]{\{0,1\}^{#1}}
\renewcommand{\vec}[1]{\mathbf{#1}}
\newcommand{\mat}[1]{\mathbf{#1}}
\newcommand{\inner}[2]{\langle #1, #2 \rangle}
\newcommand{\transpose}{\mathtt{T}}
\newcommand{\round}[1]{\lfloor #1 \rceil}
\renewcommand{\dist}{\mathsf{dist}}
\renewcommand{\Pr}[2][]{{\text{Pr}_{#1}\left[#2\right]}}
\newcommand{\Exp}[2][]{{\mathbb{E}_{#1}\left[#2\right]}}
\newcommand{\mathcm}[2][1cm]{\hspace{#1}{\mbox{/\!\!/ } \text{\scriptsize#2}}}

%% lattice problems
\newcommand{\SIS}{\mathsf{SIS}}
\newcommand{\ISIS}{\mathsf{ISIS}}
\newcommand{\nfSIS}{\mathsf{nfSIS}}
\newcommand{\dSIS}{\mathsf{dSIS}}
\newcommand{\LWE}{\mathsf{LWE}}
\newcommand{\nfLWE}{\mathsf{nfLWE}}
\newcommand{\nfdLWE}{\mathsf{nfdLWE}}
\newcommand{\sLWE}{\mathsf{sLWE}}
\newcommand{\dLWE}{\mathsf{dLWE}}
\newcommand{\SVP}{\mathsf{SVP}}
\newcommand{\CVP}{\mathsf{CVP}}
\newcommand{\SIVP}{\mathsf{SIVP}}
\newcommand{\GapSVP}{\mathsf{GapSVP}}
\newcommand{\BDD}{\mathsf{BDD}}
\newcommand{\NTRU}{\mathsf{NTRU}}
\newcommand{\sNTRU}{\mathsf{sNTRU}}
\newcommand{\dNTRU}{\mathsf{dNTRU}}

%% lattice macros
\newcommand{\TT}{\mathbb{T}}
\newcommand{\ring}{\mathcal{R}}
\newcommand{\lattice}{\mathcal{L}}
\newcommand{\piped}{\mathcal{P}}
\newcommand{\ball}{\mathcal{B}}
\newcommand{\Hyb}{\mathsf{Hyb}}
\newcommand{\lspan}{\mathsf{span}}
\newcommand{\rank}{\mathsf{rank}}
\newcommand{\lsb}{\mathsf{LSB}}
\newcommand{\pubparam}{\mathsf{pp}}

%% group macros

%% syntax
\newcommand{\mpk}{\mathsf{mpk}}
\newcommand{\msk}{\mathsf{msk}}
\newcommand{\msg}{\mathsf{msg}}
\newcommand{\rnd}{\mathsf{rnd}}
\newcommand{\ctxt}{\mathsf{ctxt}}
\newcommand{\com}{\mathsf{com}}
\newcommand{\td}{\mathsf{td}}
\newcommand{\id}{\mathsf{id}}
\newcommand{\stmt}{\mathsf{stmt}}
\newcommand{\wit}{\mathsf{wit}}
\newcommand{\tx}{\mathsf{tx}}
\newcommand{\aux}{\mathsf{aux}}
\newcommand{\ek}{\mathsf{ek}}

\newcommand{\Setup}{\mathsf{Setup}}
\newcommand{\Commit}{\mathsf{Com}}
\newcommand{\TrapGen}{\mathsf{TrapGen}}
\newcommand{\SampD}{\mathsf{SampD}}
\newcommand{\SampPre}{\mathsf{SampPre}}
\newcommand{\Prove}{\mathsf{Prove}}
\newcommand{\Verify}{\mathsf{Verify}}
\newcommand{\val}{\mathsf{val}}

%% primitive/scheme name
\newcommand{\PKE}{\mathsf{PKE}}
\newcommand{\LTDF}{\mathsf{LTDF}}
\newcommand{\rsagen}{\mathsf{RSAGen}}
\newcommand{\rsa}{\mathsf{RSA}}
\newcommand{\LHE}{\mathsf{LHE}}
\newcommand{\CS}{\mathcal{CS}}
\newcommand{\NTRUEncrypt}{\mathsf{NTRUEncrypt}}

%% others
\newcommand{\oracle}{\mathcal{O}}
\newcommand{\pcas}{~\mathbf{as}~}

\newcommand{\polylog}[1][\secpar]{\mathsf{polylog}(#1)}

\newcommand{\indrsidcpa}{\mathrm{IND\$}\mhyphen\mathrm{sID}\mhyphen\mathrm{CPA}}
%\newcommand{\oplus}{\, \texttt{XOR} \,} % shorthand for typing the XOR operator in mathmode
\usepackage{tikz}
\usetikzlibrary{decorations.pathreplacing}
\usetikzlibrary{decorations.pathmorphing}


\definecolor{cgreen}{RGB}{0, 153, 51}
\definecolor{cblue}{RGB}{0, 102, 204}
\definecolor{cyellow}{RGB}{255, 204, 0} 
\definecolor{cred}{RGB}{204, 51, 0} 

\newcommand{\commentline}[2]{%
    \tikz[remember picture, overlay]{
        \node [black,anchor=west,xshift=10pt] at (#1) {#2};
    }
}

\newcommand{\blockcomment}[3]{%
    \tikz[remember picture, overlay]{
        \draw [decorate,decoration={lineto,amplitude=10pt,mirror,raise=4pt},yshift=0pt,very thick,{#3}] 
        (#1) -- (#2) node [black,midway,xshift=10pt] {};
    }
}



\usepackage{biblatex}
\addbibresource{references.bib}

\begin{document}         
\author{Lorenzo Tucci}
\title{Atomic swaps from timed commitments}

\maketitle

\tableofcontents
\section{Introduction}

Blockchains are distributed ledgers managed by peer-to-peer computer networks by nodes, which they collectevely adhere to a consensus protocol to add and validate record of transactions in a decentralized, permissionless and trustless manner. \\
Given the increasing number of blockchain systems, one of the most critical features in the cryptocurrency landscape is blockchain interopability, particularly the ability to exchange heterogeneous assets between different ledgers while maintining the trustless and decentralized properties of the individual chains. \\

An atomic cross-chain swap is an exchange of assets acrtoss different blockchains between two users without any additional trust assumption. \\
Consider two ledgers $\mathbb{A}$ and $\mathbb{B}$, where Alice holds assets $a$ in $\mathbb{A}$ and Bob holds assets beta in $\mathbb{B}$. We want to ensure that Alice transfers her assets $a$ in $\mathbb{B}$ to Bob if and only if Bob transfers his asset $b$ to Alice in $\mathbb{B}$. \\


\todo[inline]{TODO starts}
\begin{itemize}
    \item General description of problem (swapping), motivations, etc.
    \item Introduce Universal Atomic Swaps (UAS) and its building blocks, briefly discuss how the protocol works, and highlight the issue of both parties needing to solve a VTS
    \item The ``our contribution'' subsection: e.g. ``We propose an alternative construction ... we provide an open source implementation ...''
    \item (Not so) Related work (subsection or section): Other ways to get atomic swaps, e.g. TEE, relays, HTLC, ...
    \item \url{https://ieeexplore.ieee.org/stamp/stamp.jsp?tp=&arnumber=9797357&tag=1} provides an alternative construction of atomic swap without a trusted setup but requiring special chain verification logic. To capture this, we want to abstract away the time-lock functionality so that it can be implemented in different ways, e.g. with TLP, custom chain logic, or new ways like commit transactions.
    \item Add comments to pseudocode. Split chunks of the protocol to be run in parallel into subroutines. Give more descriptive variable names (e.g. instead of $P_0, P_1, T_0, T_1$)
\end{itemize}

\todo[inline]{TODO ends}

\subsection{Existing solutions}

%\begin{table}[H]
%\centering
%\begin{tabular}{|c|c|c|c|}
%\hline
%\textbf{Type} & \textbf{Trustless} & \textbf{Transparent} & \textbf{Scriptless} \\
%\hline
%TEE [BJZLZBDJ'17] & & \checkmark & \checkmark \\
%\hline
%HTLC [Gugger '20] & \checkmark & &\\
%\hline
%Relays [LMP'21] & \checkmark & &  \\
%\hline
%Universal [MTS'20] & \checkmark & \checkmark & \checkmark \\
%\hline
%\end{tabular}
%\caption{Properties comparison of different atomic swaps protocol}
%\end{table}
%\subsubsection{Hashed Timelock Contract (HTLC)}

\subsubsection{Hash Time Lock Contracts}

By utilizing a blockchain's scripting language and timelock functionalities it is possible to create a script that acts as a timed escrow, which can be claimed if certain condition are met or refunded upon expiration.
Specifically, a Hash Time Lock Contract (HTLC) is defined by a tuple $(\mathsf{amnt_a}, h, T, \mathsf{pk_0}, \mathsf{pk_1})$ where 
\begin{itemize}
	\item $\mathsf{amnt_a}$ denotes the amount of $\mathsf{a}$ assets to be exchanged
	\item $h$ is a hash value
	\item $T$ the timeout
	\item $\mathsf{pk_{tx}}$ and $\mathsf{pk_{rx}}$ the public key addresses of two users
\end{itemize}

The HTLC transfers $\mathsf{amnt_a}$ to $\mathsf{pk_1}$ if invoked before timeout $T$ with input value $r$ such that $\mathcal{H}(r) = h$. 
If the contract is invoked after timeout $T$, it refunds the assets $\mathsf{amnt_a}$ to $\mathsf{pk_0}$ unconditionally.

Using HTLCs as a building block, an atomic swap protocol can be constructed as follows: \\
1) Alice chooses $r$, computes $h = \mathcal{H}(r)$, transfers $\mathsf{amnt_a}$ into an $(\mathsf{amnt_a}, h, T_0, \mathsf{pk_0}, \mathsf{pk_1})$ on blockchain $\mathbb{A}$ and sends $h,T$ to Bob. \\
2) Bob finishes the setup of the exchange by choosing a time $T_1 < T_0$ and transferring his $\mathsf{amnt_b}$ assets into an HTLC$(\mathsf{amnt_b}, h, T_1, \mathsf{pk_{tx}}, \mathsf{pk_{rx}})$ on blockchain $\mathbb{B}$. 

Bob cannot claim the HTLC yet as $r$ is only known by Alice, thus there are only two possible outcomes:
\begin{itemize}
\item Alice claims the HTLC on $\mathbb{B}$ before $T_0$, effectively revealing $r$ to bob (and anyone observing $\mathbb{B}$), Bob can then proceed to compute $h = \mathcal{H}(r)$ to claim the counterpart HTLC on chain $\mathbb{A}$ 
\item Alice does not claim the HTLC on $\mathbb{B}$ in time and thus Bob cannot claim the HTLC on $\mathbb{A}$. After the respective timeouts they can refund the assets by invoking the contracts.
\end{itemize}

This functionality can be achieved without requiring complex scripting functionality \cite{h4sh3d} using semi-scriptless scripts. \\
Additionally, a similar protocol can also be realized even if only one blockchain supports HTLCs or timelocks, as done in \cite{h4sh3d} in the Monero counterpart.

%\todo[inline]{explain (splitting secret key, BTC party needs to move first)}

%\subsubsection{Relays}
%Another strategy to achieve atomic swaps relies on relays. Relays are abstractions (in general a smart contract or a script) hosted on some
%chain $\mathbb{B}_a$ that has light client like verification capabilities over chain $\mathbb{B}_b$. For each new block appended to chain $\mathbb{B}_a$,
%the block header is passed on to the relay on chain $\mathbb{B}_b$. \\
%The relay itself implements the standard verification procedure of chain $\mathbb{B}_a$’s consensus algorithm and can therefore verify the
%validity of the block. Once the proof of work has been verified,
%in the case of a Proof of Work (or PoW) blockchain, or the
%two-thirds of validators signatures, in the case of a Byzantine
%Fault Tolerant (or BFT) blockchain, it is possible to verify any
%transaction of chain $\mathbb{B}_a$ from chain $\mathbb{B}_b$. With light client
%like verification capabilities of chain $\mathbb{B}_a$ from chain $\mathbb{B}_b$,
%we can imagine the following scenario. Bob has X assets of
%chain $\mathbb{B}_b$. He is willing to exchange them for Y assets of
%chain $\mathbb{B}_a$.  \\
%    Bob sets up a smart contract SC1 and locks his
%assets in it (1). This smart contract SC1 is set to release the
%assets to anyone providing the proof that they made a payment
%of Y assets of chain $\mathbb{B}_a$ to Bob’s address. Alice, who is
%interested in this trade, transfers Y assets to Bob’s address (2).
%She retrieves the transaction hash tx and provide it to SC1 (3).
%SC1 calls the relay and asks for verification of transaction tx
%(4). The relay verifies that the transfer has taken place and if
%so, returns ok to SC1 (5). On receiving the answer from the
%relay, SC1 transfers the X assets of $\mathbb{B}_b$ to Alice’s address.

%\begin{figure}[H]
%\begin{pchstack}[center, boxed]
%\pseudocode{
%    \textbf{$P_0(pk(0)\:, sk(0)$)} \< \< \textbf{$P_1(pk(1)\:, sk(1))$} \\[0.1\baselineskip ][\hline] 
%    \<\< \\[-0.5\baselineskip ]
%    \< \sendmessage*{<->}{top={{\Gamma_{\mathsf{KeyGen}}(\mathbb{G},G,q)}}, bottom={\xlonggets{} (sk_0(01), pk(01)) \\ (sk_1(01), pk(01)) \xlongrightarrow{} }} \< \\
%    \<\< (C, \pi) \gets \Pi_{\mathsf{VTD}}.\mathsf{Commit}(sk_1, T) \\
%    \< \sendmessageleft*{(C, \pi)} \< \\
%    \mathsf{starts}\: \mathsf{Timeout}(T - \Delta)
%    \<\< \mathsf{starts}\: \mathsf{Timeout}(T - \Delta) \\
%    \mathsf{if}\: \Pi_{\mathsf{VTD}}.\mathsf{Verify}(pk, C, \pi) \neq 1 \\
%    \qquad \mathsf{abort}\\
%    tx_\mathsf{frz} \gets \mathsf{InitTx}(pk(0), pk(01), \mathsf{swp(a)}, \mathbb{A}) \\
%    \sigma_{\mathsf{frz}} \gets \Pi_{\mathsf{DS}}.\mathsf{Sign}(sk(0), tx_\mathsf{frz}) \\
%    \mathsf{PubTx}(\sigma_{\mathsf{frz}}, tx_\mathsf{frz}, \mathbb{A}) \\
%    \mathsf{starts}\: \Pi_{\mathsf{VTD}}.\mathsf{ForceOp}(C) \\
%    \<\< \mathsf{do}\: \mathsf{bal} \gets \mathsf{GetBal}(pk(01), \mathbb{A}) \\
%    \<\< \mathsf{while}\: \mathsf{bal} \neq \mathsf{swp(a)}\\
%    \< \sendmessageleft*{pk(1)} \< \\
%    (pk(10), sk(10)) \gets \Pi_{\mathsf{DS}}.\mathsf{KeyGen}(1^\lambda) \\
%    tx_\mathsf{swp} \gets \mathsf{InitTx}(pk(1), pk(10), \mathsf{swp(b)}, \mathbb{A}) \\
%    \< \sendmessage*{<->}{top={{\Gamma_{\mathsf{Swap}} \qquad \qquad \\ P_0 \xlongrightarrow{} (sk_0(01), tx_\mathsf{swp}) \\ (sk_1(01), sk(1)) \xlonggets{} P_1 }}, bottom={lk := \sigma_{swp}(10) \oplus sk_0(01) \xlongrightarrow{} \\  \xlongleftarrow{} \sigma_{swp}(10) \qquad \qquad  }} \< \\
%    \mathsf{PubTx}(\sigma_{\mathsf{swp(10)}}, tx_\mathsf{frz}, \mathbb{A}) \\
%    \<\< \mathsf{do}\: \sigma_{swp}(10) \gets \mathsf{GetSig}(pk(1), \mathbb{B}) \\
%    \<\< \qquad sk(01) \gets (lk \oplus \sigma_{swp}(10)) + sk_1 \\
%    \<\< \qquad \sigma_{m} \gets \Pi_{\mathsf{DS}}.\mathsf{Sign}(sk(01), 1) \\
%    \<\< \mathsf{while}\: \Pi_{\mathsf{DS}}.\mathsf{Verify}(m, pk, \sigma_{m}) \neq 1 \\
%}
%\end{pchstack}
%\caption{Universal atomic swap protocol execution for a successful swap}
%\end{figure}
%

\subsubsection{Universal Atomic Swaps}


While HTLCs can be an effective solution when supported, they are infeasible to realize when both blockchains lack a timelock functionality. Furthermore, transactions incure in higher execution costs due to scripting, and generate additional on-chain data that can potentially compromise privacy.

Thyagarajan et al. \cite{uas}, introduced one of the first atomic swap protocols substituting blockchain timelocks with a cryptographic primitive, utilizing Verifiable Timed Signatures (VTS) \cite{vts} as the core building block.

A VTS lets a user generate a timed commitment $C$ of a signature $\sigma$ on a message $m$ under a public key $\mathsf{pk}$. The commitment $C $ must hide the signature $\sigma$ for time $\mathsf{T}$ and producing a proof $\pi$ that $C$ contains a valid signature $\sigma$. This ensures that $\sigma$ can be publicly recovered in time $\mathsf{T}$ by anyone who solves the computational puzzle.

Let $P_0$ and $P_1$ be two parties where $P_0$ wants to exchange $\mathsf{amnt_a}$ on blockchain $\mathbb{A}$ from their address $\mathsf{pk_{init(\mathbb{A})}}$ for $\mathsf{amnt_b}$ on blockchain $\mathbb{B}$ to $\mathsf{pk_{swp(\mathbb{B})}}$ and vice-versa for $P_1$.

In the setup phase of the protocol, the parties run a 2PC protocol to setup two freeze address on the respective chains $\mathsf{pk_{frz(\mathbb{A})}}$ and $\mathsf{pk_{frz(\mathbb{B})}}$, where each party posseses one share of the respective secret keys, e.g. $\mathsf{sk_{frz(\mathbb{A})}} := \mathsf{sk_{frz0}} \oplus  \mathsf{sk_{frz1}}$. \\
Now the parties create a refund transaction transferring back the assets in case of timeout, for $P_0$ we have $\mathsf{tx_{rfnd(\mathbb{A})}}$ refunding $\mathsf{amnt_a}$ from $\mathsf{pk_{frz(\mathbb{A})}}$ to $\mathsf{pk_{init(\mathbb{A})}}$ and similarly for $P_1$ $\mathsf{tx_{rfnd(\mathbb{B})}}$. \\
Each party generates a timed commitment on the signature of the counterparty's refund transaction, where $P_0$ receives a $\mathsf{VTS}$ with commitment $C_0$ and timeout $T_0 = T_1 + \Delta$ and $P_1$ receives a $\mathsf{VTS}$ with commitment $C_1$ and timeout $T_1$. Once both $\mathsf{VTS}$ commitment are verified the parties proceed to transfer the assets to the freeze addresses, assured to retrieve the refund transaction signatures after force opening the commitments following the specified timeouts. \\

In the subsequential lock phase, parties first initialize the swap transactions $\mathsf{tx_{swp}}$ transferring $\mathsf{amnt}$ from $\mathsf{pk_{frz}}$ to $\mathsf{pk_{swp}}$ for the respective chains. They then compute via 2PC $\mathsf{lk} := \sigma_{\mathsf{swp}(\mathbb{A})} \oplus \mathcal{H}(\sigma_{\mathsf{swp}(\mathbb{B})})$, where  $P_0$ receives $\sigma_{\mathsf{swp}(\mathbb{B})}$ and $P_1$ receives $\mathsf{lk}$.  When $P_0$ publishes $\mathsf{tx_{swp(\mathbb{B})}}$ together with  $\sigma_{\mathsf{swp}(\mathbb{B})}$, $P_1$ can unmask $\mathsf{lk}$ by computing $\mathcal{H}(\sigma_{\mathsf{swp}(\mathbb{B})})$ to retrieve $\sigma_{\mathsf{swp}(\mathbb{A})}$ and publish $\mathsf{tx_{swp(\mathbb{A})}}$.

 If $P_0$ fails to publish $\mathsf{tx_{swp(\mathbb{B})}}$ before $T_1$, $P_1$ will publish the refund transaction $\mathsf{tx_{rfnd(\mathbb{B})}}$ and similarly for $P_0$ if $P_1$ timeouts during the protocol execution.

However that the protocol fails to take into account two possible racing conditions: \\
1) $P_0$ can wait until $T_1$ to post $\mathsf{tx_{swp(\mathbb{B})}}$, concurrently with $P_1$ posting the refund transaction $\mathsf{tx_{rfnd(\mathbb{B})}}$. Since $\mathsf{tx_{rfnd(\mathbb{B})}}$ is not confirmed until $\mathbb{B}.\mathsf{ctime}$, there is an equal probability that $\mathsf{tx_{swp(\mathbb{B})}}$ will succeed, preventing $P_1$ from refunding the assets. \\ 
2) Similarly, if $P_0$ posts $\mathsf{tx_{swp(\mathbb{B})}}$ after $T_1 - \mathbb{B}.\mathsf{ctime}$, $P_1$ can simultaneously post $\mathsf{tx_{rfnd(\mathbb{B})}}$ and $\mathsf{tx_{swp(\mathbb{A})}}$, which can result in $P_1$ obtaining both $\mathsf{amnt_a}$ and $\mathsf{amnt_b}$. \\

To preserve the atomicity property, we propose the following changes: \\
1) When $P_1$ timeouts at $T_1$, the party should check if $P_0$ is publishing $\mathsf{tx_{swp(\mathbb{B})}}$ while $\mathsf{tx_{rfnd(\mathbb{B})}}$ is unconfirmed. If so, retrieve $\sigma_{swp(\mathbb{A})}$ and publish $\mathsf{tx_{swp(\mathbb{A})}}$. It is also required that $\Delta > \mathbb{A}.\mathsf{ctime} + \mathbb{B}.\mathsf{ctime}$ \\
2) Require $P_0$ to post $\mathsf{tx_{swp(\mathbb{B})}}$ no later than $T_1 - \mathbb{B}.\mathsf{ctime}$, and abort otherwise 

Note that parties must also take into account potential differences in the computational power available for force opening the VTS commitments. This prevent scenarios where one party force opens its VTS commitments earlier than expected, potentially stealing 
 the other party's assets during the swap lock or complete phase. Therefore,  $\Delta$ (such that T0 = T1 + $\Delta$) must be large enough to tolerate time differences in opening the VTS commitments. \\

Parties are incentivized to make worst-case estimates for the counterparty's computational power which can be orders of magnitude faster when accounting for novel algorithms \cite{squaring_algo} and application-specific integrated circuits (ASIC) \cite{squaring_asic}. \\
This results in a potentially impractical protocol, as parties may need to uninterruptedly compute a force opening for a significant amount of time.

\subsection{Our contribution}

We propose a new atomic swap protocol leveraging timed cryptographic commitments, using  Verifiable Timed Discretelog (VTD) as core building block. This protocol addresses potential differences in computation power when setting timelocks by taking advantage of an asymmetric design. \\

\subsubsection*{Asymmetric design}
\textbf{Exchange server}: one party, acting as an exchange server, is responsible for force opening a refund timed commitment with timelock $T_0$ in case of a protocol timeout. This centralizes the computation burden and reduces the requirments on the client side. \\
\textbf{Client}: the other party, acting as the client , is only required to compute a timed commitment $T_2$ to enforce a swap if the server becomes uncooperative at the end of protocol. This division minimizes computing power requirements and computations performed in a protocol run.

\subsubsection*{Improved abort}
The client has the flexibility to safely abort from the protocol by posting the refund transaction at any time before $T_0 - \mathbb{B}.\mathsf{ctime}$, without having to force open any timed commitment. This ensures that a client can safely reclaim their assets in case of a protocol timeout without having to force open a timed commitment.

\subsubsection*{Flexible time parameters}

The time parameter $T_0$ can be extended independently of the client's fixed timed commitment $T_2$, allowing for better adaptability to network conditions and computational capabilities. \\
The client is required to compute, at worst, a timed commitment for $T_2 = 2\mathbb{B}.\mathsf{ctime} + \mathbb{A}.\mathsf{ctime}$.

\subsubsection*{Scriptless and fungibility}
The protocol is scriptless, resulting in lower transaction fees and faster processing times compared to traditional HTLC-based swaps. \\
Furthermore, the protocol is \textit{fungible}, as it does not add any additional on-chain data on the transaction, improving the privacy of the swap protocol.

\section{Preliminaries}

\todo[inline]{
Put the basic definitions here, e.g. signatures, hash functions, commitment, blockchains (maybe for this report just do it informally), etc.}

\subsection{Notation}

\begin{itemize}[nosep, noitemsep]
    \item $\mathbf{wait} \: \mathsf{fn}()$ - waits until the function $\mathsf{fn}()$ execution is complete. If $\mathsf{fn}()$ returns $\perp$, then it causes to abort the execution block returning $\perp$. \\
    \item $\mathbf{wait} \: \{...\}$ - enforces $\mathbf{wait}$ to all routines calls inside the block. If the block $\{...\}$ returns $\perp$, then it causes to abort the execution block returning $\perp$. Returns the value of the last evaluated routine. \\
    \item $\mathbf{select} \: \{...\}$ - concurrently runs all the $\mathbf{wait}$ routines in the block and returns the value of the first routine that successfully returns a value or $\perp$ if all the $\mathbf{wait}$ routines aborted with $\perp$. When a $\mathbf{wait}$ routine aborts returning $\perp$, it waits until execution of another $\mathbf{wait}$ is successfully completed. \\
\end{itemize}

In the protocol definition, variables and functions that are blockchain-specific (unless clear from context) are denoted with a subscript, example for a publick key on chain $\mathbb{B}$ we denote $\mathsf{pk_{(\mathbb{B})}}$. \\

\subsection{Blockchain interface}

We define the following routines to interact with the blockchains. The subscript $\mathbb{A}$ indicates that we are interacting with chain $\mathbb{A}$.

\begin{itemize}[topsep=0pt, itemsep=0pt, leftmargin=2em]
    \item $\mathbf{0/1} \gets \mathbf{PubTx}_{(\mathbb{A})}(\sigma_{\mathsf{tx}}, \mathsf{tx})$: publish the transaction $\mathsf{tx}$ with signature $\sigma_{\mathsf{tx}}$. Outputs 1 if the transaction is accepted, 0 otherwise.
    \item $\mathbf{tx}_{\mathbb{A}} := (\mathsf{pk_{tx}}, \mathsf{pk_{rx}}, \mathsf{amnt}, \mathsf{id})  \gets \mathbf{InitTx}_{(\mathbb{A})}(\mathsf{pk_{tx}}, \mathsf{pk_{rx}}, \mathsf{amnt})$: creates an unsigned transaction with the unique identifier $\mathsf{id}$ paying $\mathsf{amnt}$ from $\mathsf{pk_{tx}}$ to $\mathsf{pk_{rx}}$.
    \item $\mathbf{0/1} \gets \mathbf{VerifyTx}_{(\mathbb{A})}(\sigma_{\mathsf{tx}}, \mathsf{tx})$: verifies the signature of the transaction and the validity of the transaction based on consensus rules. Outputs 1 if the verification succeeds, 0 otherwise.
    \item $\mathbf{0/1} \gets \mathbf{WatchTx}_{(\mathbb{A})}(\mathsf{tx})$: wait for the transaction $\mathsf{tx}$ to be confirmed.
    \item $\mathbf{amnt} \gets \mathbf{GetBal}_{(\mathbb{A})}(\mathsf{pk})$: get the balance of assets held by $\mathsf{pk}$
    \item $(\sigma_{\mathsf{tx}}, \mathsf{tx}) \gets \mathbf{GetLatestTx}_{(\mathbb{A})}(\mathsf{pk})$: get the the latest confirmed transaction and its signature from $\mathsf{pk}$'s record on the chain.
    \item $(\mathbf{tx}_{\mathbb{A}}, \mathbf{0/1})  \gets \mathbf{FullTx}_{(\mathbb{A})}(\mathsf{pk_{tx}}, \mathsf{pk_{rx}}, \mathsf{amnt}, \mathsf{sk_{tx}})$: an helper routine that initializes a transaction, signs it with the given secret key and publishes it to the chain. Returns the transaction $\mathbf{tx}_{\mathbb{A}}$ and 1 if the transaction is accepted, 0 otherise.
\end{itemize}

\subsection{Verifiable Timed Dlog}
A $\mathsf{VTD}$ for the group $\mathbb{G}$ with generator $G$ and order $q$ is a tuple of four algorithms ($\mathsf{Commit}, \mathsf{Verify}, \mathsf{Open}, \mathsf{ForceOp}$) where:
\begin{itemize}
	\item $(C, \pi) \gets \mathsf{Commit}(x, \textbf{T}$ ): the commit algorithm (randomized) takes as input a discrete log value $x \in \mathbb{Z}_q$ generated using $\mathsf{KGen}(1^{\lambda}$ ) and a hiding time $\textbf{T}$ and outputs a commitment $C$ and a proof $\pi$
	\item $0/1 \gets \mathsf{Verify}(H, C, \pi)$ : the verify algorithm takes as input a group element $H$ , a commitment $C$ of hardness $\textbf{T}$ and a proof $\pi$ and outputs 1 if and only if, the value $x$ embedded in $C$ satisfies $H = G^x$ . Otherwise it outputs 0.
	\item $(x, r) \gets \mathsf{Open(C)}$ : the open algorithm (run by committer) takes as input a commitment $C$ and outputs the committed value $x$ and the randomness $r$ used in generating $C$ .
	\item $x \gets \mathsf{ForceOp}(C)$ : the force open algorithm takes as input the commitment $C$ and outputs the committed value $x$ and the randomness $r$ used in generating $C$
\end{itemize}


A valid VTD scheme must adhere to the following security requirements: \\
\textbf{soundness}: a prover can compute a valid proof for a false statement with negligible probability. Thus a verifier is convinced that, given C, the ForceOp algorithm will produce the committed dlog value $x$ in time \textbf{T} \\
\textbf{privacy}: all PRAM algorithms whose running time is at most $t$ (where $t < \textbf{T}$ ) succeed in extracting $x$ from the commitment $C$ and $\pi$ with at most negligible probability. \\


\section{Atomic swaps from timed commitments}

%\todo[inline]{For the current report, put a formal syntax, but maybe informal definitions of correctness and security. Especially, there should be an (informal) definition of commit transaction. }




%\todo[inline]{Set things up (e.g. let ... be a signature scheme, ... be a ... scheme. We construct a ... in Figure ...)}



Let $T_0$, $T_1$, $T_2$ be three timeouts where  $T_0 > 2\cdot T_2$, $T_1 =  \mathbb{B}.\mathsf{ctime}$ and $T_2 \geq \mathbb{A}.\mathsf{ctime} + \mathbb{B}.\mathsf{ctime}$. 

We denote with $\mathbb{A}$ and $\mathbb{B}$ two different blockchains, where $P_0$ wants to swap $\mathsf{amnt_a}$ assets from $\mathsf{pk_{init(\mathbb{A})}}$ for $\mathsf{amnt_b}$ to $\mathsf{pk_{swp(\mathbb{B})}}$ and respectively $P_1$ wants to swap $\mathsf{amnt_b}$ assets from $\mathsf{pk_{init(\mathbb{B})}}$ for $\mathsf{amnt_a}$ to $\mathsf{pk_{swp(\mathbb{A})}}$.



%\textbf{Definition 1} (encrypted Verifiable Timed Discrelog) \\
%Public parameters $(\mathbb{G}, G, T)$ \\
%- $(eC, C, d, \pi) \gets \mathsf{Commit}(x, y, \textbf{T}$): the commit algorithm (randomized) takes as input a discrete log value $x \in \mathbb{Z}_q$, a hiding time $\textbf{T}$ and an encryption key $ y \in \mathbb{Z}_q$ and outputs a commitment $eC$ containing the encrypted puzzles, a commitment $C$ of $y$ with opening $d$ and a proof $\pi$. \\
%- $(0,1) \gets \mathsf{Verify}(H, eC, C, \pi)$: returns 1 if and only if the value $x$ embedded in $C$ decrypts $eC$ such that the value $y$ in decrypted $eC$ satisfies $H = G^y$  \\ \\
%- $(y, d) \gets \mathsf{ForceOp}(eC, x)$: the force open algorithm takes as input the encrypted commitment $eC$ and the encryption key $x$, outputs the committed value $y$ and the randomness $d$ used in generating $eC$

\[
    \mathcal{L}_{\mathsf{lock}} := \left\{\begin{array}{lr}  \mathsf{stmt} = (lk, y, h, \pi, T) : \exists (x) \:\: \text{s.t} \\
    (\mathcal{H}(x) = h \land \Pi_\mathsf{VTD}.\mathsf{Verify}(y, (lk \oplus x), \pi, T)) \end{array}\right\}
\]

\begin{figure}[H]
\begin{pchstack}[center, boxed]
\vspace{-1cm}
\pseudocode{
	P_0(\mathsf{sk_{frz0(\mathbb{B})}}) \< \< P_1(\mathsf{sk_{frz1(\mathbb{B})}}, \mathsf{tx_{rfnd}}) \\[0.1\baselineskip ][\hline] 
    \<\< \\[-0.5\baselineskip ]
    \mathsf{sk_{frz}} := \mathsf{sk_{frz0}} + \mathsf{sk_{frz1}} \\
    \sigma_\mathsf{rfnd(\mathbb{B})} \gets \Pi_{\mathsf{DS}}.\mathsf{Sign_{(\mathbb{B})}}(\mathsf{sk_{frz}}, \mathsf{tx_{rfnd}}) \\
    \mathsf{hsig} := \mathcal{H}(\sigma_\mathsf{rfnd(\mathbb{B})})
}
\end{pchstack}
\caption{Protocol definition of 2PC $\Gamma_{\mathsf{Refund}}$}
\end{figure}

\begin{figure}[H]
\begin{pchstack}[center, boxed]
\pseudocode{
    P_0(\mathsf{sk_{frz0(\mathbb{B})}}, \mathsf{tx_{swp}}) \< \< P_1(\mathsf{sk_{frz1(\mathbb{B})}}, \mathsf{hpk}) \\[0.1\baselineskip ][\hline] 
    \<\< \\[-0.5\baselineskip ]
    \mathsf{\textbf{if}} \:\: \mathcal{H}(\mathsf{tx_{swp}}.\mathsf{pk_{rx}}) \neq \mathsf{hpk} \lor \mathsf{tx_{swp}}.\mathsf{amnt} \neq \mathsf{amnt_b} \\
    \qquad \mathsf{\textbf{return}} \perp \\
    \mathsf{sk_{frz}} := \mathsf{sk_{frz0}} + \mathsf{sk_{frz1}} \\
    \sigma_\mathsf{swp(\mathbb{B})} \gets \Pi_{\mathsf{DS}}.\mathsf{Sign_{(\mathbb{B})}}(\mathsf{sk_{frz}}, \mathsf{tx_{swp}})
}
\end{pchstack}
\caption{Protocol definition of 2PC $\Gamma_{\mathsf{Swap}}$}
\end{figure}


\begin{figure}[H]
\hspace{-2cm}
\begin{minipage}[t]{0.6\textwidth}
\begin{pchstack}[boxed]
\pseudocode{
    \text{Global input} \:\: (T_0, T_1, T_2, \mathsf{amnt_a}, \mathsf{amnt_b},\mathbb{A}, \mathbb{B}) \\[0.1\baselineskip ][\hline] \\
    (\mathsf{sk_{frz0}}, \mathsf{pk_{frz}})_{(\mathbb{A})} \gets \mathsf{\textbf{wait}} \:\: \Gamma_{\mathsf{KeyGen}_{(\mathbb{A})}} \\
    (\mathsf{sk_{frz1}}, \mathsf{pk_{frz}})_{(\mathbb{B})} \gets \mathsf{\textbf{wait}} \:\: \Gamma_{\mathsf{KeyGen}_{(\mathbb{B})}} \\
    \mathsf{hsig} \gets \mathsf{\textbf{wait}} \:\:  \Gamma_{\mathsf{Refund}}(\mathsf{sk_{frz0(\mathbb{B})}}) \\
    (\pi_1, \pi_{\mathsf{lk_0}}, \mathsf{lk_0} ) \gets \mathsf{\textbf{wait}} \:\: \mathsf{receive}(P_1) \\
    {[\mathsf{sk_{frz1(\mathbb{A})}}]} := \mathsf{pk_{frz(\mathbb{A})}} - [\mathsf{sk_{frz0(\mathbb{A})}}] \\
    \mathsf{stmnt_0} := (\mathsf{lk_0}, {[\mathsf{sk_{frz1(\mathbb{A})}}]}, \mathsf{hsig}, \pi_1, T_1) \\
    \mathsf{\textbf{if}} \:\: \Pi_{\mathsf{ZK}\mathcal{L}_{\mathsf{lock}}}.\mathsf{Vr}(\mathsf{stmnt_0}, \pi_{\mathsf{lk_0}}) \neq 1 \\
        \quad \mathsf{\textbf{return}} \perp \\
    (\mathsf{pk_{swp}}, \mathsf{sk_{swp}})_{(\mathbb{B})} \gets \Pi_{\mathsf{DS}}.\mathsf{KeyGen_{(\mathbb{B})}} \\
    \mathsf{tx_{swp(\mathbb{B})}} \gets \mathsf{InitTx}_{(\mathbb{B})}(\mathsf{pk_{frz}}, \mathsf{pk_{swp}}, \mathsf{amnt_b}) \\
    (C_2, \pi_2) \gets \Pi_\mathsf{VTD}.\mathsf{Commit}(\mathsf{sk_{frz0}}, T_2) \\
    \mathsf{lk_1} := \mathsf{pk_{swp(\mathbb{B})}} \oplus C_2 \\
    \mathsf{hpk} \gets \mathcal{H}(\mathsf{pk_{swp(\mathbb{B})}}) \\
    \mathsf{stmnt_1} := (\mathsf{lk_1}, [\mathsf{sk_{frz0}}], \mathsf{hpk}, \pi_2, T_2) \\
    \pi_{\mathsf{lk_1}} \gets \Pi_{\mathsf{ZK}\mathcal{L}_{\mathsf{lock}}}.\mathsf{Pr}(\mathsf{stmnt_1}, \mathsf{pk_{swp(\mathbb{B})}}) \\
    \mathsf{send}(P_1, \pi_2, \pi_{\mathsf{lk_1}}, \mathsf{hpk}, \mathsf{lk_1}) \\
    (C_0, \pi_0) \gets \mathsf{\textbf{wait}} \:\: \mathsf{receive}(P_1) \\
    \mathsf{\textbf{if}} \:\: \Pi_{\mathsf{VTD}}.\mathsf{Verify}({[\mathsf{sk_{frz1(\mathbb{A})}}]}, C_0, \pi_0) \neq 1 \\
        \quad \mathsf{\textbf{return}} \perp \\
    \mathsf{\textbf{select}} \:\: \{ \\
        \quad \mathsf{\textbf{wait}} \:\: \{ \\
            \qquad \mathsf{sk_{frz1}} \gets \Pi_{\mathsf{VTD}}.\mathsf{ForceOp}(C_0) \\
            \qquad \mathsf{sk_{frz(\mathbb{A})}} := \mathsf{sk_{frz0}} + \mathsf{sk_{frz1}} \\
            \qquad \mathsf{FullTx}_{(\mathbb{A})}(\mathsf{pk_{frz}}, \mathsf{pk_{init}}, \mathsf{amnt_a}, \mathsf{sk_{frz}}) \\
        \quad \} \\
        \quad \mathsf{\textbf{wait}} \:\: \{ \\
            \qquad \mathsf{\textbf{do}} \:\: (\sigma_{\mathsf{tx_{last}}}, \_) \gets \mathsf{GetLatestTx}_{(\mathbb{B})}(\mathsf{pk_{frz}}) \\
            \qquad \mathsf{\textbf{while}} \:\: \mathcal{H}(\sigma_{\mathsf{tx_{last}}}) \neq {\mathsf{hsig}} \\
            \qquad C_1 := \mathsf{lk_0} \oplus \sigma_{\mathsf{tx_{last}}} \\
            \qquad \mathsf{sk_{frz1(\mathbb{A})}} \gets \Pi_\mathsf{VTD}.\mathsf{ForceOp}(C_1) \\
            \qquad \mathsf{sk_{frz(\mathbb{A})}} := \mathsf{sk_{frz0}} + \mathsf{sk_{frz1}} \\
            \qquad \mathsf{FullTx}_{(\mathbb{A})}(\mathsf{pk_{frz}}, \mathsf{pk_{init}}, \mathsf{amnt_a}, \mathsf{sk_{frz}}) \\
        \quad \} \\
        \quad \mathsf{\textbf{wait}} \:\: \{ \\
        \qquad \mathsf{FullTx}_{(\mathbb{A})}(\mathsf{pk_{init}}, \mathsf{pk_{frz}}, \mathsf{amnt_a}, \mathsf{sk_{init}}) \\
            \qquad \mathsf{\textbf{do}} \:\: \mathsf{bal_b} \gets \mathbf{wait} \:\: \mathsf{GetBal}_{(\mathbb{B})}(\mathsf{pk_{frz}}) \\
            \qquad \mathsf{\textbf{while}} \:\: \mathsf{bal_b} \neq {\mathsf{amnt_b}} \\
\
            \qquad \sigma_{\mathsf{swp(\mathbb{B})}} \gets \Gamma_{\mathsf{Swap}}(\mathsf{sk_{frz0(\mathbb{B})}}, \mathsf{tx_{swp(\mathbb{B})}}) \\
            \qquad \textbf{if} \:\: \mathsf{VerifyTx}_{(\mathbb{B})}(\sigma_{\mathsf{swp}}, \mathsf{tx_{swp}}) \neq 1 \\
            \qquad \quad \mathsf{\textbf{return}} \perp \\
            \qquad \mathsf{PubTx}_{(\mathbb{B})}(\sigma_{\mathsf{swp}}, \mathsf{tx_{swp}}) \\
            \qquad \mathsf{WatchTx}_{(\mathbb{B})}(\sigma_{\mathsf{swp}}, \mathsf{tx_{swp}}) \\
            \qquad \mathsf{send}(P_1, \mathsf{sk_{frz0(\mathbb{A})}}) \\
        \quad \} \\
    \}
}

\blockcomment{-0.26,-9.8}{-0.26,-0.5}{cblue}
\commentline{-0.55,-3}{Setup}
\blockcomment{-0.73,-12.5}{-0.73,-10.2}{cyellow}
\commentline{-1.1,-11}{Timeout}
\blockcomment{-1.2,-17}{-1.2,-13}{cred}
\commentline{-1.6,-15}{EarlyRefund}
\blockcomment{-1.665,-22}{-1.665,-17.5}{cgreen}
\commentline{-2,-19}{Swap}
\hspace{0.2cm}

\end{pchstack}
\end{minipage}%
\hspace{0.04\textwidth} 
\begin{minipage}[t]{0.5\textwidth}
\begin{pchstack}[boxed]
\pseudocode{
    \text{Global input} \:\: (T_0, T_1, T_2, \mathsf{amnt_a}, \mathsf{amnt_b},\mathbb{A}, \mathbb{B}) \\[0.1\baselineskip ][\hline] \\
    (\mathsf{sk_{frz1}}, \mathsf{pk_{frz}})_{(\mathbb{A})} \gets \mathsf{\textbf{wait}} \:\: \Gamma_{\mathsf{KeyGen}_{(\mathbb{A})}} \\
    (\mathsf{sk_{frz1}}, \mathsf{pk_{frz}})_{(\mathbb{B})} \gets \mathsf{\textbf{wait}} \:\: \Gamma_{\mathsf{KeyGen}_{(\mathbb{B})}} \\
    \mathsf{tx_{rfnd(\mathbb{B})}} \gets \mathsf{InitTx}_{(\mathbb{B})}(\mathsf{pk_{frz}}, \mathsf{pk_{init}}, \mathsf{amnt_b}) \\
    \sigma_{\mathsf{rfnd(\mathbb{B})}} \gets \Gamma_{\mathsf{Refund}}(\mathsf{sk_{frz1(\mathbb{B})}}, \mathsf{tx_{rfnd(\mathbb{B})}}) \\
    \mathsf{\textbf{if}} \:\: \mathsf{VerifyTx}_{(\mathbb{B})}(\sigma_{\mathsf{rfnd}}, \mathsf{tx_{rfnd}}) \neq 1 \\
    \quad \mathsf{\textbf{return}} \perp \\
    (C_1, \pi_1) \gets \Pi_\mathsf{VTD}.\mathsf{Commit}(\mathsf{sk_{frz1(\mathbb{A})}}, T_1) \\
    \mathsf{lk_0} := \sigma_{\mathsf{rfnd(\mathbb{B})}} \oplus C_1 \\
    \mathsf{hsig} := \mathcal{H}(\sigma_{\mathsf{rfnd(\mathbb{B})}}) \\
    \mathsf{stmnt_0} := (\mathsf{lk_0}, {[\mathsf{sk_{frz1(\mathbb{A})}}]}, \mathsf{hsig}, \pi_1, T_1) \\
    \pi_{\mathsf{lk_0}} \gets \Pi_{\mathsf{ZK}\mathcal{L}_{\mathsf{lock}}}.\mathsf{Pr}(\mathsf{stmnt_0}, \sigma_{\mathsf{rfnd(\mathbb{B})}}) \\
    \mathsf{\textbf{wait}} \:\: \mathsf{send}(P_0,\: (\pi_1, \pi_{\mathsf{lk_0}}, \mathsf{lk_0})) \\
    (\pi_2, \pi_{\mathsf{lk_1}}, \mathsf{hpk}, \mathsf{lk_1}) \gets \mathsf{receive}(P_0) \\
    {[\mathsf{sk_{frz0(\mathbb{A})}}]} := \mathsf{pk_{frz(\mathbb{A})}} - [\mathsf{sk_{frz1(\mathbb{A})}}] \\
    \mathsf{stmnt_1} := (\mathsf{lk_1}, {[\mathsf{sk_{frz0(\mathbb{A})}}]}, \mathsf{hpk}, \pi_2, T_2) \\
    \mathsf{\textbf{if}} \:\: \Pi_{\mathsf{ZK}\mathcal{L}_{\mathsf{lock}}}.\mathsf{Vr}(\mathsf{stmnt_1}, \pi_{\mathsf{lk_1}}) \neq 1 \\
        \quad \mathsf{\textbf{return}} \perp \\
    (C_0, \pi_0) \gets \Pi_{\mathsf{VTD}}.\mathsf{Commit}(\mathsf{sk_{frz1(\mathbb{A})}}, T_0) \\
    \mathsf{\textbf{wait}} \:\: \mathsf{send}(P_0,\: (C_0, \pi_0) \\
    \mathsf{\textbf{select}} \:\: \{ \\
        \quad \mathsf{\textbf{wait}} \:\: \{ \\
            \qquad \mathsf{timeout}(T_0/2) \\
            \qquad \mathsf{PubTx}_{(\mathbb{B})}(\sigma_{\mathsf{rfnd}}, \mathsf{tx_{rfnd}}) \\
        \quad \} \\
        \quad \mathsf{\textbf{wait}} \:\: \{ \\
            \qquad \mathsf{\textbf{do}} \:\: (\_, \mathsf{tx_{last}}) \gets \mathsf{GetLatestTx}_{(\mathbb{B})}(\mathsf{pk_{frz}}) \\
            \qquad \mathsf{\textbf{while}} \:\: \mathcal{H}(\mathsf{tx_{last}}.\mathsf{pk_{rx}}) \neq {\mathsf{hpk}} \\
	    \qquad C_2 := \mathsf{lk_1} \oplus \mathsf{tx_{last(\mathbb{B})}}.\mathsf{pk_{rx}} \\
            \qquad \mathsf{sk_{frz0}} \gets \Pi_\mathsf{VTD}.\mathsf{ForceOp}(C_2) \\
            \qquad \mathsf{sk_{frz(\mathbb{A})}} := \mathsf{sk_{frz0}} + \mathsf{sk_{frz1}} \\
            \qquad \mathsf{FullTx}_{(\mathbb{A})}(\mathsf{pk_{frz}}, \mathsf{pk_{swp}}, \mathsf{amnt_a}, \mathsf{sk_{frz}}) \\
        \quad \} \\
        \quad \mathsf{\textbf{wait}} \:\: \{ \\
            \qquad \mathbf{wait} \:\: \mathsf{FullTx}_{(\mathbb{B})}(\mathsf{pk_{init}}, \mathsf{pk_{frz}}, \mathsf{amnt_b}, \mathsf{sk_{init}}) \\
            \qquad \mathsf{\textbf{do}} \:\: \mathsf{bal_a} \gets \mathsf{GetBal}_{(\mathbb{A})}(\mathsf{pk_{frz}}) \\
            \qquad \mathsf{\textbf{while}} \:\: \mathsf{bal_a} \neq {\mathsf{amnt_a}} \\
            \qquad \Gamma_{\mathsf{Swap}}(\mathsf{sk_{frz1(\mathbb{B})}}, \mathsf{hpk}) \\
            \qquad \mathsf{sk_{frz0(\mathbb{A})}} \gets \mathsf{receive}(P_0) \\
            \qquad \mathsf{sk_{frz(\mathbb{A})}} := \mathsf{sk_{frz0}} + \mathsf{sk_{frz1}} \\
            \qquad \mathsf{FullTx}_{(\mathbb{A})}(\mathsf{pk_{frz}}, \mathsf{pk_{swp}}, \mathsf{amnt_a}, \mathsf{sk_{frz}}) \\
        \quad \} \\
    \} \\
}
\blockcomment{0.224,-9.8}{0.224,-0.5}{cblue}
\commentline{-0.15,-3.5}{Setup}
\blockcomment{-0.26,-12.4}{-0.26,-10.4}{cyellow}
\commentline{-0.7,-11.2}{Timeout}
\blockcomment{-0.73,-16.2}{-0.73,-12.6}{cred}
\commentline{-1.12,-15}{ForceSwap}
\blockcomment{-1.2,-21}{-1.2,-16.5}{cgreen}
\commentline{-1.62,-19}{Swap}

\hspace{0.5cm}

\end{pchstack}
\end{minipage}%
\caption{Full protocol execution for $P_0$ and $P_1$, respectively left and right (alternative syntax)}
\end{figure}





\newpage

\section{Proof sketch}

\subsubsection*{Security definitions}

\textbf{Definition 1} (Atomicity): \textit{Either both parties successfully exchange each other's assets or neither party performs a successful swap.}
\vspace{0.5em}
\\
That is, after the protocol run there can be only two outcomes: \\
1) $P_0$ holds $\mathsf{amnt_b}$ on $\mathsf{pk_{swp(\mathbb{B})}}$ and $P_1$ holds $\mathsf{amnt_a}$ on $\mathsf{pk_{swp(\mathbb{A})}}$ (successful swap) \\
2) $P_0$ holds $\mathsf{amnt_a}$ on $\mathsf{pk_{init(\mathbb{A})}}$ and $P_1$ holds $\mathsf{amnt_b}$ on $\mathsf{pk_{init(\mathbb{B})}}$ or $\mathsf{pk_{rfnd(\mathbb{B})}}$ (swap aborted) \\

\textbf{Definition 2} (Ownership): \textit{All parties gain exclusive knowledge of the secret keys of their respective wallets holding the exchanged or refunded assets upon completion of the protocol.} 
\vspace{0.5em}
\\
That is after a successful protocol run $P_0$ owns $\mathsf{sk_{swp(\mathbb{B})}}$ and $P_1$ owns $\mathsf{sk_{swp(\mathbb{A})}}$, and respectively for the aborted swap  $\mathsf{sk_{init(\mathbb{A})}}$ and $\mathsf{sk_{init(\mathbb{B})}}$ or $\mathsf{sk_{rfnd(\mathbb{B})}}$. \\


We model the interaction in a zero trust setting: both parties are mutually distrusting and potentially malicious as they have an incentive to deceive the counter-party. \\
We further assume that a party cannot deny the counter-party from transacting on the blockchain or performing local computations.


\subsection{Party $P_0$}
We want to prove that $P_0$ at the end of a protocol run either holds $\mathsf{amnt_a}$ on $\mathsf{pk_{init(\mathbb{A})}}$ with exclusive knowledge of $\mathsf{sk_{init(\mathbb{A})}}$ or they hold $\mathsf{amnt_b}$ on $\mathsf{pk_{swp(\mathbb{B})}}$ with exclusive knowledge of $\mathsf{sk_{swp(\mathbb{B})}}$. \\

\subsubsection*{Setup phase}
The parties first run the following 2PC protocols 
\begin{itemize}
    \item  $\Gamma_{\mathsf{KeyGen(\mathbb{A}}}$, generating $\mathsf{pk_{frz(\mathbb{A})}}$ outputing one share of the secret key $\mathsf{sk_{frz(\mathbb{A})}} := \mathsf{sk_{frz0(\mathbb{A})}} + \mathsf{sk_{frz1(\mathbb{A})}}$ to each respective party 
    \item  $\Gamma_{\mathsf{KeyGen(\mathbb{B}}}$, generating $\mathsf{pk_{frz(\mathbb{B})}}$ outputing one share of the secret key $\mathsf{sk_{frz(\mathbb{B})}} := \mathsf{sk_{frz0(\mathbb{B})}} + \mathsf{sk_{frz1(\mathbb{B})}}$ to each respective party
    \item  $\Gamma_{\mathsf{Refund}}$, jointly computing a signature of the refund transaction $\sigma_{\mathsf{rfnd(\mathbb{B})}}$, sets as output for $P_1$ and the relative hash $\mathsf{hsig}$ as output for $P_0$
\end{itemize}

$P_1$ then proceeds to
\begin{itemize}
\item Prepare two $\mathsf{VTD}$ commitments on $\mathsf{sk_{frz1(\mathbb{A})}}$ with respective time parameters $T_0, T_1$, generating $(C_0, \pi_0)$ and $(C_1, \pi_1)$
\item Mask $C_1$ with  $\mathsf{lk_0} := C_1 \oplus \sigma_{\mathsf{rfnd(\mathbb{B})}}$, sets the statement $\mathsf{stmnt_0} := (\mathsf{lk_0}, {[\mathsf{sk_{frz1(\mathbb{A})}}]}, \mathsf{hsig}, \pi_1, T_1)$ for $\mathcal{L}_{\mathsf{lock}}$ and generate a proof $\pi_{\mathsf{lk_0}}$ using $\sigma_{\mathsf{rfnd(\mathbb{B})}}$ as the witness.
\item Send $(C_0, \pi_0, \pi_1, \pi_{\mathsf{lk_0}}, \mathsf{lk_0})$ to $P_0$ 
\end{itemize}
\vspace{0.1cm}

$P_0$ then completes the setup phase and continues execution of the protocol only if both the following conditions are satisfied: 
\begin{itemize}
\item $\Pi_{\mathsf{VTD}}.\mathsf{Verify}([\mathsf{sk_{frz1(\mathbb{A})}}], C_0, \pi_0) = 1$, thus by the soundess property of the VTD scheme $P_0$ is guaranteed of retrieving the commited discrete log value $x$ in $C_0$ that satisfies $[x] = [\mathsf{sk_{frz1(\mathbb{A})}}]$ after $T_0$ steps. \\
Note that $P_0$ sets ${[\mathsf{sk_{frz1(\mathbb{A})}}]}$ with $\mathsf{pk_{frz(\mathbb{A})}} - [\mathsf{sk_{frz0(\mathbb{A})}}]$, since $\mathsf{pk_{frz(\mathbb{A})}} := [\mathsf{sk_{frz0(\mathbb{A})}} + \mathsf{sk_{frz1(\mathbb{A})}}]$ and thus $[\mathsf{sk_{frz0(\mathbb{A})}} + \mathsf{sk_{frz1(\mathbb{A})}}] - [\mathsf{sk_{frz0(\mathbb{A})}}] = [\mathsf{sk_{frz1(\mathbb{A})}}]$
\item $\Pi_{\mathsf{ZK}\mathcal{L}_{\mathsf{lock}}}.\mathsf{Vr}((\mathsf{lk_0}, {[\mathsf{sk_{frz1(\mathbb{A})}}]}, \mathsf{hsig}, \pi_1, T_1), \pi_2) = 1$, thus verifying that 
    $\exists y \:\: \text{s.t} \:\: \mathcal{H}(y) = \mathsf{hsig} \land \Pi_\mathsf{VTD}.\mathsf{Verify}(\mathsf{pk_{frz}} - [\mathsf{sk_{frz1}}], (\mathsf{lk_0} \oplus y), \pi, T_1)$ \\
    This checkes that $\sigma_\mathsf{rfnd(\mathbb{B})}$ is a valid witness for $\mathcal{L}_{\mathsf{lock}}$ and thus can be used to retrieve $C_1 := \sigma_\mathsf{rfnd(\mathbb{B})} \oplus \mathsf{lk_0}$, a commitment to a discrete log $[x] = [\mathsf{sk_{frz1}}]$ as above
\end{itemize}

The party now proceeds to concurrently run the following three execution branches. \\

\subsubsection*{Timeout branch}
\begin{itemize}
    \item Force opens the commitment $C_0$, retrieving  $\mathsf{sk_{frz1(\mathbb{A})}}$
    \item Computes $\mathsf{sk_{frz(\mathbb{A})}}$ and signs and publishes $\mathsf{tx_{rfnd(\mathbb{A})}}$ transferring back $\mathsf{amnt_a}$ from $\mathsf{pk_{frz(\mathbb{A})}}$ to $\mathsf{pk_{init(\mathbb{A})}}$ \\
\end{itemize}
\subsubsection*{EarlyRefund branch}
\begin{itemize}
    \item Continuosly queries the signature $\sigma_{\mathsf{tx_{last}}}$ of the latest transaction from $\mathsf{pk_{frz(\mathbb{B})}}$ until it finds the refund signature by checking $\mathcal{H}(\sigma_{\mathsf{tx_{last}}}) = \mathsf{hsig}$, thus $\sigma_{\mathsf{tx_{last}}}$ is a valid witness for $\mathcal{L}_{\mathsf{lock}}$
    \item Retrieves $C_1 := \mathsf{lk_0} \oplus \sigma_{\mathsf{tx_{last}}}$ and proceeds to force open it in time $T_1$, retrieving $\mathsf{sk_{frz1(\mathbb{A})}}$
    \item Computes $\mathsf{sk_{frz(\mathbb{A})}}$ and signs and publishes $\mathsf{tx_{rfnd(\mathbb{A})}}$ transferring back $\mathsf{amnt_a}$ from $\mathsf{pk_{frz(\mathbb{A})}}$ to $\mathsf{pk_{init(\mathbb{A})}}$
\end{itemize}

\subsubsection*{Swapping branch}
\begin{itemize}
    \item Initializes, signs and publishes the transaction $\mathsf{tx_{frz(\mathbb{A})}}$ transferring $\mathsf{amnt_a}$ from $\mathsf{pk_{init(\mathbb{A})}}$ to $\mathsf{pk_{frz(\mathbb{A})}}$
    \item Waits until $\mathsf{pk_{frz(\mathbb{B})}}$ holds $\mathsf{amnt_b}$ assets
    \item Generates a new keypair $(\mathsf{pk_{swp}}, \mathsf{sk_{swp}})_{(\mathbb{B})}$ and initializes the transaction $\mathsf{tx_{swp(\mathbb{B})}}$ transferring $\mathsf{amnt_b}$ from $\mathsf{pk_{init(\mathbb{B})}}$ to $\mathsf{pk_{swp(\mathbb{B})}}$
    \item Prepares a $\mathsf{VTD}$ commitment on $\mathsf{sk_{frz0(\mathbb{A})}}$ with time parameter $T_2$ , generating $(C_2, \pi_2)$.
    \item Masks $C_2$ with $lk_1 := C_2 \oplus \mathsf{pk_{swp(\mathbb{B})}}$, sets the statement $\mathsf{stmnt_1} := (\mathsf{lk_1}, {[\mathsf{sk_{frz0(\mathbb{A})}}]}, \mathsf{hpk} := \mathcal{H}(\mathsf{pk_{swp(\mathbb{B})}}), \pi_2, T_2)$ for $\mathcal{L}_{\mathsf{lock}}$ and generates a proof $\pi_{\mathcal{lk_1}}$ using $\mathsf{pk_{swp(\mathbb{B})}}$ as the witness.
    \item Sends  $(\pi_2, \pi_{\mathsf{lk_1}}, \mathsf{hpk}, \mathsf{lk_1})$ to $P_)0$
    \item Runs the 2PC protocol $\Gamma_{\mathsf{Swap}}$ where the parties jointly sign a signature $\sigma_{\mathsf{swp}}$ on $\mathsf{tx_{swp(\mathbb{B})}}$
    \item If it successfully verifies that $\sigma_{\mathsf{swp}}$ is valid on $\mathsf{tx_{swp(\mathbb{B})}}$, it publishes $\mathsf{tx_{swp(\mathbb{B})}}$
\end{itemize}

Note that $P_1$ gains no information on $\mathsf{pk_{swp(\mathbb{B})}}$ and thus can retrieve $C_1$ from $\mathsf{lk_1}$ and force open it only after $\mathsf{tx_{swp}}$ has been published by $P_0$. Thus if the protocol faults or timeouts before $P_0$ publishes $\mathsf{tx_{swp(\mathbb{B})}}$, $P_1$ cannot retrieve $\mathsf{sk_{frz0(\mathbb{A})}}$, resulting in the succesful execution of the timeout branch at $T_0$ and thus in $P_0$ holding $\mathsf{amnt_a}$ on $\mathsf{pk_{init(\mathbb{A})}}$. 

Now assume $P_0$ posts $\mathsf{tx_{swp(\mathbb{B})}}$, resulting in $P_1$ obtaining $C_2$: the VTD scheme's privacy property guarantees that $P_1$ will retrieve $\mathsf{sk_{frz0(\mathbb{A})}}$ only after time $T_2 > 2\mathbb{B}.\mathsf{ctime}  + \mathbb{A}.\mathsf{ctime}$. Now there are two possible outcomes: \\
1) The swap transaction gets confirmed after $\mathbb{B}.\mathsf{ctime}$  \\
2) $P_1$ tries to publish $\mathsf{tx_{rfnd(\mathbb{B})}}$ before $\mathsf{tx_{swp(\mathbb{B})}}$ is confirmed, resulting in a double spend attempt and in only one of $(\mathsf{tx_{rfnd(\mathbb{B})}}, \mathsf{tx_{swp(\mathbb{B})}}$ being accepted with roughly probability 1/2 for each

If (1) occurs, $P_0$ owns $\mathsf{amnt_b}$ on $\mathsf{pk_{swp(\mathbb{B})}}$ with exclusive knowledge of $\mathsf{sk_{swp(\mathbb{B})}}$ and thus we are done. \\
If (2) occurs, note that when $P_1$ posts the refund transaction $\mathsf{tx_{rfnd(\mathbb{B})}}$ $P_0$ will obtain the signature $\sigma_{\mathsf{tx_{rfnd}}}$ that hashes to $\mathcal{H}(\sigma_{\mathsf{tx_{rfnd}}}) = \mathsf{hsig}$  and thus successfully execute the EarlyRefund branch.

Since $P_1$ needs to post $\mathsf{tx_{rfnd(\mathbb{B})}}$ before $\mathbb{B}.\mathsf{ctime}$ and $T_1 = \mathbb{B}.\mathsf{ctime}$, the EarlyRefund branch will complete in time strictly less than $2\mathbb{B}.\mathsf{ctime} + \mathbb{A}.\mathsf{ctime}$ after $P_1$ obtains $C_2$, so $P_0$ is guaranteed to publish and confirm $\mathsf{tx_{rfnd(\mathbb{A})}}$, resulting in $P_0$ holding $\mathsf{amnt_a}$ on $\mathsf{pk_{init(\mathbb{A})}}$.


\subsection{Party $P_1$}
We want to prove that $P_1$ at the end of a protocol run either holds $\mathsf{amnt_b}$ on $\mathsf{pk_{init(\mathbb{B})}}$ with exclusive knowledge of $\mathsf{sk_{init(\mathbb{B})}}$ or they hold $\mathsf{amnt_a}$ on $\mathsf{pk_{swp(\mathbb{A})}}$ with exclusive knowledge of $\mathsf{sk_{swp(\mathbb{A})}}$. \\

After jointly generating $\mathsf{pk_{frz(\mathbb{A})}}$ with the 2PC protocol $\Gamma_{\mathsf{KeyGen}}$, $P_1$ commits $\mathsf{sk_{frz1}}$ with $\Pi_{\mathsf{VTD}}.\mathsf{Commit}(\mathsf{sk_{frz1}}, T_0)$. 
The party proceeds to concurrently runs two execution paths denoted by $\textbf{wait}$, with one being the refund branch and the other performing the swap. We note that from this point onwards, if any routine in the swapping branch timeouts or faults, $P_1$ will keep executing the refund branch. \\

The refund branch has a timeout 
It then proceeds to wait until $\mathsf{amnt_a}$ has been transferred to $\mathsf{pk_{frz(\mathbb{A})}}$ and retrieves a tuple $(\pi_1, \pi_2, \mathsf{hpk}, lk)$ from $P_0$ and proceeds to run $\Pi_{\mathsf{ZK}\mathcal{L}_{\mathsf{lock}}}.\mathsf{Vr}((lk, \mathsf{pk_{frz}} - [\mathsf{sk_{frz1}}], \mathsf{hpk}, \pi_1, T_1), \pi_2)$, verifying that $\exists(pk) \:\: \text{s.t} \:\: \mathcal{H}(pk) = h \land \Pi_\mathsf{VTD}.\mathsf{Verify}(\mathsf{pk_{frz}} - [\mathsf{sk_{frz1}}], (lk \oplus pk), \pi, T_1)$. Thus $P_1$ now knows the hash value $hpk$ of a public key $pk$ that can unlock a timed commitment from $lk$ which holds a discrete log value $x$ that satisfies  $[x] = \mathsf{pk_{frz}} - [\mathsf{sk_{frz1}}] = [\mathsf{sk_{frz0}} + \mathsf{sk_{frz1}}] - [\mathsf{sk_{frz1}}] = [\mathsf{sk_{frz0}}]$, hence $x = \mathsf{sk_{frz0}}$.

The parties then run the 2PC protocol $\Gamma_{\mathsf{Swap}}$ where $P_1$ signs $\mathsf{tx_{swp(\mathbb{B})}}$ (which is unknown by the 2PC input privacy property) unless $\mathcal{H}(\mathsf{tx_{swp}}.\mathsf{pk_{rx}}) \neq \mathsf{hpk} \lor  \mathsf{tx_{swp}}.\mathsf{amnt} \neq \mathsf{amnt_b}$, in which case it aborts from the protocol. \\

Since $\mathcal{H}(\mathsf{tx_{swp(\mathbb{B})}}.\mathsf{pk_{rx}}) = \mathsf{hpk}$,  $\mathsf{pk_{rx}}$ is a valid witness for $\pi_2$ and can be retrieved as soon as $P_0$ publishes $\mathsf{tx_{swp(\mathbb{B})}}$, allowing $P_1$ to start force opening $C_1$. \\
We have that $T_0 = s(\mathbb{A}.\mathsf{ctime} + \mathbb{B}.\mathsf{ctime})$ with $s$ being a scaling factor $s > 2$, assume $P_0$ waits until exactly $T_0/2$ before publishing $\mathsf{tx_{swp(\mathbb{B})}}$: $P_1$ force opens $C_1$ in $T_1 = \mathbb{B}.\mathsf{ctime})$ time, retrieves $\mathsf{sk_{frz0}}$ to sign and publish $\mathsf{tx_{swp(\mathbb{A})}}$ transferring $\mathsf{amnt_a}$ from $\mathsf{pk_{frz(\mathbb{A})}}$ to $\mathsf{pk_{swp(\mathbb{A})}}$. The transaction will be confirmed in $\mathbb{A}.\mathsf{ctime}$ and the whole swapping process from $P_1$ completes in $\mathbb{A}.\mathsf{ctime} + \mathbb{B}.\mathsf{ctime} < \frac{T_0}{2}$, thus if $P_0$ posts $\mathsf{tx_{swp(\mathbb{B})}}$ $P_1$ is guaranteed to successfly swap their assets resulting in $P_1$ holding $\mathsf{amnt_a}$ on $\mathsf{pk_{swp(\mathbb{A})}}$. \\

Now assume instead that some routine in the protocol execution timed out or faulted, resulting in executing the refund branch: $P_1$ will generate a new key pair $(\mathsf{pk_{rfnd}}, \mathsf{sk_{rfnd}})$ on $\mathbb{B}$, and sign and publish

\printbibliography

\newpage
\appendix

\section*{Appendix}
\subsection*{Commit transaction}

Commit transactions can be easily realized by a blockchain using a public ledger model by modifying consensus rules, and require no additional scripting capabilities.  \\
A commit transaction allows to lock assets for a specified transfer until a certain timeout T, commiting to a secret value $x$ in $C$. If $C$ gets opened by broadcasting the revealed $x$ before timeout T, the transfer is completed unconditionally. \\ 
After timeout T, the commit transaction is considered expired, unfreezing the assets and ignoring openings on $C$.

We further define the following routines for blockchains that support commitment transaction.
\begin{itemize}[topsep=0pt, itemsep=0pt, leftmargin=2em]
	\item $\mathbf{ctx}_{\mathbb{A}} := (\mathsf{tx_{\mathbb{A}}}, C) \gets \mathbf{CommitTx}_{(\mathbb{A})}(C, \mathsf{pk_{tx}}, \mathsf{pk_{rx}}, \mathsf{amnt}, \mathsf{T})$: creates an unsigned commit transaction paying $\mathsf{amnt}$ from $\mathsf{pk_{tx}}$ to $\mathsf{pk_{rx}}$ valid until time $\mathsf{T}$, whith  $\mathsf{amnt}$ being locked from being spent on transactions from $\mathsf{pk_{tx}}$ until $\mathsf{T}$. The transaction can be later finalize by opening and broadcasting $C$ through $\mathsf{RevTx}$.
\item $ \mathbf{0/1} \gets \mathbf{RevTx}_{(\mathbb{A})}(\mathsf{sec}, \mathsf{ctx})$: open the (on-chain) committed transaction $\mathsf{ctx}$ by revealing the commited secret $\mathsf{sec}$. Return 0 if the opening fails or the timeout on $\mathsf{ctx}$ has expired. If the revealing succeeds, return 1 and finalize $\mathsf{ctx}$.
\end{itemize}

By assuming this functionality on both chains we can now realize the following atomic swap protocol.

\begin{figure}[H]
\begin{minipage}[t]{0.5\textwidth}
\begin{pchstack}[boxed]
\pseudocode{
    \text{Global input} \:\: (T, \mathsf{amnt_a}, \mathsf{amnt_b},\mathbb{A}, \mathbb{B}) \\[0.1\baselineskip ][\hline] \\
    \mathsf{sec} \gets \mathbb{Z}_q \\
    (C_0, r) \gets \Pi_{\mathsf{PCom}}.\mathsf{Commit}(\mathsf{sec_0}) \\ % verify that sec_0 has been commited
    \mathsf{ctx_{(\mathbb{A})}} \gets \mathsf{CommitTx}_{(\mathbb{A})}(C_0, \mathsf{pk_{init}}, \mathsf{pk_{swp}}, \mathsf{amnt_a}, T + \Delta) \\
    \sigma_{\mathsf{ctx_{(\mathbb{A})}}} \gets \Pi_{\mathsf{DS}}.\mathsf{Sign}_{(\mathbb{A})}(\mathsf{sk_{init}}, \mathsf{ctx}) \\
    \mathsf{PubTx}_{(\mathbb{A})}(\sigma_{\mathsf{ctx}}, \mathsf{ctx}) \\
    \mathsf{\textbf{select}} \:\: \{ \\
    \quad \mathsf{\textbf{wait}} \:\: \{ \\
    \qquad \mathsf{timeout}(T) \\
    \quad \} \\
    \quad \mathsf{\textbf{wait}} \:\: \{ \\
    \qquad \mathsf{send}(\mathsf{ctx_{(\mathbb{A})}}) \\
    \qquad \mathsf{receive}(\mathsf{ctx_{(\mathbb{B})}}) \\
    \qquad \mathsf{res_1} \gets \mathsf{\textbf{wait}} \:\: \mathsf{WatchTx}_{(\mathbb{B})}(\mathsf{ctx}) \\
    \qquad \mathsf{\textbf{if}} \:\: \mathsf{ctx}_{(\mathbb{B})}.C \neq C_0 \lor \mathsf{ctx}_{(\mathbb{B})}.T \neq T \lor \mathsf{res_1} \neq 1 \\ % verify that commit transaction is valid and accepted
    \qquad \quad \mathsf{\textbf{return}} \perp \\
    \qquad \mathsf{rtx_{swp(\mathbb{B})}} \gets \mathsf{RevTx}_{(\mathbb{B})}(\mathsf{sec}, \mathsf{ctx}) \\
    \quad \} \\
    \} \\
}
\end{pchstack}
\end{minipage}%
\hspace{0.4cm}
\begin{minipage}[t]{0.5\textwidth}
\begin{pchstack}[boxed]
\pseudocode{
    \text{Global input} \:\: (T, \mathsf{amnt_a}, \mathsf{amnt_b},\mathbb{A}, \mathbb{B}) \\[0.1\baselineskip ][\hline] \\
    \mathsf{receive}(\mathsf{ctx_{(\mathbb{A})}}) \\
    \mathsf{res_0} \gets \mathsf{\textbf{wait}} \:\: \mathsf{WatchTx}_{(\mathbb{A})}(\mathsf{ctx}) \\
    \mathsf{\textbf{select}} \:\: \{ \\
    \quad \mathsf{\textbf{wait}} \:\: \{ \\
    \qquad \mathsf{timeout}(T) \\
    \quad \} \\
    \quad \mathsf{\textbf{wait}} \:\: \{ \\
    \qquad \mathsf{\textbf{if}} \:\: \mathsf{ctx_{(\mathbb{A})}}.T \neq T + \Delta \lor \mathsf{res_0} \neq 1 \\ % verify that commit transaction is valid and accepted
    \qquad \quad \mathsf{\textbf{return}} \perp \\
    \qquad \mathsf{ctx}_{(\mathbb{B})} \gets \mathsf{CommitTx}_{(\mathbb{B})}(C_0, \mathsf{pk_{init}}, \mathsf{pk_{swp}}, \mathsf{amnt_b}, T) \\
    \qquad \mathsf{send}(\mathsf{ctx}_{(\mathbb{B})}) \\
    \qquad \mathsf{rtx_{swp(\mathbb{B})}} \gets \mathsf{\textbf{wait}} \:\: \mathsf{GetLatestTx}_{(\mathbb{B})}(\mathsf{pk_{swp}}) \\
    \qquad \mathsf{sec} := \mathsf{rtx_{swp(\mathbb{B})}}.\mathsf{rev} \\
    \qquad \mathsf{rtx_{swp(\mathbb{A})}} \gets \mathsf{RevTx}_{(\mathbb{A})}(\mathsf{sec}, \mathsf{ctx}) \\
    \quad \} \\
    \} \\
}
\end{pchstack}
\end{minipage}%
\caption{Full protocol execution for $P_0$ and $P_1$, respectively left and right}
\end{figure}


\end{document}
