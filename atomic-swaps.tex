\documentclass{article}      	% Style of the document                     
\usepackage{fullpage}
\usepackage{amsmath,amsthm}     	   	% Maths    
\newtheorem{definition}{Definition}                                      
\usepackage[utf8]{inputenc}	% UTF-8 characters                                               
\usepackage[T1]{fontenc}    	% Tuki ääkkösille (Finnish names don't cause problems)                                            
\usepackage{parskip}        		% Linebreak between paragraphs                
\usepackage{svg}
\usepackage{graphicx}       		% Graphics package for adding figures                        
\usepackage{epstopdf}       		% Possibility to add *.eps figures
 \usepackage{ dsfont }            % Symbol for real numbers
\usepackage{extarrows}
\usepackage{float}
\usepackage{makeidx}
\usepackage{enumitem}        % possibility to label list items by alphabet
\usepackage[a4paper, top=0.5in]{geometry}
\newcommand{\M}[1]{\ensuremath{\text{\texttt{#1}}}}
\usepackage[
    lambda,
    operators,
    advantage,
    sets,
    adversary,
    landau,
    probability,
    notions,
    logic,
    ff,
    mm,
    primitives,
    events,
    complexity,
    asymptotics,
    keys]{cryptocode}

\usepackage{todonotes}

 \usepackage{amsmath,amsfonts,graphicx,amssymb,amsthm}


\usepackage[bookmarksdepth=2,draft=false]{hyperref}
\hypersetup{colorlinks=true,linkcolor={red!50!black},citecolor=darkgray,linkcolor=darkgray}
\usepackage[capitalize]{cleveref}

\mathchardef\mhyphen="2D

 %% general
\mathchardef\mhyphen="2D
\newcommand{\fdv}{\mathcal{F}}
\newcommand{\tdv}{\mathcal{T}}
\newcommand{\vdv}{\mathcal{V}}
\newcommand{\cX}{\mathcal{X}}
\newcommand{\cF}{\mathcal{F}}
\newcommand{\cG}{\mathcal{G}}
\newcommand{\ID}{\mathcal{I}}
\newcommand{\bits}[1][]{\{0,1\}^{#1}}
\renewcommand{\vec}[1]{\mathbf{#1}}
\newcommand{\mat}[1]{\mathbf{#1}}
\newcommand{\inner}[2]{\langle #1, #2 \rangle}
\newcommand{\transpose}{\mathtt{T}}
\newcommand{\round}[1]{\lfloor #1 \rceil}
\renewcommand{\dist}{\mathsf{dist}}
\renewcommand{\Pr}[2][]{{\text{Pr}_{#1}\left[#2\right]}}
\newcommand{\Exp}[2][]{{\mathbb{E}_{#1}\left[#2\right]}}
\newcommand{\mathcm}[2][1cm]{\hspace{#1}{\mbox{/\!\!/ } \text{\scriptsize#2}}}

%% lattice problems
\newcommand{\SIS}{\mathsf{SIS}}
\newcommand{\ISIS}{\mathsf{ISIS}}
\newcommand{\nfSIS}{\mathsf{nfSIS}}
\newcommand{\dSIS}{\mathsf{dSIS}}
\newcommand{\LWE}{\mathsf{LWE}}
\newcommand{\nfLWE}{\mathsf{nfLWE}}
\newcommand{\nfdLWE}{\mathsf{nfdLWE}}
\newcommand{\sLWE}{\mathsf{sLWE}}
\newcommand{\dLWE}{\mathsf{dLWE}}
\newcommand{\SVP}{\mathsf{SVP}}
\newcommand{\CVP}{\mathsf{CVP}}
\newcommand{\SIVP}{\mathsf{SIVP}}
\newcommand{\GapSVP}{\mathsf{GapSVP}}
\newcommand{\BDD}{\mathsf{BDD}}
\newcommand{\NTRU}{\mathsf{NTRU}}
\newcommand{\sNTRU}{\mathsf{sNTRU}}
\newcommand{\dNTRU}{\mathsf{dNTRU}}

%% lattice macros
\newcommand{\TT}{\mathbb{T}}
\newcommand{\ring}{\mathcal{R}}
\newcommand{\lattice}{\mathcal{L}}
\newcommand{\piped}{\mathcal{P}}
\newcommand{\ball}{\mathcal{B}}
\newcommand{\Hyb}{\mathsf{Hyb}}
\newcommand{\lspan}{\mathsf{span}}
\newcommand{\rank}{\mathsf{rank}}
\newcommand{\lsb}{\mathsf{LSB}}
\newcommand{\pubparam}{\mathsf{pp}}

%% group macros

%% syntax
\newcommand{\mpk}{\mathsf{mpk}}
\newcommand{\msk}{\mathsf{msk}}
\newcommand{\msg}{\mathsf{msg}}
\newcommand{\rnd}{\mathsf{rnd}}
\newcommand{\ctxt}{\mathsf{ctxt}}
\newcommand{\com}{\mathsf{com}}
\newcommand{\td}{\mathsf{td}}
\newcommand{\id}{\mathsf{id}}
\newcommand{\stmt}{\mathsf{stmt}}
\newcommand{\wit}{\mathsf{wit}}
\newcommand{\tx}{\mathsf{tx}}
\newcommand{\aux}{\mathsf{aux}}
\newcommand{\ek}{\mathsf{ek}}

\newcommand{\Setup}{\mathsf{Setup}}
\newcommand{\Commit}{\mathsf{Com}}
\newcommand{\TrapGen}{\mathsf{TrapGen}}
\newcommand{\SampD}{\mathsf{SampD}}
\newcommand{\SampPre}{\mathsf{SampPre}}
\newcommand{\Prove}{\mathsf{Prove}}
\newcommand{\Verify}{\mathsf{Verify}}
\newcommand{\val}{\mathsf{val}}

%% primitive/scheme name
\newcommand{\PKE}{\mathsf{PKE}}
\newcommand{\LTDF}{\mathsf{LTDF}}
\newcommand{\rsagen}{\mathsf{RSAGen}}
\newcommand{\rsa}{\mathsf{RSA}}
\newcommand{\LHE}{\mathsf{LHE}}
\newcommand{\CS}{\mathcal{CS}}
\newcommand{\NTRUEncrypt}{\mathsf{NTRUEncrypt}}

%% others
\newcommand{\oracle}{\mathcal{O}}
\newcommand{\pcas}{~\mathbf{as}~}

\newcommand{\polylog}[1][\secpar]{\mathsf{polylog}(#1)}

\newcommand{\indrsidcpa}{\mathrm{IND\$}\mhyphen\mathrm{sID}\mhyphen\mathrm{CPA}}
%\newcommand{\oplus}{\, \texttt{XOR} \,} % shorthand for typing the XOR operator in mathmode
\usepackage{tikz}
\usetikzlibrary{decorations.pathreplacing}
\usetikzlibrary{decorations.pathmorphing}


\definecolor{cgreen}{RGB}{0, 153, 51}
\definecolor{cblue}{RGB}{0, 102, 204}
\definecolor{cyellow}{RGB}{255, 204, 0} 
\definecolor{cred}{RGB}{204, 51, 0} 

\newcommand{\commentline}[2]{%
    \tikz[remember picture, overlay]{
        \node [black,anchor=west,xshift=10pt] at (#1) {#2};
    }
}

\newcommand{\blockcomment}[3]{%
    \tikz[remember picture, overlay]{
        \draw [decorate,decoration={lineto,amplitude=10pt,mirror,raise=4pt},yshift=0pt,very thick,{#3}] 
        (#1) -- (#2) node [black,midway,xshift=10pt] {};
    }
}



\usepackage{biblatex}
\addbibresource{references.bib}

\begin{document}         
\author{Lorenzo Tucci}
\title{RingCCT: confidential commit transactions and atomic swaps}

\maketitle

\tableofcontents
\newpage

\section{Introduction}
\section{RingCCT: Ring confidential commit transaction}
We present an extension of ring confidential transactions (RingCT), called ring confidential commit transactions (RingCCT).

\subsection{Syntax}

A RingCCT scheme is initialised by running the algorithm $\mathsf{Setup}$ that generates the public parameters and initial state $\mathsf{st}$. The system can be joined by running $\mathsf{KGen}$ to generate master public key pair consisting of $\mathsf{mpk}$ and $\mathsf{msk}$.
Accounts in the systems are defined by the tuple $(\mathsf{spk}, \mathsf{tpk}, \mathsf{rpk}, \mathsf{rsk}, \mathsf{co})$, where 
\begin{itemize}
	\item $\mathsf{co}$ is a commitment of the account data $\mathsf{accd}$, which is a tuple of $(a, \mathsf{time})$. $a$ denotes the amount stored in the account, while $\mathsf{time}$ is an optional epoch timeout parameter that defines a cutoff epoch for the ownership of the commit account.
	\item $\mathsf{spk}$ is the public key of the owner of the account. In case $\mathsf{time}$ is defined as 0, this is the only owner of the account, as in normal RingCT.
	\item $\mathsf{tpk}$ is the public key of the joint owner of a commit transaction account, the ccommit account is jointly owned by $\mathsf{spk}, \mathsf{tpk}$ until the epoc defined $\mathsf{time}$.
	\item $\mathsf{rpk}$ is a recovery public key, which becomes the owner of a commit transaction account after $\mathsf{time}$.
\end{itemize}

\textbf{Definition} A RingCCT scheme (Ring Commit Confidential Transactions) scheme consists of the PPT algorithms ($\mathsf{Setup,KGen,Tx,Vf,StExt,TxExt,SrcChk,TgtChk}$) whose interfaces are defined as follows.

\begin{itemize}
    \item $(\mathsf{pp,st}) \gets \mathsf{Setup}(1^\lambda)$: the setup algorithm generates the public parameters $\mathsf{st}$ and an initial global state $\mathsf{st}$.
    \item $(\mathsf{mpk},\mathsf{msk}) \gets \mathsf{KGen}(\mathsf{pp})$: the key generation algorithm generates a master public key $\mathsf{mpk}$ and a matching secret key $\mathsf{msk}$.
    \item $(\mathsf{sk},\mathsf{accd}) \gets \mathsf{KDer}(\mathsf{msk, \tau})$: the key derivation algorithm generates derives the keys-account data tuple given the master key $\mathsf{msk}$ owning the account and the token $\tau$ of the account.
    \item $(\mathsf{tx,TK}) \gets \mathsf{TxGen}(\mathsf{st},P,R,\mathcal{S},\mathcal{T})$: the transaction algorithm inputs a state $\mathsf{st}$, a predicate $P: \mathbb{Z}^S \times \mathbb{Z}^T \rightarrow \{0,1\}$, an index set R called the ring, a set of source accouts information $\mathcal{S} = \{\mathsf{ssk}_i, \mathsf{tsk}_i, \mathsf{rsk}_i, \mathsf{accd}_i\}_{i\in S}$ and some targets account information $\mathcal{T} = \{\mathsf{smpk}_i, \mathsf{tmpk}_i, \mathsf{rmpk}_i, \mathsf{accd}'_i\}_{i\in T}$; where $\mathsf{ssk,tsk,rsk}$ and $\mathsf{spk,tpk,rpk}$ are the source, target and recovery secret and public keys respectively. If source or target account is not commit based, only the source key pair is defined. Each account has some $\mathsf{accd} := (a,\mathsf{time})$ defined, where $a$ represents some amount and $\mathsf{time}$ sets a specific epoch timeout of the ownership of commit account by the target key pair, and empty otherwise. 
    \item $(b,\mathsf{st}') \gets \mathsf{Vf}(\mathsf{st,tx})$: The verification algorithm outputs a bit b deciding whether to accept or reject that the transaction $\mathsf{tx}$ is a valid relative to the state $\mathsf{st}$, outputting an updated state $\mathsf{st}'$ if the verification is sucessful.
\item $\mathsf{AC}_U \gets \mathsf{StExt}(\mathsf{st})$: The state extraction algorithm
extracts the set of universe accounts $\mathsf{AC}_U = \{\mathsf{ac}_i\}_{i \in U}$ encoded in the state $\mathsf{st}$.
\item $\mathsf{AC}_T \gets \mathsf{TxExt}(\mathsf{tx})$: The transaction extraction algorithm
extracts the set of universe accounts $\mathsf{AC}_T = \{\mathsf{ac}_i\}_{i \in T}$ encoded in the state $\mathsf{st}$.
\item $b \gets \mathsf{SrcChk}(\mathsf{ac,r,ssk,tsk,rsk,accd,clock})$: The source checking algorithm outputs a bit $b$ deciding whether to accept or reject that the account $\mathsf{ac}$ is associated to the provided secret keys and that $\mathsf{accd}$ has been commited with randomness $r$. If the account is commit based, it checks validity of $\mathsf{ssk,tsk}$ when $\mathsf{clock} <= \mathsf{time}$ and of $\mathsf{rsk}$ otherwise; if the account is standard only $\mathsf{ssk}$ is required.
\item $b \gets \mathsf{TgtChk}(\mathsf{ac,accd})$: The target checking algorithms outputs a bit $b$ deciding whether to accept or reject that the $\mathsf{accd}$ has been commited in $\mathsf{ac}$. 
\end{itemize}

\subsection{Correctness}

\subsection{Security}
We here define the security properties of RingCCT.

\begin{definition}[Balance] A RingCCT scheme is balanced if: \\
	1. A commit transaction account ownership changes from $\mathsf{ssk}, \mathsf{tsk}$ to $\mathsf{rsk}$ based on some epoch $\mathsf{time}$, i.e. for any PPT adversary $\mathcal{A}$ it holds that
\end{definition}
$\mathsf{Pr}\left[
    \begin{cases} 
	\mathsf{SrcChk}(\mathsf{ac}, r, (\mathsf{ssk}, \mathsf{tsk}, \perp), \mathsf{accd}, \mathsf{time}+1) \tabularnewline 
	\mathsf{SrcChk}(\mathsf{ac}, r, (\perp, \perp, \mathsf{rsk}), \mathsf{accd}, \mathsf{time}-1) \tabularnewline
        \Gamma.\mathsf{Com}(\mathsf{accd}, r) = \mathsf{co} 
    \end{cases} 
    \middle|(
    \begin{aligned}
	(\mathsf{pp}, \mathsf{st}) \gets \mathsf{Setup}(1^\lambda) \\
	\mathsf{ac}, \mathsf{ssk}, \mathsf{tsk}, \mathsf{rsk}, \mathsf{accd}) \gets \mathcal{A}(\mathsf{pp}) \\
    	\mathsf{time} := \mathbf{parse} \: \mathsf{accd} \\
    \end{aligned}
\right]
\leq \negl
$


\begin{figure}[H]
\begin{pchstack}[center, boxed]
\pseudocode{
    \text{Balance} \\[0.1\baselineskip ][\hline] 
    (\mathsf{pp}, \mathsf{st}_0) \gets \mathsf{Setup}(1^\lambda) \\
    (\mathsf{tx}_i)_{i \in \mathbb{Z}_l} \gets \mathcal{A}(\mathsf{pp}, \mathsf{st}_0) \\
    (P_i, R_i, S_i, T_i)_{i \in \mathbb{Z}_{l}} \gets \mathcal{E}_{\mathsf{A}} (\mathsf{pp}, \mathsf{st}_0, (\mathsf{tx}_i)_{i \in \mathbb{Z}_{l}}) \\
    \{ \mathsf{sks}_{i,j}, \mathsf{accd}_{i,j} \}_{j \in S_i} := \mathbf{parse} \: (S_i)_{i \in \mathbb{Z}_{l}} \\
    \{ \mathsf{mpks}_{i,j},\mathsf{tk}_{i,j}, \mathsf{accd}'_{i,j} \}_{j \in T_i}) := \mathbf{parse} \: (T_i)_{i \in \mathbb{Z}_{l}} \\
    \mathbf{for} \: t \in \mathbb{Z}_l \: \mathbf{do} \: (b_t, \mathsf{st}_{t+1}) := \mathsf{Vf}(\mathsf{st_t}, \mathsf{tx_t}) \\
    \mathbf{for} \: i \in \mathbb{Z}_l \: \mathbf{do} \\
    \qquad \mathsf{accd}_{i, S_i} := (\mathsf{accd}_{i,j})_{j \in S_i},
    \mathsf{accd}'_{i, T_i} := (\mathsf{accd}'_{i,j})_{j \in T_i} \\
    \qquad \{ \mathsf{ac}_{i,j} \}_{j \in U_i} := \mathsf{StExt}(\mathsf{st}_i), \{ \mathsf{ac'}_{i,j} \}_{j \in T_i} := \mathsf{TxExt}(\mathsf{tx}_i) \\
    \qquad b'_i := 
    \begin{cases}
	\mathsf{TxExt} \subseteq \mathsf{StExt}(\mathsf{st}_{i+1}) \vspace{0.3em} \tabularnewline
	P_i \in \mathcal{P} \tabularnewline
	P_i(\mathsf{accd}_{i, S_i}, \mathsf{accd}'_{i, T_i}) \vspace{0.3em} \tabularnewline
	S_i \subseteq R_i \subseteq U_i \vspace{0.3em} \tabularnewline
	\mathsf{SrcChk}(\mathsf{StExt}(\mathsf{st}_i)[j], \mathsf{sks}_{i,j}, \mathsf{accd}_{i,j}) = 1 \:\:\: \forall j \in \mathsf{S}_i \vspace{0.3em} \tabularnewline
	\mathsf{TgtChk}(\mathsf{TxExt}(\mathsf{tx}_i)[j], \mathsf{mpks}_{i,j}, \mathsf{accd}_{i,j}) = 1 \:\:\: \forall j \in \mathsf{S}_i \vspace{0.3em} \tabularnewline
    \end{cases} \\
    b'' := (\exists i_0 < i_1, S_{i_0} \cap S_{i_1} = \emptyset) \\
    \mathbf{return} \bigwedge_{i \in \mathbb{Z}_l} b_i \land \neg (\bigwedge_{i \in \mathbb{Z}_l} b_i' \land b_i'')
}
\end{pchstack}
\caption{Protocol definition of 2PC $\Gamma_{\mathsf{CommitTx}}$}
\end{figure}


\subsection*{Construction}

\begin{equation*}
\mathcal{R}(\mathsf{stmnt}, \mathsf{wit}) := \begin{cases} 
    S \subseteq R \\ 
    \xi_{\phi S(i)} = \Delta.\mathsf{Eval}(s_i) \qquad \forall i \in S \\
    \mathsf{SrcChk}(\mathsf{ac}_i, r, \mathsf{sks}_i, \mathsf{accd}_i, \mathsf{clock}) = 1 \qquad \forall i \in S \\ 
    \mathsf{TgtChk}(\mathsf{ac'}_i, \mathsf{accd}'_i) = 1 \qquad \forall i \in T \\ 
    P(a_S, a'_T) = 1
\end{cases}
\end{equation*}

\begin{equation*}
\mathsf{stmnt} := (P,\mathsf{AC}_R,\mathcal{Z}_{\bar{S}}, \mathsf{AC}_T, \mathsf{clock}) \\
\end{equation*}
\begin{equation*}
\mathsf{wit} := ((r,\mathsf{sks}_i, \mathsf{accd}_i)_{i\in S}), (\mathsf{mpks}_i, \mathsf{accd}'_i)_{i\in T}) \\
\end{equation*}



\begin{figure}
\begin{minipage}[t]{\textwidth}
\begin{pchstack}[boxed]

\begin{pcvstack}
\pseudocode{
    \mathsf{Setup}(1^\lambda) \\ [0.1\baselineskip ][\hline]
    \mathsf{crs} \gets \Pi.\mathsf{Setup}(1^\lambda) \\
    \mathsf{ck} \gets \Gamma.\mathsf{Gen}(1^\lambda) \\
    \mathsf{pp_\Delta} \gets \Delta.\mathsf{Setup}(1^\lambda) \\
    \mathbf{return} \: (\mathsf{pp}, \mathsf{st})
}
\vspace{1em}
\pseudocode{
    \mathsf{KGen}(\mathsf{pp}) \\[0.1\baselineskip ][\hline]
    \mathsf{msk} \sample\mathcal{K} \\
    \mathsf{mpk} := \Delta.\mathsf{KGen(msk)} \\
    \mathbf{return} \: (\mathsf{mpk}, \mathsf{msk})
}
\vspace{1em}
\pseudocode{
    \mathsf{Vf}(\mathsf{st},\mathsf{tx}) \\[0.1\baselineskip ][\hline]
    (\mathsf{AC}_U, \mathcal{Z}_U) := \mathbf{parse} \: \mathsf{st} \\
    \{\mathsf{ac}_i\}_{i \in U} := \mathbf{parse} \: \mathsf{AC}_U \\
    (P,R,\mathsf{AC}_T, \mathcal{Z}_{\bar{S}}) := \mathbf{parse} \: \mathsf{tx} \\
    \mathsf{AC}_R := \{\mathsf{ac}_i\}_{i \in R} \\
    \mathsf{stmnt} := (P,\mathsf{AC}_R,\mathsf{AC}_T,\mathcal{Z}_{\bar{S}}, \mathsf{clock}) \\
    \mathbf{if} \: \begin{cases}
        P \in \mathcal{P} \tabularnewline
        R \subseteq U \tabularnewline
        \Pi.\mathsf{Vf}(\mathsf{crs}, \mathsf{stmnt}, \pi) = 1 \tabularnewline
        \mathcal{Z}_{\bar{S}} \cap \mathcal{Z}_{\bar{U}} =  
    \end{cases} \: \mathbf{then} \\
    \:\: \mathbf{return} \: (1, \mathsf{st}') \\
    \mathbf{else} \: \mathbf{return} \: (0, \mathsf{st})
}
\vspace{1em}
\pseudocode{
    \mathsf{StExt}(\mathsf{st}) \\[0.1\baselineskip ][\hline]
    (\mathsf{AC}_U, \mathcal{Z}_U) := \mathbf{parse} \: \mathsf{st} \\
    \mathbf{return} \: \mathsf{AC}_U
}
\vspace{1em}
\pseudocode{
    \mathsf{TxExt}(\mathsf{tx}) \\[0.1\baselineskip ][\hline]
    (P,R,\mathsf{AC}_T, \mathcal{Z}_{\bar{S}}) := \mathbf{parse} \: \mathsf{tx} \\
    \mathbf{return} \: \mathsf{AC}_T
}
\vspace{1em}
\pseudocode{
    \mathsf{TgtChk}(\mathsf{ac}, \mathsf{accd}) \\[0.1\baselineskip ][\hline]
    \mathsf{co} := \mathbf{parse} \: \mathsf{ac} \\
    (r, \mathsf{accd}') := \mathbf{parse} \: \mathsf{tk} \\
    \mathbf{return} \: \begin{cases}
        %\mathsf{accd}' \overset{?}{=} \mathsf{accd} \tabularnewline
        \mathsf{co} \overset{?}{=} \Gamma.\mathsf{Com}(\mathsf{accd},r)
    \end{cases} 
}
\end{pcvstack}
\qquad
\begin{pcvstack}
\pseudocode{
    \mathsf{KDer}(\mathsf{msk},\tau) \\[0.1\baselineskip ][\hline]
    (r, \delta, \mathsf{accd} := (a, \mathsf{time})) := \mathbf{parse} \: \tau \\
    \mathsf{sk} := \mathsf{msk}+\delta \\
    \mathbf{return} \: (\mathsf{sk}, r, \mathsf{accd})
}
\vspace{1em}
\pseudocode{
    \mathsf{TxGen}(\mathsf{st}, P, R, \mathcal{S}, \mathcal{T}) \\[0.1\baselineskip ][\hline]
    \{\mathsf{sks}_i, \mathsf{accd}_i\}_{i\in S} := \mathbf{parse} \: \mathcal{S} \\
    \{\mathsf{mpks}_i, \mathsf{accd}'_i\}_{i\in T} := \mathbf{parse} \: \mathcal{T} \\
    \mathbf{for} \; i \in T \: \mathbf{do} \\
    \:\: r'_i \sample \chi \\
    \:\: \delta'_i \sample \mathcal{K} \\
    \:\: \mathsf{co}'_i := \Gamma.\mathsf{Com}(\mathsf{accd}'_i, r'_i) \\
    \:\: (\mathsf{smpk}_i, \mathsf{tmpk}_i, \mathsf{rmpk}_i) := \mathbf{parse} \: \mathsf{mpks}_i \\
    \:\: \mathsf{spk}_i := \mathsf{smpk}_i + \Delta.\mathsf{Eval}(\delta''_i) \\
    \:\: \mathbf{if} \: \mathsf{rmpk}_i \neq \: \perp \land \: \mathsf{tmpk}_i \neq \: \perp \\
    \:\:\:\: \delta''_i \sample \mathcal{K} \\
    \:\:\:\: \delta'''_i \sample \mathcal{K} \\
    \:\:\:\: \mathsf{tpk}_i := \mathsf{tmpk}_i + \Delta.\mathsf{Eval}(\delta'_i) \\
    \:\:\:\: \mathsf{rpk}_i := \mathsf{rmpk}_i + \Delta.\mathsf{Eval}(\delta''_i) \\
    \:\:\:\: \mathsf{pks}_i := (\mathsf{spk}_i, \mathsf{tpk}_i, \mathsf{rpk}_i) \\
    \:\:\;\: \mathsf{tk}_i := (r'_i, \delta'_i, \delta''_i, \delta'''_i, \mathsf{accd}'_i) \\
    \;\: \mathbf{else} \\
    \:\:\:\: \mathsf{pks}_i := (\mathsf{spk}_i, \perp, \perp) \\
    \:\:\;\: \mathsf{tk}_i := (r'_i, \delta'_i, \mathsf{accd}'_i) \\
    \mathsf{ac}'_i := (\mathsf{pks}_i, \mathsf{co}'_i) \\
    \{\mathsf{ac}_i\}_{i \in U} := \mathsf{StExt(st)} \\
    \mathsf{AC}_R := \{\mathsf{ac}_i\}_{i \in R} \\
    \mathsf{AC}_T := \{\mathsf{ac}_i\}_{i \in T} \\
    \mathcal{Z}_{\bar{S}} := \{\Delta.\mathsf{Eval}(s_i)\}_{\phi S(i)\in \bar{S}} \\
    \mathsf{stmnt} := (P,\mathsf{AC}_R,\mathsf{AC}_T,\mathcal{Z}_{\bar{S}}, \mathsf{clock}) \\
    \mathsf{wit} := ((r,\mathsf{sks}_i, \mathsf{accd}_i)_{i\in S}), (\mathsf{mpks}_i, \mathsf{accd}'_i)_{i\in T}) \\
    \pi \gets \Pi.\mathsf{Prove}(\mathsf{crs},\mathsf{stmnt},\mathsf{wit}) \\
    \mathsf{tx} := (P,\mathsf{AC}_R,\mathsf{AC}_T,\mathcal{Z}_{\bar{S}}, \pi) \\
    \mathsf{TK} := {\mathsf{tk}_i}_{i \in T} \\
    \mathbf{return} \: (\mathsf{tx}, \mathsf{TK})
}
\vspace{1em}
\pseudocode{
    \mathsf{SrcChk}(\mathsf{ac}, r, \mathsf{sks}, \mathsf{accd}, \mathsf{clock}) \\[0.1\baselineskip ][\hline]
    (\mathsf{ssk}, \mathsf{tsk}, \mathsf{rsk}) := \mathbf{parse} \: \mathsf{sks} \\
    (\mathsf{pks}, \mathsf{co}) := \mathbf{parse} \: \mathsf{ac} \\
    (\mathsf{spk}, \mathsf{tpk}, \mathsf{rpk}) := \mathbf{parse} \: \mathsf{pks} \\
    \mathsf{time} := \mathbf{parse} \: \mathsf{accd} \\
    \mathbf{if} \: \mathsf{co} \neq \Gamma.\mathsf{Com}(\mathsf{accd}, r) \\
    \:\: \mathbf{return} \: 0 \\
    \mathbf{if} \: \mathsf{time} = \: \perp \\
    \:\: \mathbf{return} \: \mathsf{spk} \overset{?}{=} \Delta.\mathsf{KGen}(\mathsf{ssk}) \\
    \mathbf{else} \: \mathbf{if} \: \mathsf{time} <= \mathsf{clock} \\
    \:\: \mathbf{return} \: \begin{cases}
        \mathsf{tpk} \overset{?}{=} \Delta.\mathsf{KGen}(\mathsf{tsk}) \tabularnewline
        \mathsf{spk} \overset{?}{=} \Delta.\mathsf{KGen}(\mathsf{ssk}) \tabularnewline
    \end{cases} \\
    \mathbf{else} \\
    \:\: \mathbf{return} \: \mathsf{rpk} \overset{?}{=} \Delta.\mathsf{KGen}(\mathsf{rsk}) \\
}
\end{pcvstack}
\end{pchstack}
\end{minipage}%
\end{figure}

\newpage

\section{Commit transaction based atomic swaps}

\begin{figure}[H]
\begin{pchstack}[center, boxed]
\pseudocode{
    P_0(\mathsf{ssk_{\mathbb{A}}},\mathsf{rsk_{\mathbb{A}}},\mathsf{tsk_{\mathbb{B}}}) \qquad \qquad P_1(\mathsf{ssk_{\mathbb{B}}},\mathsf{rsk_{\mathbb{B}}},\mathsf{tsk_{\mathbb{A}}}) \\[0.1\baselineskip ][\hline] 
    \<\< \\[-0.4\baselineskip ]
    \mathcal{S_\mathbb{A}} := (\mathsf{ssk}_\mathbb{A}, \mathsf{rsk}_\mathbb{A}, \mathsf{tsk}_\mathbb{A}, \mathsf{accd}_\mathbb{A}) \\
    \mathsf{(tx_\mathbb{A}, TK_\mathbb{A})} := \mathsf{TxGen_\mathbb{A}}(\mathsf{st},P,R,\mathcal{S},\mathcal{T}) \\
    \mathcal{S_\mathbb{B}} := (\mathsf{ssk}_\mathbb{B}, \mathsf{rsk}_\mathbb{B}, \mathsf{tsk}_\mathbb{B}, \mathsf{accd}_\mathbb{B}) \\
    \mathsf{(tx_\mathbb{B}, TK_\mathbb{B})} := \mathsf{TxGen_\mathbb{B}}(\mathsf{st},P,R,\mathcal{S},\mathcal{T}) \\
    \mathbf{output} \: \mathsf{(tx_\mathbb{A} \oplus \mathsf{tx_\mathbb{B}}, TK_\mathbb{A})} \: \mathbf{to} \: P_1 \\
    \mathbf{output} \: \mathsf{(tx_\mathbb{B}, TK_\mathbb{B})} \: \mathbf{to} \: P_0 \\
}
\end{pchstack}
\caption{Protocol definition of 2PC $\Gamma_{\mathsf{CommitTx}}$}
\end{figure}


\begin{figure}[H]
\begin{minipage}[t]{0.5\textwidth}
\begin{pchstack}[boxed]
\pseudocode{
    \text{Global input} \:\: (T, \mathsf{amnt_a}, \mathsf{amnt_b},\mathbb{A}, \mathbb{B}) \\[0.1\baselineskip ][\hline] \\
    \mathsf{(rmpk, rmsk)} \gets \mathsf{KGen}_\mathsf{A}(\mathsf{pp}) \\
    \mathcal{S}_0 := \{(\mathsf{ssk}_\mathsf{A}, \perp, \perp, (\mathsf{amnt_a}, \perp))\} \\
    \mathcal{T}_0 := \{(\mathsf{smpk}_\mathsf{A}, \mathsf{tmpk}_\mathsf{A}, \mathsf{rmpk}_\mathsf{A}, (\mathsf{amnt_a}, T_0))\} \\
    \mathsf{(tx_0, TK_0)}_\mathbb{A} := \mathsf{TxGen_\mathbb{A}}(\mathsf{st},P,R,\mathcal{S}_0,\mathcal{T}_0) \\
    (\_, \mathsf{st}') := \mathsf{Vf}_\mathbb{A}(\mathsf{tx_0}) \\
    \mathsf{send}(\mathsf{(tx_0, TK_0)}_\mathbb{A}) \\
    \mathsf{\textbf{select}} \:\: \{ \\
    \quad \mathsf{\textbf{wait}} \:\: \{ \\
    \qquad \mathsf{timeout}(T_0) \\
    \quad \} \\
    \quad \mathsf{\textbf{wait}} \:\: \{ \\
    \qquad \mathsf{receive}(\mathsf{(tx_0, TK_0)}_\mathbb{B}) \\
    \qquad (\mathsf{res}, \mathsf{st}') := \mathsf{Vf}_\mathbb{B}(\mathsf{tx_0}) \\
    \qquad \mathsf{\textbf{if}} \:\: \mathsf{res} \neq 1 \\ % verify that commit transaction is valid and accepted
    \qquad \quad \mathsf{\textbf{return}} \perp \\
    \qquad (\mathsf{tx_1, TK_1}) \gets \Gamma.\mathsf{CommitTx}(\mathsf{ssk_\mathbb{A}}, \mathsf{rsk_\mathbb{A}}, \mathsf{tsk_\mathbb{B}}) \\
    \qquad (\mathsf{res}, \mathsf{st}') := \mathsf{Vf}_\mathbb{B}(\mathsf{tx_1}) \\
    \qquad \mathsf{\textbf{if}} \:\: \mathsf{res} \neq 1 \\ % verify that commit transaction is valid and accepted
    \qquad \quad \mathsf{\textbf{return}} \perp \\
    \quad \} \\
    \} \\
}
\end{pchstack}
\end{minipage}%
\hspace{0.4cm}
\begin{minipage}[t]{0.5\textwidth}
\begin{pchstack}[boxed]
\pseudocode{
    \text{Global input} \:\: (T, \mathsf{amnt_a}, \mathsf{amnt_b},\mathbb{A}, \mathbb{B}) \\[0.1\baselineskip ][\hline] \\
    \mathsf{(rmpk, rmsk)} \gets \mathsf{KGen}_\mathsf{B}(\mathsf{pp}) \\
    \mathcal{S}_0 := \{(\mathsf{ssk}_\mathsf{B}, \perp, \perp, (\mathsf{amnt_b}, \perp))\} \\
    \mathcal{T}_0 := \{(\mathsf{smpk}_\mathsf{B}, \mathsf{tmpk}_\mathsf{B}, \mathsf{rmpk}_\mathsf{B}, (\mathsf{amnt_b}, T_1))\} \\
    \mathsf{(tx_0, TK_0)}_\mathbb{B} := \mathsf{TxGen_\mathbb{B}}(\mathsf{st},P,R,\mathcal{S}_0,\mathcal{T}_0) \\
    (\_, \mathsf{st}') := \mathsf{Vf}_\mathbb{B}(\mathsf{tx_0}) \\
    \mathsf{send}(\mathsf{(tx_0, TK_0)}_\mathbb{B}) \\
    \mathsf{\textbf{select}} \:\: \{ \\
    \quad \mathsf{\textbf{wait}} \:\: \{ \\
    \qquad \mathsf{timeout}(T_1) \\
    \quad \} \\
    \quad \mathsf{\textbf{wait}} \:\: \{ \\
    \qquad \mathsf{receive}(\mathsf{(tx_0, TK_0)}_\mathbb{A}) \\
    \qquad (\mathsf{res}, \mathsf{st}') := \mathsf{Vf}_\mathbb{A}(\mathsf{tx_0}) \\
    \qquad \mathsf{\textbf{if}} \:\: \mathsf{res} \neq 1 \\ % verify that commit transaction is valid and accepted
    \qquad \quad \mathsf{\textbf{return}} \perp \\
    \qquad (\mathsf{lk, TK_1}) \gets \Gamma.\mathsf{CommitTx}(\mathsf{ssk_\mathbb{B}}, \mathsf{rsk_\mathbb{B}}, \mathsf{tsk_\mathbb{A}}) \\
    \qquad (\mathsf{res}, \mathsf{st}') := \mathsf{Vf}_\mathbb{B}(\mathsf{tx_1}) \\
    \qquad \mathsf{\textbf{if}} \:\: \mathsf{res} \neq 1 \\ % verify that commit transaction is valid and accepted
    \qquad \quad \mathsf{\textbf{return}} \perp \\
    \quad \} \\
    \} \\
}
\end{pchstack}
\end{minipage}%
\caption{Full protocol execution for $P_0$ and $P_1$, respectively left and right}
\end{figure}

\subsection{Efficient 2PC}
\vspace{1em}

$[M_0] := \begin{pmatrix}
[1] & 0 & 0 & -[s] \\
0 & [1] & 0 & -[t] \\
0 & 0 & [1] & -[r]
\end{pmatrix} $
$w_1 := \begin{bmatrix} s \\ t \\ r \\ 1 \\ \end{bmatrix}$
$[w_2] := \begin{pmatrix}
1-b & 0 & 0 \\
0 & c(1-b) & 0 \\
0 & 0 & b \\
\end{pmatrix}$
$[0] :=
\begin{pmatrix}
0 & 0 & 0 \\
0 & 0 & 0 \\
0 & 0 & 0 \\
\end{pmatrix} $
$([M_0]w_1)^T[w_2] = [0]$



\end{document}
