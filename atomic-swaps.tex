\documentclass{article}      	% Style of the document                     
\usepackage{fullpage}
\usepackage{amsmath,amsthm}     	   	% Maths    
\newtheorem{definition}{Definition}                                      
\usepackage[utf8]{inputenc}	% UTF-8 characters                                               
\usepackage[T1]{fontenc}    	% Tuki ääkkösille (Finnish names don't cause problems)                                            
\usepackage{parskip}        		% Linebreak between paragraphs                
\usepackage{svg}
\usepackage{graphicx}       		% Graphics package for adding figures                        
\usepackage{epstopdf}       		% Possibility to add *.eps figures
\usepackage{ dsfont }            % Symbol for real numbers
\usepackage{extarrows}
\usepackage{float}
\usepackage{makeidx}
\usepackage{enumitem}        % possibility to label list items by alphabet
\newcommand{\M}[1]{\ensuremath{\text{\texttt{#1}}}}
\usepackage[
    lambda,
    operators,
    advantage,
    sets,
    adversary,
    landau,
    probability,
    notions,
    logic,
    ff,
    mm,
    primitives,
    events,
    complexity,
    asymptotics,
    keys]{cryptocode}

% \usepackage{todonotes}
\usepackage{tcolorbox}

\usepackage{amsmath,amsfonts,graphicx,amssymb,amsthm}


\usepackage[bookmarksdepth=2,draft=false]{hyperref}
\hypersetup{colorlinks=true,linkcolor={red!50!black},citecolor=darkgray,linkcolor=darkgray}
\usepackage[capitalize]{cleveref}

\usepackage[draft]{comments-bugs-todos}

% TeX root = main.tex

%% general
\mathchardef\mhyphen="2D
\newcommand{\fdv}{\mathcal{F}}
\newcommand{\tdv}{\mathcal{T}}
\newcommand{\vdv}{\mathcal{V}}
\newcommand{\cX}{\mathcal{X}}
\newcommand{\cF}{\mathcal{F}}
\newcommand{\cG}{\mathcal{G}}
\newcommand{\ID}{\mathcal{I}}
\newcommand{\bits}[1][]{\{0,1\}^{#1}}
\renewcommand{\vec}[1]{\mathbf{#1}}
\newcommand{\mat}[1]{\mathbf{#1}}
\newcommand{\inner}[2]{\langle #1, #2 \rangle}
\newcommand{\transpose}{\mathtt{T}}
\newcommand{\round}[1]{\lfloor #1 \rceil}
\renewcommand{\dist}{\mathsf{dist}}
\renewcommand{\Pr}[2][]{{\text{Pr}_{#1}\left[#2\right]}}
\newcommand{\Exp}[2][]{{\mathbb{E}_{#1}\left[#2\right]}}
\newcommand{\mathcm}[2][1cm]{\hspace{#1}{\mbox{/\!\!/ } \text{\scriptsize#2}}}

%% lattice problems
\newcommand{\SIS}{\mathsf{SIS}}
\newcommand{\ISIS}{\mathsf{ISIS}}
\newcommand{\nfSIS}{\mathsf{nfSIS}}
\newcommand{\dSIS}{\mathsf{dSIS}}
\newcommand{\LWE}{\mathsf{LWE}}
\newcommand{\nfLWE}{\mathsf{nfLWE}}
\newcommand{\nfdLWE}{\mathsf{nfdLWE}}
\newcommand{\sLWE}{\mathsf{sLWE}}
\newcommand{\dLWE}{\mathsf{dLWE}}
\newcommand{\SVP}{\mathsf{SVP}}
\newcommand{\CVP}{\mathsf{CVP}}
\newcommand{\SIVP}{\mathsf{SIVP}}
\newcommand{\GapSVP}{\mathsf{GapSVP}}
\newcommand{\BDD}{\mathsf{BDD}}
\newcommand{\NTRU}{\mathsf{NTRU}}
\newcommand{\sNTRU}{\mathsf{sNTRU}}
\newcommand{\dNTRU}{\mathsf{dNTRU}}

%% lattice macros
\newcommand{\TT}{\mathbb{T}}
\newcommand{\ring}{\mathcal{R}}
\newcommand{\lattice}{\mathcal{L}}
\newcommand{\piped}{\mathcal{P}}
\newcommand{\ball}{\mathcal{B}}
\newcommand{\Hyb}{\mathsf{Hyb}}
\newcommand{\lspan}{\mathsf{span}}
\newcommand{\rank}{\mathsf{rank}}
\newcommand{\lsb}{\mathsf{LSB}}
\newcommand{\pubparam}{\mathsf{pp}}

%% group macros

%% syntax
\newcommand{\mpk}{\mathsf{mpk}}
\newcommand{\msk}{\mathsf{msk}}
\newcommand{\msg}{\mathsf{msg}}
\newcommand{\rnd}{\mathsf{rnd}}
\newcommand{\ctxt}{\mathsf{ctxt}}
\newcommand{\com}{\mathsf{com}}
\newcommand{\td}{\mathsf{td}}
\newcommand{\id}{\mathsf{id}}
\newcommand{\stmt}{\mathsf{stmt}}
\newcommand{\wit}{\mathsf{wit}}
\newcommand{\tx}{\mathsf{tx}}
\newcommand{\aux}{\mathsf{aux}}
\newcommand{\ek}{\mathsf{ek}}

\newcommand{\Setup}{\mathsf{Setup}}
\newcommand{\Commit}{\mathsf{Com}}
\newcommand{\TrapGen}{\mathsf{TrapGen}}
\newcommand{\SampD}{\mathsf{SampD}}
\newcommand{\SampPre}{\mathsf{SampPre}}
\newcommand{\Prove}{\mathsf{Prove}}
\newcommand{\Verify}{\mathsf{Verify}}
\newcommand{\val}{\mathsf{val}}

%% primitive/scheme name
\newcommand{\PKE}{\mathsf{PKE}}
\newcommand{\LTDF}{\mathsf{LTDF}}
\newcommand{\rsagen}{\mathsf{RSAGen}}
\newcommand{\rsa}{\mathsf{RSA}}
\newcommand{\LHE}{\mathsf{LHE}}
\newcommand{\C}{\mathcal{CS}}
\newcommand{\NTRUEncrypt}{\mathsf{NTRUEncrypt}}

%% others
\newcommand{\oracle}{\mathcal{O}}
\newcommand{\pcas}{~\mathbf{as}~}

\newcommand{\polylog}[1][\secpar]{\mathsf{polylog}(#1)}

\newcommand{\indrsidcpa}{\mathrm{IND\$}\mhyphen\mathrm{sID}\mhyphen\mathrm{CPA}}
%\newcommand{\oplus}{\, \texttt{XOR} \,} % shorthand for typing the XOR operator in mathmode



\usepackage{tikz}
\usetikzlibrary{decorations.pathreplacing}
\usetikzlibrary{decorations.pathmorphing}


\definecolor{cgreen}{RGB}{0, 153, 51}
\definecolor{cblue}{RGB}{0, 102, 204}
\definecolor{cyellow}{RGB}{255, 204, 0} 
\definecolor{cred}{RGB}{204, 51, 0} 

\newtcolorbox{todobox}{colback=yellow!3!white, colframe=white!75!black}

\newcommand{\commentline}[2]{%
    \tikz[remember picture, overlay]{
        \node [black,anchor=west,xshift=10pt] at (#1) {#2};
    }
}

\newcommand{\blockcomment}[3]{%
    \tikz[remember picture, overlay]{
        \draw [decorate,decoration={lineto,amplitude=10pt,mirror,raise=4pt},yshift=0pt,very thick,{#3}] 
        (#1) -- (#2) node [black,midway,xshift=10pt] {};
    }
}



\usepackage{biblatex}
\addbibresource{references.bib}

\begin{document}         
\author{Lorenzo Tucci}
\title{RingCCT: confidential commit transactions and atomic swaps}

\maketitle

\tableofcontents
\newpage


\begin{todobox}
\textbf{Narrative}
\begin{itemize}
\item Universal atomic swaps allow atomic swaps across arbitrary pairs of chains which support ordinary transactions, in particular without requiring support of scripting, time-lock contracts, etc.
\item This is appealing because most privacy chains (e.g. ZCash, Monero, MimbleWimble) do not support scripting.
\item However, UAS requires both parties to solve TLPs, which are computationally intensive especially for lightweight clients.
\item Moreover, it is tricky from an implementation perspective to properly set the difficulty level TLPs in UAS. For example, we identify a minor flaw ... and propose a fix.
\end{itemize}

\textbf{Contributions}
\begin{itemize}
\item We identify that a minimal chain functionality -- commit transactions -- suffices for achieving atomic swaps. Concretely, we propose a generic construction of an atomic swap protocol using only commit transactions and other basic functionalities of the chains. (To avoid dealing with UC, maybe we write this as an informal theorem?)
\item We propose an extension of RingCT, the underlying transaction scheme of Monero, which allows to realise commit transactions in privacy-preserving cryptocurrencies. We propose a generic construction and show how to efficiently instantiate it over groups where discrete logarithm and other related problems are hard. 
\item We provide a prototype implementation of CommitTx-based atomic swaps
\end{itemize}

\textbf{Discussion}
\begin{itemize}
    \item Why chains might want to support commit transactions but not "hash time lock contracts (HTLC)"? 
    \item From UAS paper: Both chains need to support the same hash function for HTLC-based atomic swaps to work. 
    \item On the other hand, commit transactions is a "local" functionality. 
    \item HTLC implies commit transaction (?) (But then this will imply HTLC-based atomic swaps do not need to share the same hash function?)
    \item In our construction, CommitTx accounts are trivially distinguishable from ordinary accounts. We justify this design choice by arguing that it is impossible to achieve "cross-type account indistinguishability" without violating availability.
\end{itemize}
\textbf{TODO}
\begin{itemize}
    \item Find out what are already known/considered for commit transactions (or however it is called)
    \item Discuss "natural applications" of commit transactions, beyond atomic swaps, e.g. P2P escrow system 
    \item Find out what goes wrong if one tries to realise HTLC in privacy-preserving chains, e.g. Monero.
    \item A more detailed comparison between CommitTx-based and HTLC-based atomic swaps, and see how exactly does the former avoid needing both chains to share common configurations, e.g. the same hash function.
\end{itemize}
\end{todobox}

% TeX root = main.tex

\begin{todobox}
    Potential Deadlines: 
    \begin{itemize}
        \item PETS 2026: Aug 31, 2025 and Nov 30, 2025
        \item EuroSP 2026: Oct-ish, 2025 
        \item SP 2026: Nov 13, 2025 
    \end{itemize}

    TODOs:
    \begin{itemize}
        \item Format: Prepare IEEE format and PETS format, and have a rough idea of what will end up in the submission version
        \item Intro: Comparison with other swap protocols for privacy-preserving chains: SwapCT \url{https://eprint.iacr.org/2021/631}, Sweep-UC \url{https://eprint.iacr.org/2022/1605}
        \item Atomic Swap: (Probably not this project?) Formal modelling and proving of atomic swap protocol security
        \item RingCCT Definition: Reduce number of master public keys from 3 to 2, and introduce the notion of ``type'' to the tags
        \item RingCCT Construction: Replace the tagging scheme with a PRF
        \item Instantiation and performance evaluation: Write it!
    \end{itemize}
\end{todobox}

\section{Introduction}

An atomic cross-chain swap, or atomic swap for short, is a cryptographic protocol that enables two mutually distrustful parties to exchange assets residing on separate blockchains, ensuring the \textit{atomicity} property: either both parties receive the corresponding asset from the other, or no transfer takes place. 
This guarantee must hold even in the presence of network delays or adversarial behavior, ensuring that neither party can cause a loss of assets for the other. 
To achieve this, atomic swap protocols rely on mechanisms that place each participant’s assets into a ``conditional transfer state—locked''\rlai{stete-locked? state-lock?} in such a way that it can be claimed by the counterparty upon satisfying specific conditions (e.g. revealing a secret or solving a puzzle), or refunded to the sender after a designated timeout. 

The predominant approach for implementing atomic swaps is via Hash Time-Locked Contracts (HTLCs), which combine hash preimages with time-locks to enforce atomicity. While effective, HTLCs require on-chain scripting capabilities, a requirement that is incompatible with many privacy-focused cryptocurrencies such as Monero or ZCash. These systems intentionally restrict or eliminate scripting functionality to enhance privacy and efficiency, rendering HTLCs infeasible in their context.

This limitation poses a significant barrier to interoperability: despite the growing demand for cross-chain trading, privacy-preserving blockchains remain largely isolated due to the lack of compatible atomic swap mechanisms. Prior works, such as Universal Atomic Swaps (UAS), have explored protocols that replace hash-locks with time-lock puzzles (TLPs), i.e. replacing the assumption of chains supporting scripting with a cryptographic assumption. While theorically sound, TLP-based protocols face serious practical challenges. In particular, their security critically depends on potential difference in (linear) computational power, which may be of several orders of magnitude faster with certain hardware capabilities (e.g. ASICs). As a result, tuning puzzle difficulty to balance security and usability leads to protocols that are either impractical for average honest users or insecure against adversaries with specialized hardware.
This motivates the development of an atomic swap protocol that requires minimal on-chain functionality, making it suitable for privacy-focused blockchains like Monero and ZCash, which lack scripting capabilities, yet still supporting efficient and secure cross-chain asset transfers.

\subsection{Our Contributions}
In this work, we propose a new minimal blockchain functionality called \emph{commit transactions} which enables supporting blockchains to perform atomic swap through minimal and efficient protocols. Our main contributions are outlined below.

\paragraph*{Commit transactions.} We propose \emph{commit transactions}, a general-purpose\rlai{We should argue why commit transactions is a more preferable/reasonable blockchain assumption than HTLCs. For example, we can expand on the ``general-purpose'' aspect, arguing why a chain might want to support it anyway even if it doesn't care about atomic swaps.} blockchain functionality for expressing time-based conditional asset transfers. Commit accounts allow users to lock funds under dynamic ownership conditions, such as transitioning from shared control to unilateral recovery based on a predefined timeout. The functionality is compatible with both UTXO and account-based ledger models and can be integrated into public ledgers with minimal modification to the transaction verification logic, requiring no general-purpose scripting and incurring no additional transaction size overhead.

\paragraph*{RingCCT (Ring Confidential Commit Transaction).} We extend the Ring Confidential Transaction (RingCT) model to Ring Confidential Commit Transaction (RingCCT) which includes the commit transaction functionality. A RingCCT scheme incorporates commit accounts with shared ownership semantics and time-conditional spending rules, while preserving the confidentiality and anonymity guarantees of standard RingCT. Extending the RingCT construction Omniring \cite{CCS:LRRSTW19}, we present a generic construction of RingCCT based on the same primitives, i.e. non-interactive zero-knowledge (NIZK) arguments, commitment schemes, and key-homomorphic pseudorandom functions, which can be efficiently instantiated in any Diffie-Hellman-hard groups. 

\paragraph*{Practical atomic swaps.} 
Leveraging the commit transaction functionality, we design an efficient and secure generic protocol for secure cross-chain swaps. We further show how to construct practical atomic swap protocols that can be executed between any pair of ledgers -- whether both are public blockchains, both are RingCCT-based, or a mix of the two.


\subsection{Related Work}

% TeX root = atomic-swaps.tex

\section{Preliminaries}

\begin{todobox}
    General notation, e.g. security parameter, $[n]$, PPT, negligible, ...
\end{todobox}

\subsection{Basic primitives}

\begin{todobox}
    commitments, ZKP, ...
\end{todobox}
\subsection{Non-Interactive Zero Knowledge Proofs}

Let $R: \{0, 1\}^* \times \{0, 1\}^* \rightarrow \{0, 1\}$ be a NP-witness-relation with corresponding NP-language $\mathcal{L} := \{x : \exists w \:\: \text{s.t.} \:\: R(x, w) = 1\}$

A non-interactive zero-knowledge proof (NIZK) system for R consist of the following algorithms:
\begin{itemize}
    \item $\mathsf{cr} \gets \mathsf{ZK}_\mathcal{L}.\mathsf{Setup}(1^\lambda)$ takes on input the security parameter, outputs a common reference string $\mathsf{crs}$
    \item $\pi \gets \mathsf{ZK}_\mathcal{L}.\mathsf{Pr}(\mathsf{crs}, x, w)$ takes on input the reference string $\mathsf{crs}$, a statement $x$ and a witness $w$, outputs a proof $\pi$
    \item $0/1 \gets \mathsf{ZK}_\mathcal{L}.\mathsf{Vr}(\mathsf{crs}, x,\pi)$ takes on input the reference string $\mathsf{crs}$, a statement $x$ and a proof $\pi$. Outputs 1 if $w$ is a witness for the statment $x$, 0 otherwise.
\end{itemize}
We require a NIZK system to be \textit{zero-knowledge}, where the verifier does not learn more than the validity of the statement $x$, and \textit{simulation sound} where it is hard for any prover
to convince a verifier of an invalid statement (chosen by the prover) even after having access to polynomially many simulated proofs for statements of his choosing.

\subsection{Secure 2-Party Computation}
A secure 2-party computa-tion (2PC) protocol allows two participating users $P_0$ and $P_1$ to securely compute some function $f$ over their private inputs $x_0$ and $x_1$ respectively.

We require the standard $privacy$ property, which states that the only information learned by the parties in the computation is that specified by the function output.
We also require the standard security with aborts, where the adversary can decide whether the honest party will receive the output of the computation or not, and thus there are no assumptions on fairness or guaranteed output delivery.

\subsection{Computational Assumptions}

\begin{todobox}
    General setting of group-based crypto, implicit notation, assumptions that we need e.g. DLOG 
\end{todobox}

% TeX root = main.tex

\section{Atomic Swaps and Existing Solutions}\label{sec:atomic_swap_overview}

\rlai*{Define (formally or informally) atomic swaps, overview existing constructions from HTLCs and TLPs.}

\subsection{Existing solutions}

\subsubsection{Hash Time Lock Contracts}

A Hash Time-Lock Contract (HTLC) is a contract that enables conditional payment based on the revelation of a cryptographic secret within a certain time window. Formally, an HTLC is characterized by a tuple  $(\mathsf{amnt_a}, h, T, \mathsf{pk_0}, \mathsf{pk_1})$ where
\begin{itemize}
	\item $\mathsf{amnt_a}$ denotes the amount of $\mathsf{a}$ assets to be exchanged
	\item $h$ is the hash of a secret value $r$, i.e., $h = \mathcal{H}(r)$ for some cryptographic hash function $\mathcal{H}$
	\item $T$ is a timelock parameter, typically a block height or timestamp, indicating the deadline after which funds can be refunded
	\item $\mathsf{pk_0}$ is the public key of the sender (who can reclaim the funds after the timeout).
	\item $\mathsf{pk_1}$ is the public key of the intended recipient (who can claim the funds upon presenting the correct preimage $r$ before the timeout).
\end{itemize}


The HTLC transfers $\mathsf{amnt_a}$ to $\mathsf{pk_1}$ if invoked before timeout $T$ with input value $r$ such that $\mathcal{H}(r) = h$. 
If the contract is invoked after timeout $T$, it refunds the assets $\mathsf{amnt_a}$ to $\mathsf{pk_0}$ unconditionally.

Using HTLCs as a building block, an atomic swap protocol can be constructed as follows: \\
1) Alice chooses $r$, computes $h = \mathcal{H}(r)$, transfers $\mathsf{amnt_a}$ into an $(\mathsf{amnt_a}, h, T_0, \mathsf{pk_0}, \mathsf{pk_1})$ on blockchain $\bca$ and sends $h,T$ to Bob. \\
2) Bob finishes the setup of the exchange by choosing a time $T_1 < T_0$ and transferring his $\mathsf{amnt_b}$ assets into an HTLC$(\mathsf{amnt_b}, h, T_1, \mathsf{pk_{tx}}, \mathsf{pk_{rx}})$ on blockchain $\bcb$.

HTLCs serve as the core building block in many atomic swap protocols. A standard two-party protocol between Alice and Bob proceeds as follows:

Alice chooses a uniformly random secret value $r$, computes $h = \mathcal{H}(r)$, and initiates an HTLC on blockchain $\bca$ locking $\mathsf{amnt_a}$ to Bob under the tuple $(\mathsf{amnt_a}, h, T_0, \mathsf{pk}\text{Alice}, \mathsf{pk}\text{Bob})$. She then sends $(h, T_0)$ to Bob.

Bob selects a smaller timeout $T_1 < T_0$ and sets up an HTLC on blockchain $\bcb$ locking his $\mathsf{amnt_b}$ to Alice under $(\mathsf{amnt_b}, h, T_1, \mathsf{pk}\text{Bob}, \mathsf{pk}\text{Alice})$.

Alice redeems the funds from Bob’s HTLC on $\bcb$ by revealing $r$. Since transactions on public blockchains are visible, Bob can then observe $r$ on-chain and use it to redeem the funds from Alice’s HTLC on $\bca$ before her timeout $T_0$.

This protocol ensures atomicity: either both parties receive their respective assets, or after the timeout, each party reclaims their original funds.

While HTLCs are widely used in practice, especially in early cross-chain systems and off-chain protocols such as the Lightning Network, they suffer from several critical limitations, especially when applied in a privacy-preserving or cross-paradigm setting.

\paragraph*{Compatibility of hash functions}
HTLC-based atomic swaps require both blockchains involved to support the same hash function. For instance, if chain $\bca$ supports SHA-256 and chain $\bcb$ supports Blake2, an HTLC constructed with a hash $h = \mathcal{H}(r)$ cannot be replicated on the second chain unless both parties can compute and verify the same preimage relation. One possible workaround is to have Alice compute $h = \mathcal{H}(r)$ and $h' = \mathcal{H}'(r)$ and publish a non-interactive zero-knowledge (NIZK) proof that both hashes are computed from the same preimage. However, this may increase protocol complexity and setup cost, and depends on the availability of NIZK-friendly primitives on both chains. Since the same preimage $r$ is reused on both chains, an external observer can trivially link the two transactions, breaking the unlinkability that privacy-preserving systems aim to provide. A workaround—such as committing to a shifted value $r + r'$ with an accompanying NIZK proof—adds cryptographic overhead and requires careful implementation to avoid privacy leaks.
\paragraph*{(In)compatibility with private ledgers}
Many privacy-preserving cryptocurrencies, such as Monero or Zcash, do not support expressive scripting or global hash preimage verifiability. Even when HTLCs are theoretically implementable (e.g., in a limited form over RingCT), their transaction structure becomes easily distinguishable from standard private transfers, harming plausible deniability. In such cases, the protocol becomes asymmetric: the public-chain participant must move first, revealing the preimage, which the private-chain user can then observe off-chain—this violates the symmetry typically desired in fair exchange protocols.
\paragraph*{Miner incentives attacks} HTLCs may also suffer from incentive misalignments, particularly in adversarial mining environments. Miners observing the hash preimage during the redemption phase may attempt to front-run or extract value, especially when the reward from a successful claim is higher than the block reward or other fees. These risks have been documented in Kolluri et al., 2022, which explores protocol designs vulnerable to such "griefing" attacks in HTLC settings.

\subsubsection{Universal Atomic Swaps}
Thyagarajan et al. \cite{uas} introduced one of the first atomic swap protocols substituting blockchain timelocks with a cryptographic primitive. 
The core building block utilized are Verifiable Timed Signatures (VTS) \cite{vts}, which lets a user generate a timed commitment $C$ of a signature $\sigma$ on a message $m$ under a public key $\mathsf{pk}$. The commitment $C $ must hide the signature $\sigma$ for time $\mathsf{T}$ and producing a proof $\pi$ that $C$ contains a valid signature $\sigma$. This ensures that $\sigma$ can be publicly recovered in time $\mathsf{T}$ by anyone who solves the computational puzzle. We note that a similar construction called Verifiable Timed Discretelog (VTD) allows to commit on a dlog value instead of a signature, and can be alternatively used in the protocol.

Let $P_0$ and $P_1$ be two parties where $P_0$ wants to exchange $\mathsf{amnt_a}$ on blockchain $\mathbb{A}$ from their address $\mathsf{pk_{init(\mathbb{A})}}$ for $\mathsf{amnt_b}$ on blockchain $\mathbb{B}$ to $\mathsf{pk_{swp(\mathbb{B})}}$ and vice-versa for $P_1$.

In the setup phase of the protocol, the parties run a 2PC protocol to setup two freeze addresses on the respective chains $\mathsf{pk_{frz(\mathbb{A})}}$ and $\mathsf{pk_{frz(\mathbb{B})}}$, where each party posseses one share of the respective secret keys, e.g. $\mathsf{sk_{frz(\mathbb{A})}} := \mathsf{sk_{frz0}} \oplus  \mathsf{sk_{frz1}}$. \\
Now the parties create a refund transaction transferring back the assets in case of timeout, for $P_0$ we have $\mathsf{tx_{rfnd(\mathbb{A})}}$ refunding $\mathsf{amnt_a}$ from $\mathsf{pk_{frz(\mathbb{A})}}$ to $\mathsf{pk_{init(\mathbb{A})}}$ and similarly for $P_1$ $\mathsf{tx_{rfnd(\mathbb{B})}}$. \\
Each party generates a timed commitment on the signature of the counterparty's refund transaction, where $P_0$ receives a $\mathsf{VTS}$ with commitment $C_0$ and timeout $T_0 = T_1 + \Delta$ and $P_1$ receives a $\mathsf{VTS}$ with commitment $C_1$ and timeout $T_1$. Once both $\mathsf{VTS}$ commitment are verified the parties proceed to transfer the assets to the freeze addresses, assured to retrieve the refund transaction signatures after force opening the commitments with timeouts $T_0$ and $T_1$.

In the subsequential lock phase, parties first initialize the swap transactions $\mathsf{tx_{swp}}$ transferring $\mathsf{amnt}$ from $\mathsf{pk_{frz}}$ to $\mathsf{pk_{swp}}$ for the respective chains. They then compute via 2PC $\mathsf{lk} := \sigma_{\mathsf{swp}(\mathbb{A})} \oplus \mathcal{H}(\sigma_{\mathsf{swp}(\mathbb{B})})$, where  $P_0$ receives $\sigma_{\mathsf{swp}(\mathbb{B})}$ and $P_1$ receives $\mathsf{lk}$.  When $P_0$ publishes $\mathsf{tx_{swp(\mathbb{B})}}$ together with  $\sigma_{\mathsf{swp}(\mathbb{B})}$, $P_1$ can unmask $\mathsf{lk}$ by computing $\mathcal{H}(\sigma_{\mathsf{swp}(\mathbb{B})})$ to retrieve $\sigma_{\mathsf{swp}(\mathbb{A})}$ and publish $\mathsf{tx_{swp(\mathbb{A})}}$.

 If $P_0$ fails to publish $\mathsf{tx_{swp(\mathbb{B})}}$ before $T_1$, $P_1$ will publish the refund transaction $\mathsf{tx_{rfnd(\mathbb{B})}}$ and similarly for $P_0$ if $P_1$ timeouts during the protocol execution.

Note that parties must also take into account potential differences in the computational power available for force opening the VTS commitments. This prevent scenarios where one party force opens its VTS commitments earlier than expected, potentially stealing 
 the other party's assets during the swap lock or complete phase. Therefore,  $\Delta$ (such that T0 = T1 + $\Delta$) must be large enough to tolerate time differences in opening the VTS commitments. \\

% TeX root = atomic-swaps.tex

\section{Commit Transactions and Atomic Swaps}

\subsection{Commit Transactions}

\begin{itemize}[topsep=0pt, itemsep=0pt, leftmargin=2em]
	\item Commit phase: A user makes a transaction sending $a$ coins to an account which is a binding commitment to the amount $a$, a main secret key $\mathsf{sk}_0$, a list of auxiliary secret keys $\mathsf{sk}_1, \dots, \mathsf{sk}_k$, and two (or more) index sets $I_0, I_1 \subseteq [k]$.
	\item Reveal phase: Before time $T$, the main key $\mathsf{sk}_0$ and the auxiliary keys $(\mathsf{sk}_i)_{i \in I_0}$ are needed to spend from the account. After time $T$, the main key $\mathsf{sk}_0$ and the auxiliary keys $(\mathsf{sk}_i)_{i \in I_1}$ are needed to spend from the account. The tag of an account is deterministically computed from the main key $\mathsf{sk}_0$.
\end{itemize}
%$(\mathsf{hpk}, \mathsf{tx}) \gets \mathbf{CommitTx}(\mathsf{spk}, \mathsf{rpk}, \mathsf{tpk}, \mathsf{ssk}, \mathsf{amnt}, \mathsf{T})$: creates a commit account, paying $\mathsf{amnt}$ from  to $\mathsf{rpk}$ valid until time $\mathsf{T}$, whith  $\mathsf{amnt}$ being locked from being spent on transactions from $\mathsf{pk_{tx}}$ until $\mathsf{T}$. The transaction can be later finalized by opening and broadcasting $C$ through $\mathsf{RevTx}$.

\subsection{Commit-Transaction-based Atomic Swaps}

\begin{figure}[H]
    \begin{pchstack}[center, boxed]
    \pseudocode{
        P_0(\mathsf{ssk_{\mathbb{A}}},\mathsf{rsk_{\mathbb{A}}},\mathsf{tsk_{\mathbb{B}}}) \qquad \qquad P_1(\mathsf{ssk_{\mathbb{B}}},\mathsf{rsk_{\mathbb{B}}},\mathsf{tsk_{\mathbb{A}}}) \\[0.1\baselineskip ][\hline] 
        \<\< \\[-0.4\baselineskip ]
        \mathcal{S_\mathbb{A}} := (\mathsf{ssk}_\mathbb{A}, \mathsf{rsk}_\mathbb{A}, \mathsf{tsk}_\mathbb{A}, \mathsf{accd}_\mathbb{A}) \\
        \mathsf{(tx_\mathbb{A}, TK_\mathbb{A})} := \mathsf{TxGen_\mathbb{A}}(\mathsf{st},P,R,\mathcal{S},\mathcal{T}) \\
        \mathcal{S_\mathbb{B}} := (\mathsf{ssk}_\mathbb{B}, \mathsf{rsk}_\mathbb{B}, \mathsf{tsk}_\mathbb{B}, \mathsf{accd}_\mathbb{B}) \\
        \mathsf{(tx_\mathbb{B}, TK_\mathbb{B})} := \mathsf{TxGen_\mathbb{B}}(\mathsf{st},P,R,\mathcal{S},\mathcal{T}) \\
        \mathbf{output} \: \mathsf{(tx_\mathbb{A} \oplus \mathsf{tx_\mathbb{B}}, TK_\mathbb{A})} \: \mathbf{to} \: P_1 \\
        \mathbf{output} \: \mathsf{(tx_\mathbb{B}, TK_\mathbb{B})} \: \mathbf{to} \: P_0 \\
    }
    \end{pchstack}
    \caption{Protocol definition of 2PC $\Gamma_{\mathsf{CommitTx}}$}
    \end{figure}
    
    
    \begin{figure}[H]
    \begin{minipage}[t]{0.5\textwidth}
    \begin{pchstack}[boxed]
    \pseudocode{
        \text{Global input} \:\: (T, \mathsf{amnt_a}, \mathsf{amnt_b},\mathbb{A}, \mathbb{B}) \\[0.1\baselineskip ][\hline] \\
        \mathsf{(rmpk, rmsk)} \gets \mathsf{KGen}_\mathsf{A}(\mathsf{pp}) \\
        \mathcal{S}_0 := \{(\mathsf{ssk}_\mathsf{A}, \perp, \perp, (\mathsf{amnt_a}, \perp))\} \\
        \mathcal{T}_0 := \{(\mathsf{smpk}_\mathsf{A}, \mathsf{tmpk}_\mathsf{A}, \mathsf{rmpk}_\mathsf{A}, (\mathsf{amnt_a}, T_0))\} \\
        \mathsf{(tx_0, TK_0)}_\mathbb{A} := \mathsf{TxGen_\mathbb{A}}(\mathsf{st},P,R,\mathcal{S}_0,\mathcal{T}_0) \\
        (\_, \mathsf{st}') := \mathsf{Vf}_\mathbb{A}(\mathsf{tx_0}) \\
        \mathsf{send}(\mathsf{(tx_0, TK_0)}_\mathbb{A}) \\
        \mathsf{\textbf{select}} \:\: \{ \\
        \quad \mathsf{\textbf{wait}} \:\: \{ \\
        \qquad \mathsf{timeout}(T_0) \\
        \quad \} \\
        \quad \mathsf{\textbf{wait}} \:\: \{ \\
        \qquad \mathsf{receive}(\mathsf{(tx_0, TK_0)}_\mathbb{B}) \\
        \qquad (\mathsf{res}, \mathsf{st}') := \mathsf{Vf}_\mathbb{B}(\mathsf{tx_0}) \\
        \qquad \mathsf{\textbf{if}} \:\: \mathsf{res} \neq 1 \\ % verify that commit transaction is valid and accepted
        \qquad \quad \mathsf{\textbf{return}} \perp \\
        \qquad (\mathsf{tx_1, TK_1}) \gets \Gamma.\mathsf{CommitTx}(\mathsf{ssk_\mathbb{A}}, \mathsf{rsk_\mathbb{A}}, \mathsf{tsk_\mathbb{B}}) \\
        \qquad (\mathsf{res}, \mathsf{st}') := \mathsf{Vf}_\mathbb{B}(\mathsf{tx_1}) \\
        \qquad \mathsf{\textbf{if}} \:\: \mathsf{res} \neq 1 \\ % verify that commit transaction is valid and accepted
        \qquad \quad \mathsf{\textbf{return}} \perp \\
        \quad \} \\
        \} \\
    }
    \end{pchstack}
    \end{minipage}%
    \hspace{0.4cm}
    \begin{minipage}[t]{0.5\textwidth}
    \begin{pchstack}[boxed]
    \pseudocode{
        \text{Global input} \:\: (T, \mathsf{amnt_a}, \mathsf{amnt_b},\mathbb{A}, \mathbb{B}) \\[0.1\baselineskip ][\hline] \\
        \mathsf{(rmpk, rmsk)} \gets \mathsf{KGen}_\mathsf{B}(\mathsf{pp}) \\
        \mathcal{S}_0 := \{(\mathsf{ssk}_\mathsf{B}, \perp, \perp, (\mathsf{amnt_b}, \perp))\} \\
        \mathcal{T}_0 := \{(\mathsf{smpk}_\mathsf{B}, \mathsf{tmpk}_\mathsf{B}, \mathsf{rmpk}_\mathsf{B}, (\mathsf{amnt_b}, T_1))\} \\
        \mathsf{(tx_0, TK_0)}_\mathbb{B} := \mathsf{TxGen_\mathbb{B}}(\mathsf{st},P,R,\mathcal{S}_0,\mathcal{T}_0) \\
        (\_, \mathsf{st}') := \mathsf{Vf}_\mathbb{B}(\mathsf{tx_0}) \\
        \mathsf{send}(\mathsf{(tx_0, TK_0)}_\mathbb{B}) \\
        \mathsf{\textbf{select}} \:\: \{ \\
        \quad \mathsf{\textbf{wait}} \:\: \{ \\
        \qquad \mathsf{timeout}(T_1) \\
        \quad \} \\
        \quad \mathsf{\textbf{wait}} \:\: \{ \\
        \qquad \mathsf{receive}(\mathsf{(tx_0, TK_0)}_\mathbb{A}) \\
        \qquad (\mathsf{res}, \mathsf{st}') := \mathsf{Vf}_\mathbb{A}(\mathsf{tx_0}) \\
        \qquad \mathsf{\textbf{if}} \:\: \mathsf{res} \neq 1 \\ % verify that commit transaction is valid and accepted
        \qquad \quad \mathsf{\textbf{return}} \perp \\
        \qquad (\mathsf{lk, TK_1}) \gets \Gamma.\mathsf{CommitTx}(\mathsf{ssk_\mathbb{B}}, \mathsf{rsk_\mathbb{B}}, \mathsf{tsk_\mathbb{A}}) \\
        \qquad (\mathsf{res}, \mathsf{st}') := \mathsf{Vf}_\mathbb{B}(\mathsf{tx_1}) \\
        \qquad \mathsf{\textbf{if}} \:\: \mathsf{res} \neq 1 \\ % verify that commit transaction is valid and accepted
        \qquad \quad \mathsf{\textbf{return}} \perp \\
        \quad \} \\
        \} \\
    }
    \end{pchstack}
    \end{minipage}%
    \caption{Full protocol execution for $P_0$ and $P_1$, respectively left and right}
    \end{figure}

\subsection{Comparison with HTLC-based Atomic Swaps}

\begin{definition}[Hash Time Lock Contract (HTLC)]
    A Hash Time Lock Contract (HTLC) is defined by a tuple $(\mathsf{amnt_a}, h, T, \mathsf{pk_0}, \mathsf{pk_1})$ where 
    \begin{itemize}
        \item $\mathsf{amnt_a}$ denotes the amount of $\mathsf{a}$ assets to be exchanged
        \item $h$ is a hash value
        \item $T$ the timeout
        \item $\mathsf{pk_{tx}}$ and $\mathsf{pk_{rx}}$ the public key addresses of two users
    \end{itemize}
\end{definition}

The HTLC transfers $\mathsf{amnt_a}$ to $\mathsf{pk_1}$ if invoked before timeout $T$ with input value $r$ such that $\mathcal{H}(r) = h$. 
If the contract is invoked after timeout $T$, it refunds the assets $\mathsf{amnt_a}$ to $\mathsf{pk_0}$ unconditionally.

Using HTLCs as a building block, an atomic swap protocol can be constructed as follows: \\
1) Alice chooses $r$, computes $h = \mathcal{H}(r)$, transfers $\mathsf{amnt_a}$ into an $(\mathsf{amnt_a}, h, T_0, \mathsf{pk_0}, \mathsf{pk_1})$ on blockchain $\mathbb{A}$ and sends $h,T$ to Bob. \\
2) Bob finishes the setup of the exchange by choosing a time $T_1 < T_0$ and transferring his $\mathsf{amnt_b}$ assets into an HTLC$(\mathsf{amnt_b}, h, T_1, \mathsf{pk_{tx}}, \mathsf{pk_{rx}})$ on blockchain $\mathbb{B}$.


\paragraph*{Comparison}
Clearly, HTLCs have the following drawbacks:
\begin{itemize}
	\item An HTLCs-based cross-chain protocol requires both chains to support the same hash function. (This could be avoided if the mover provides the hash-value $h'$ of $r$ supported by chain $\mathbb{B}$, together with a NIZK proving that $H(r) = h \land H'(r) = h'$ ; but complicates the protocol).
	\item The hash preimage is reused across different chains, creates a link between those payments, which is not privacy friendly. (Similarly to above we can give $H(r+r')$, and give $r'$ together with a proof that $H(r+r') = h' \land H(r) = h$).
	\item If naively implemented on RingCT, we make transaction easily distinguishable. If privately implemented, the protocol is asymmetric (public chain must always move first, in order to obtain preimage and mage of h).
	\item Straighforward private chain to private chain HTLCs might not be possible
	\item CommitTx, even for public chains, have no extra data on-chain except the timeout. Equivalent to a timeout-based multisig.
	\item Potential security issues (miner incentives) ? https://eprint.iacr.org/2022/546.pdf
\end{itemize}

% TeX root = atomic-swaps.tex

\section{RingCCT: Ring confidential commit transaction}
We present an extension of ring confidential transactions (RingCT), called ring confidential commit transactions (RingCCT).
RingCCT introduces an additional account abstraction that incorporates commitment-based ownership logic with epoch-based timeout semantics. More precisely, accounts in RingCCT are represented as commitments to account data, including both a token amount and a (possibly zero) timeout parameter, which governs conditional control over the committed asset.

At a high level, RingCCT abstracts the ledger into a set of accounts, each cryptographically encoded as a commitment $\mathsf{co}$ to underlying account data $\accd := (a, \time)$, where $a$ represents the committed amount and $\mathsf{time}$ optionally specifies an epoch-based timeout. Each account is associated with a tuple of public keys $(\mathsf{spk}, \mathsf{tpk}, \mathsf{rpk})$ that respectively define:

\begin{itemize}
	\item a primary owner key $\mathsf{spk}$,

	\item an optional joint-control timeout key $\mathsf{tpk}$, and

	\item a recovery key $\mathsf{rpk}$ which gains control after timeout.
\end{itemize}

Bfore the timeout epoch, spending from the account requires joint authorization from the primary and timeout keys; after the timeout, spending transitions to the recovery key alone. When no timeout is defined, the account behaves identically to a standard RingCT output, where only $\mathsf{spk}$ is required to authorize transactions.

\paragraph*{Account Types} We distinguish between two vfts of accounts:

\begin{itemize}
	\item Standard RingCT accounts: encoded as commitments to $(a, 0)$ with $\mathsf{spk}$ defined, and no meaningful $\mathsf{tpk}$ or $\mathsf{rpk}$. These replicate classic RingCT behavior.

	\item Commit accounts: commit to $(a, \time)$ with a complete triplet $(\mathsf{spk}, \mathsf{tpk}, \mathsf{rpk})$. These implement time-based joint ownership and recovery.
\end{itemize}

\paragraph*{TxGeneration Generation}
TxGenerations in RingCCT are generated via the algorithm $\mathsf{TxGen}$, which accepts a global state $\mathsf{st}$, a set of source account information $\mathcal{S}$, each including a tuple of secret keys $(\mathsf{ssk}, \mathsf{tsk}, \mathsf{rsk})$ and the committed data $\accd$, a set of target account data $\mathcal{T}$ (including public keys and updated $\accd'$), a ring of public accounts used to obfuscate the true input, and a predicate $P$ over source/target amounts (e.g., sum conservation).
The transaction enforces predicate correctness and zero-knowledge ownership proof via ring signatures, commitments, and zero-knowledge proofs that validates that knowledge of keys consistent with timeout logic, the conservation of committed amounts, and correct embedding of public keys and account data.

\paragraph*{Timeout-Aware Ownership Checks}
The algorithms $\mathsf{SrcChk}$ and $\mathsf{TgtChk}$ verify ownership and integrity of account data based on time:

$\mathsf{SrcChk}$ ensures that the provided secret keys correctly match the account’s public keys and timeout logic. If the account is a commit account with epoch timeout $\time$, it checks:

$(\mathsf{ssk}, \mathsf{tsk})$ are valid when the transaction clock $\mathsf{clock} \leq \time$,

$(\mathsf{rsk})$ is valid when $\mathsf{clock} > \time$.

$\mathsf{TgtChk}$ ensures that the target account includes a valid commitment to the new account data $\accd'$.

\paragraph*{State and TxGeneration Extraction}
The ledger state and transactions are abstracted into sets of committed accounts via $\mathsf{StExt}$ and $\mathsf{TxExt}$, enabling stateless verification, auditability, and off-chain analysis without leaking sensitive linkage information.

\subsection{Syntax}
\begin{definition}[Ring  Confidential Commit Transactions (RingCCT)]
    A RingCCT scheme consists of the PPT algorithms 
    \[\mathsf{Setup},\mathsf{KGen},\mathsf{KDer}, \mathsf{Tx},\mathsf{Vf},\mathsf{TimeVf}, \mathsf{StExt},\mathsf{TxExt}, \mathsf{TimeExt}, \mathsf{AccTimeExt}, \mathsf{SrcChk},\mathsf{TgtChk}\]
    whose interfaces are defined as follows.
    \begin{itemize}
        \item $(\mathsf{pp,st}) \gets \mathsf{Setup}(1^\lambda)$: the setup algorithm generates the public parameters $\mathsf{st}$ and an initial global state $\mathsf{st}$.
        \item $(\mathsf{mpk},\mathsf{msk}) \gets \mathsf{KGen}(\pp)$: the key generation algorithm generates a master public key $\mathsf{mpk}$ and a matching secret key $\mathsf{msk}$.
        \item $(\mathsf{sk},\accd) \gets \mathsf{KDer}(\mathsf{msk, \tau})$: the key derivation algorithm generates derives the keys-account data tuple given the master key $\mathsf{msk}$ owning the account and a token $\tau$ of the account.
	\item $(\mathsf{tx,TK}) \gets \mathsf{TxGen}(\mathsf{st},P,R,\mathcal{S},\mathcal{T})$: the transaction algorithm inputs a state $\mathsf{st}$, a predicate $P: \mathbb{Z}^S \times \mathbb{Z}^T \rightarrow \{0,1\}$, an index set R called the ring, a set of source accouts information $\mathcal{S} = \{\mathsf{sks}_i, \accd_i\}_{i\in S}$ and some targets account information $\mathcal{T} = \{\mathsf{mpks}_i, \accd'_i\}_{i\in T}$; where $\mathsf{sks}$ and $\mathsf{mpks}$ may respectively contain the source, target and recovery secret and public keys. If a source or target account is of $\atype$ 0, only the source key pair $\ssk, \mathsf{smpk}$ are defined, respectively. Each account defines its own data as $\accd := (a, \tout, \atype)$, where $a$ represents the amount of assets held by the account, $\atype$ is a bit defining the type of the account (0 for standard and 1 for commit) and $\mathsf{tout}$ sets a specific epoch timeout of the ownership of the target key pair of commit-type accounts, and set to 0 otherwise. 
        \item $(b,\st') \gets \mathsf{Vf}(\st,\tx)$: The verification algorithm outputs a bit b deciding whether to accept or reject that the transaction $\tx$ is a valid relative to the state $\st$, outputting an updated state $\st'$ if the verification is sucessful. This verification is time-independent.
\item $(b,\mathsf{st}') \gets \mathsf{TimeVf}(\st,\tx)$: The time verification algorithm performs the same verification as in $\mathsf{Vf}(\st, \tx)$ with the further constraint of checking whether the declared transaction type $\txtype$ and timeout $\tout$ are consistent with respect to the verifier's epoch $\time$ encoded in $\st$.
    \item $\mathsf{AC}_U \gets \mathsf{StExt}(\st)$: The state extraction algorithm
    extracts the set of universe accounts $\mathsf{AC}_U = \{\mathsf{ac}_i\}_{i \in U}$ encoded in the state $\st$.
    \item $\mathsf{AC}_T \gets \mathsf{TxExt}(\tx)$: The transaction extraction algorithm
    extracts the set of universe accounts $\mathsf{AC}_T = \{\mathsf{ac}_i\}_{i \in T}$ encoded in the state $\st$.
    \item $\time \gets \mathsf{TimeExt}(\tx)$: The time extraction algorithm
    extracts the epoch $\time$ encoded in the state $\mathsf{st}$.
\item $(\tout, \atype) \gets \mathsf{AccTimeExt}(\{\accd\})$: The account time extraction algorithm takes a set of account data $\{\accd\}$ and extracts the epoch timeout $\tout$ and the bit $\atype$ if and only if they are all equal in the set. Returns $\perp$ otherwise.
    \item $b \gets \mathsf{SrcChk}(\mathsf{ac,r,sks,accd,tout,txtype})$: The source checking algorithm outputs a bit $b$ deciding whether to accept or reject that the account $\mathsf{ac}$ is associated to the provided secret keys and that $\accd$ has been commited with randomness $r$. It then checks that the provided secret keys are valid according to the type of transaction $\txtype$
    \item $b \gets \mathsf{TgtChk}(\mathsf{ac,accd})$: The target checking algorithms outputs a bit $b$ deciding whether to accept or reject that $\accd$ has been commited in $\mathsf{ac}$. 
\end{itemize}

\subsection{Correctness}
\begin{definition}[Correctness] Let $\mathcal{P}$ be a family of predicates. A RingCCT scheme $\Omega$ is $\mathcal{P}$-correct if all of the following holds for any $\lambda \in \mathbb{N}$ and any $(pp, *) \in \mathsf{Setup}(1^\lambda)$.
\end{definition}

\paragraph*{Derivation correctness.} For any $(\mathsf{mpk}, \mathsf{msk}) \in \mathsf{KGen}(\pp)$, with $\mathsf{msk} \in \mathsf{msks}$, $\mathsf{mpk} \in \mathsf{mpks}$ $(\mathsf{sks}, \mathsf{ac}, \mathsf{tk}, \accd, \accd')$ satisfying $(\mathsf{sks}, \accd') \in \mathsf{KDer}(\mathsf{msks}, \mathsf{tk})$ and $\mathsf{TgtChk}(\mathsf{ac}, \mathsf{mpks}, \accd_{i}) = 1$ we have $\accd = \accd'$ and $\forall\pk \in \accd \mid \exists \sk \in \mathsf{sks} : \pk = \Delta.\mathsf{KGen}(\pk)$.


\paragraph*{TxGeneration correctness.} Define the set $V_\mathsf{pp}$ to be the collection of all tuples (\st, P, R, S, T) satisfying the following properties: 
\begin{itemize}
\item $\mathsf{P} \in \mathcal{P}$
\item $\mathsf{P}(\mathsf{a}_S, \mathsf{a}'_T) = 1$
\item $S \subseteq R \subseteq U$
\item $\mathsf{SrcChk}(\mathsf{StExt}(\mathsf{st})[i], \mathsf{sks}_{i}, \accd, \atype_{i}, \mathsf{EvalTags}(sks_i))$
\end{itemize}

where $\mathcal{S} = \{ \mathsf{sks}_i, \accd_i \}$, $\mathcal{T} = \{ \mathsf{mpks}'_i, \accd'_i \}$, $(a_i, \tout_i, \atype_i) = \accd_i$. For any $(\st, P, R, \mathcal{S}, \mathcal{T}) \in V_\mathsf{pp}$, and any $(\tx, \mathsf{tks}) \in \mathsf{TxGen}(\st,  P, R, \mathcal{S}, \mathcal{T})$, if $(b, \mathsf{st'}) = \mathsf{Vf}(\st, \tx)$, then the following hold: \\
\begin{itemize}
\item $b = 1$
\item $\mathsf{TxExt}(\tx) \subseteq \mathsf{StExt}(\st')$
\item $\mathsf{TgtChk}(\mathsf{TxExt}(\mathsf{tx}[i], \mathsf{mpks}_{i}, \accd_{i}) = 1$  for all i $\in T$
\end{itemize}

\subsection{Security}
We here define the security properties of RingCCT.

\paragraph*{Balance.} Balance guarantees account ownership and prevention of doublespending and over-speding. That is, an account can only spend owned amounts that they have not already spent. We first require that the source checking algorithms $\mathsf{SrcChk}$ is computationally binding to a set of secret keys and some amount, and just an amount for the target checking algorithm $\mathsf{TgtChk}$. We then model the balance property via the security experiment $\mathsf{Balance}_{\Omega,\mathcal{P},\adv,\epsilon_\adv}$, with $\adv$ being an adversary and $\epsilon_\adv$ a knowledge extractor. The adversary $\adv$ generates a sequence of valid transactions $\tx_i$ for $i \in \mathbb{Z}_l$. The knowledge extractor then $\epsilon_\adv$ extracts the information about source and target accounts for every transaction. The experiments returns 1 if some source or target account is ill-formed or if there exist distinct $i < i'$ such that the source account sets for the transactions $tx_i$ and $tx_i$ overlap, which corresponds to a doublespend.


\begin{definition}[Balance] A RingCCT scheme is balanced if: \\
1. The source checking algorithm $\mathsf{SrcChk}$ computationally binds an account to the stored amount and a set of secret keys determined by the account type $\atype$ and epoch timeout $\tout$, that is for any PPT adversary $\adv$ it holds that 
\vspace{0.3cm} \\
$\mathsf{Pr}\left[
    \begin{cases} 
	    \mathsf{SrcChk}(\mathsf{ac}, r, \ssk, \tsk, \rsk, \mathsf{accd}, \txtype, \mathcal{Z}_{S}) = 1 \tabularnewline
	\mathsf{SrcChk}(\mathsf{ac}, r, \ssk', \tsk', \rsk', \mathsf{accd}',\txtype, \mathcal{Z}_{S}') = 1 \tabularnewline
	(\txtype = 0 \land (\ssk, a) \neq (\ssk', a')) \: \lor \tabularnewline 
	(\txtype = 1 \land (\ssk, \tsk, a) \neq (\ssk', \tsk', a')) \: \lor \tabularnewline
	(\txtype = 2 \land (\ssk, \rsk, a) \neq (\ssk', \rsk', a'))
    \end{cases} 
    \middle|
    \begin{aligned}
	(\pp, \st) \gets \mathsf{Setup}(1^\lambda) \\
	(\mathsf{ac}, \ssk, \tsk, \rsk, \accd, \mathcal{Z}_{S}, \txtype) \gets \mathcal{A}(\pp) \\
	(\mathsf{ac}, \ssk', \tsk', \rsk', \accd',  \mathcal{Z}_{S}') \gets \mathcal{A}(\pp) \\
    \end{aligned}
\right]
\leq \negl
$ 
\vspace{0.3cm} \\
2. The target checking algorithm $\mathsf{TgtChk}$ computationally binds an account to the stored amount and account data, that is for any PPT adversary $\adv$ it holds that
\vspace{0.3cm} \\
$\mathsf{Pr}\left[
    \begin{cases} 
	\mathsf{TgChk}(\mathsf{ac}, \mathsf{tk}, (a, \tout, \atype)) = 1 \tabularnewline
	\mathsf{TgChk}(\mathsf{ac}, \mathsf{tk}', (a', \tout', \atype')) = 1 \tabularnewline
	(a, \tout, \atype) \neq (a', \tout', \atype')
    \end{cases} 
    \middle|
    \begin{aligned}
	(\pp, \st) \gets \mathsf{Setup}(1^\lambda) \\
	(\mathsf{ac}, \mathsf{tk}, \tout, a, \atype) \gets \mathcal{A}(\pp) \\
	(\mathsf{ac}, \mathsf{tk}', \tout', a', \atype') \gets \mathcal{A}(\pp) \\
    \end{aligned}
\right]
\leq \negl
$ 
\vspace{0.3cm} \\
3. For any PPT adversary $\adv$ there exists an expected polynomial-time extractor such that
\begin{equation*}
\mathsf{Pr}\left[\mathsf{Balance}_{\Omega,\mathcal{P},\adv,\epsilon_\adv}(1^\lambda) = 1\right] \leq \negl
\end{equation*}
where $\mathsf{Pr}\left[\mathsf{Balance}_{\Omega,\mathcal{P},\adv,\epsilon_\adv}\right]$ is defined in.\\
\end{definition}
The transaction verification is performed by extracting the current epoch from the state via the time extraction algorithm $\mathsf{TimeExt}$ and then by running the verification algorithm $\mathsf{TimeVf}$. This implies that the balance security properties covers both standard and commit accounts. \\
\begin{figure}[H]
\begin{pchstack}[center, boxed]
\pseudocode{
    \text{Balance} \\[0.1\baselineskip ][\hline] 
    (\pp, \mathsf{st}_0) \gets \mathsf{Setup}(1^\lambda) \\
    (\mathsf{tx}_i)_{i \in \mathbb{Z}_l} \gets \mathcal{A}(\pp, \mathsf{st}_0) \\
    (P_i, R_i, S_i, T_i)_{i \in \mathbb{Z}_{l}} \gets \mathcal{E}_{\mathsf{A}} (\pp, \mathsf{st}_0, (\mathsf{tx}_i)_{i \in \mathbb{Z}_{l}}) \\
    \{ \mathsf{sks}_{i,j}, \accd_{i,j} \}_{j \in S_i} := \parse \: (S_i)_{i \in \mathbb{Z}_{l}} \\
    \{ \mathsf{mpks}_{i,j},\mathsf{tk}_{i,j}, \accd'_{i,j} \}_{j \in T_i}) := \parse \: (T_i)_{i \in \mathbb{Z}_{l}} \\
    \mathbf{for} \: t \in \mathbb{Z}_l \: \mathbf{do} \: (b_t, \mathsf{st}_{t+1}) := \mathsf{TimeVf}(\mathsf{st_t}, \mathsf{tx_t}) \\
    \mathbf{for} \: i \in \mathbb{Z}_l \: \mathbf{do} \\
    \t \accd_{i, S_i} := (\accd_{i,j})_{j \in S_i},
    \accd'_{i, T_i} := (\accd'_{i,j})_{j \in T_i} \\
    \t \{ \mathsf{ac}_{i,j} \}_{j \in U_i} := \mathsf{StExt}(\mathsf{st}_i), \time_i := \mathsf{TimeExt}(\st_i) \\
    \t (\{ \mathsf{ac'}_{i,j} \}, \txtype_{i, j})_{j \in T_i} := \mathsf{TxExt}(\mathsf{tx}_i) \\
    \t b'_i := 
    \begin{cases}
	\mathsf{TxExt} \subseteq \mathsf{StExt}(\mathsf{st}_{i+1}) \vspace{0.3em} \tabularnewline
	P_i \in \mathcal{P} \tabularnewline
	P_i(\accd_{i, S_i}, \accd'_{i, T_i}) \vspace{0.3em} \tabularnewline
	S_i \subseteq R_i \subseteq U_i \vspace{0.3em} \tabularnewline
	\mathsf{SrcChk}(\mathsf{StExt}(\mathsf{st}_i)[j], \mathsf{sks}_{i,j}, \accd_{i,j}, \txtype_{i,j}, \mathsf{EvalTags}(sks_i)) = 1 \:\:\: \forall j \in \mathsf{S}_i \vspace{0.3em} \tabularnewline
	\mathsf{TgtChk}(\mathsf{TxExt}(\mathsf{tx}_i)[j], \mathsf{mpks}_{i,j}, \accd_{i,j}) = 1 \:\:\: \forall j \in \mathsf{T}_i \vspace{0.3em} \tabularnewline
    \end{cases} \\
    b'' := (\exists i_0 < i_1, S_{i_0} \cap S_{i_1} = \emptyset) \\
    \pcreturn \bigwedge_{i \in \mathbb{Z}_l} b_i \land \neg (\bigwedge_{i \in \mathbb{Z}_l} b_i' \land b_i'')
}
\end{pchstack}
\caption{Balance experiment definition}
\end{figure}

\paragraph*{Privacy.} Privacy captures spender and receiver anonymity alongside assets confidentiality. The property is modeled by the security experiment $\mathsf{Privacy}_{\Omega,\adv}^b$ which is parameterised by the bit $b$. The experiments first initialises the RingCCT system state $\st$ through $\mathsf{Setup}$, the adversary is then gives access to oracles for account generation, corruption, transaction and verification. \\
Standard and commit accounts can be generated by calling $\mathsf{AccGen}\mathcal{O}$ and $\mathcal{CommAccGen}\mathcal{O}$ respectively, which returns the set of associated public keys. Existing users can be corrupted via $\mathsf{Corr}\mathcal{O}$, which returns the secret key unless they belong to the set $\mathsf{ID}^*$. The verification oracle $\mathsf{TimeVf}\mathcal{O}$ takes on input a transactions and updates the system state $\st$. \\
The transaction oracle $\mathsf{TxGen}\mathcal{O}$ takes as input the tuple $(P, R, \mathcal{S}', \mathcal{S}^*, \mathcal{T}', \mathcal{T}^*)$ from $\adv$, where $\mathcal{S}'$ and $\mathcal{T}'$ are sets of source and target accounts with keypairs provided by the adversary and $\mathcal{S}^*$ and $\mathcal{T}^*$ encode instruct the oracle to retrieve keypairs of uncorrupted accounts. Combining these sets, the oracle creates the transaction and returns alongside the target account associated tokens. \\
The adversary generates a pair of transactions with input $(P, R, \mathcal{S}'_i, \mathcal{S}^*, \mathcal{T}'_i, \mathcal{T}^*)$ where $i \in {0.1}$, which corresponds to the same input except for  different sets of honest accounts. Honest accounts that are involved in verified transactions are recorded and blocked by including them in the set $\mathsf{AC}^*$. \\
The bit-parametised transaction is then returned to the adversary $\adv$, who can further interact with oracles and finally output a bit corresponding to the output of the experiment.

\begin{definition}[Privacy] A RingCCT scheme is private if for all PPT adversaries $\adv$ it holds that
\begin{equation*}
\mid \mathsf{Pr}\left[\mathsf{Privacy}_{\Omega,\adv}^0(1^\lambda) = 1\right] -  \mathsf{Pr}\left[\mathsf{Privacy}_{\Omega,\adv}^1(1^\lambda) = 1\right] \mid \:\: \leq \negl
\end{equation*}
Where $\mathsf{Privacy}_{\Omega,\adv}^b$ is defined in. 
\end{definition}

\begin{figure}
\begin{minipage}[t]{\textwidth}
\begin{pchstack}[boxed]
\begin{pcvstack}
\pseudocode{
    \mathsf{AccGen}\mathcal{O}(\mathsf{id}) \\[0.1\baselineskip ][\hline]
    \pcif \mathsf{id} \notin \mathsf{ID} \\
    \t (\mathsf{mpk},\mathsf{msk}) \gets \mathsf{KGen}(\pp) \\
    \t (\mathsf{MPK}, \mathsf{MSK})[\mathsf{id}] := ( \{ \mathsf{mpk} \}, \{ \mathsf{msk} \}) \\
    \mathsf{ID} := \mathsf{ID} \cup \{\mathsf{id}\} \\
    \pcreturn \: \mathsf{MPK}[\mathsf{id}]
}
\vspace{1em}
\pseudocode{
    \mathsf{ComAccGen}\mathcal{O}(\mathsf{id}) \\[0.1\baselineskip ][\hline]
    \pcif \mathsf{id} \notin \mathsf{ID} \\
    \t (\mathsf{smpk},\mathsf{smsk}) \gets \mathsf{KGen}(\pp) \\
    \t (\mathsf{tmpk},\mathsf{tmsk}) \gets \mathsf{KGen}(\pp) \\
    \t (\mathsf{rmpk},\mathsf{rmsk}) \gets \mathsf{KGen}(\pp) \\
    \t \mathsf{mpks} := \{ \mathsf{smpk}, \mathsf{tmpk}, \mathsf{rmpk} \} \\
    \t \mathsf{msks} := \{ \mathsf{smsk}, \mathsf{tmsk}, \mathsf{rmsk} \} \\
    \t (\mathsf{MPK}, \mathsf{MSK})[\mathsf{id}] := (\mathsf{mpks},\mathsf{msks}) \\
    \mathsf{ID} := \mathsf{ID} \cup \{\mathsf{id}\} \\
    \pcreturn \: \mathsf{MPK}[\mathsf{id}]
}
\vspace{1em}
\pseudocode{
	\mathsf{Vf}\mathcal{O}(\mathsf{tx}) \\[0.1\baselineskip ][\hline]
        \pcreturn \: \mathsf{Vf}(\mathsf{st}, \mathsf{tx})
}
\vspace{1em}
\pseudocode{
	\mathsf{TimeVf}\mathcal{O}(\mathsf{tx}) \\[0.1\baselineskip ][\hline]
        \pcreturn \: \mathsf{TimeVf}(\mathsf{st}, \mathsf{tx})
}

\end{pcvstack}
\qquad
\begin{pcvstack}
\pseudocode{
	\mathsf{TxGen}\mathcal{O}(P, R, \mathcal{S'}, \mathcal{T'}, \mathcal{S}^*, \mathcal{T}^*) \\[0.1\baselineskip ][\hline]
	 \{\mathsf{sks}_i, \accd_i\}_{i \in S'} := \parse \: \mathcal{S'}  \\
	 \{\mathsf{id}_i, \mathsf{tk}_i\}_{i \in S^*} := \parse \: \mathcal{S}^*  \\
	 \{\mathsf{mpks}_i, \accd'_i\}_{i \in T'} := \parse \: \mathcal{T'} \\
	 \{\mathsf{id}'_i, \accd'_i\}_{i \in T^*} := \parse \: \mathcal{T}^*  \\
         \pcif \mathcal{S}^* \cap \mathcal{S}' \neq \emptyset \lor \mathcal{T}' \cap \mathcal{T}^* \: \pcreturn \perp \\
         \pcif (\{\mathsf{id}_{i\in S^*}\} \cup \{\mathsf{id}_{i\in T^*}\}) \cap \mathsf{ID}^* \neq \emptyset \: \pcreturn \perp \\
         \pcif \mathsf{StExt}(\mathsf{st})[\mathcal{S}^*] \cap \mathsf{AC}^* \neq \emptyset \: \pcreturn \perp \\
         \mathcal{S} := \mathcal{S}' \cup \{\mathsf{KDer}(\mathsf{MSK}[\mathsf{id}_i], \mathsf{tk}_i)\}_{i \in S^*} \\
         \mathcal{T} := \mathcal{T}' \cup \{\mathsf{MPK}[\mathsf{id}'_i], \accd'_i\}_{i\in T^*} \\
         (\mathsf{tx}, \mathsf{tks}) \gets \mathsf{TxGen}(\mathsf{st}, P, R, \mathcal{S}, \mathcal{T}) \\
         \mathsf{AC}[\mathsf{id'}_i] := \mathsf{AC}[\mathsf{id'}_i] \cup \mathsf{TxExt}(\mathsf{tx})[i], \forall i \in T \\
         \pcreturn \: (\mathsf{tx}, \mathsf{TK})
}
\vspace{1em}
\pseudocode{
	\mathsf{Corr}\mathcal{O}(\mathsf{id}) \\[0.1\baselineskip ][\hline]
        \pcif \mathsf{id} \notin \mathsf{ID}^* \: \pcreturn \: \perp \\
        * \gets \mathsf{AccGen}\mathcal{O}(\mathsf{id}) \\
        \mathsf{ID}^* := \mathsf{ID}^* \cup \{\mathsf{id}\} \\
        \mathsf{AC}^* := \bigcup_{\mathsf{id} \in \mathsf{ID}^*} \mathsf{AC}[\mathsf{id}] \\
        \pcreturn \: \mathsf{MSK}[\mathsf{id}]
}
\end{pcvstack}
\end{pchstack}
\end{minipage}%
\caption{Oracles for privacy and availability experiments}
\end{figure}

\begin{figure}[H]
\begin{pchstack}[center, boxed]
\pseudocode{
    \mathsf{Privacy}^b_{\mathcal{A}} \\[0.1\baselineskip ][\hline] 
    (\pp, \mathsf{st}) \gets \mathsf{Setup}(1^\lambda) \\
    \mathcal{O} := \{\mathsf{AccGen}\mathcal{O}, \mathsf{CommAccGen}\mathcal{O},\mathsf{Corr}\mathcal{O}, \mathsf{TxGen}\mathcal{O}, \mathsf{TimeVf}\mathcal{O}\} \\
    (P,R, \mathcal{S}', \mathcal{T}', (\mathcal{S}^*_i, \mathcal{T}^*_i)_{i \in \{0,1\}}) \gets \mathcal{A}^\mathcal{O}(\pp) \\
    \mathbf{for} \: i \in \{0, 1\} \\
    \t (\mathsf{tx}_i, *) \gets \mathsf{TxGen}\mathcal{O}(P, R, \mathcal{S}', \mathcal{T}', \mathcal{S}^*_i, \mathcal{T}^*_i) \\
    \t (b_i, \mathsf{st'_i}) := \mathsf{TimeVf}(\mathsf{st}, \mathsf{tx}_i) \\
    \t \pcif b_i = 0 \: \pcreturn \: 0 \\
    \{ \mathsf{id}_{i,j},\mathsf{tk}_{i,j} \}_{j\in S^*_i} := \parse \: \mathcal{S}^*_i \\
    \{ \mathsf{id}'_{i,j},\accd_{i,j} \}_{j\in S^*_i} := \parse \: \mathcal{T}^*_i \\
    \mathsf{ID}^* := \mathsf{ID}^* \cup \{\mathsf{id}_{i,j}\}_{j\in S^*_i} \cup \{\mathsf{id;}_{i,j}\}_{j\in T^*_i} \\
    \mathsf{AC}^* := \mathsf{AC}^* \cup \mathsf{StExt}(\mathsf{st})[\mathcal{S}^*_i] \cup \mathsf{TxExt}(\mathsf{tx}_i)[\mathcal{T}^*_i] \\
    \mathbf{if} (|\mathcal{S^*_0}| \neq |\mathcal{S^*_1}|) \lor (|\mathsf{StExt}(\mathsf{st}'_0) \ \mathsf{StExt}(\mathsf{st})| \neq |\mathsf{StExt}(\mathsf{st}'_1) \ \mathsf{StExt}(\mathsf{st})|) \:\: \pcreturn \: 0 \\
    b' \gets \mathcal{A}^\mathcal{O}(\mathsf{tx}_b) \\
    \pcreturn \: b'
}
\end{pchstack}
\caption{Privacy experiment definition}
\end{figure}

\begin{definition}[Privacy] A RingCCT scheme is available if for all PPT adversaries $\adv$ it holds that
\begin{equation*}
\mathsf{Pr}\left[\mathsf{Availability}_{\Omega,\adv}(1^\lambda) = 1\right] \leq \negl
\end{equation*}
Where $\mathsf{Availability}_{\Omega,\adv}$ is defined in. 
\end{definition}

\begin{figure}[H]
\begin{pchstack}[center, boxed]
\pseudocode{
    \mathsf{Available}_{\mathcal{A}} \\[0.1\baselineskip ][\hline] 
    (\pp, \mathsf{st}) \gets \mathsf{Setup}(1^\lambda) \\
    \mathcal{O} := \{\mathsf{KGen}\mathcal{O}, \mathsf{Corr}\mathcal{O}, \mathsf{TxGen}\mathcal{O}, \mathsf{Vf}\mathcal{O}\} \\
    (P,R, \mathcal{S}', \mathcal{T}', (\mathcal{S}^*_i, \mathcal{T}^*_i)_{i \in \{0,1\}}) \gets \mathcal{A}^\mathcal{O}(\pp) \\
    (\mathsf{tx}, \mathsf{TK}) \gets \mathsf{TxGen}\mathcal{O}(P, R, \mathcal{S}', \mathcal{T}', \mathcal{S}^*_i, \mathcal{T}^*_i)
    \{ \mathsf{id}_j,\mathsf{tk}_j \}_{j\in S^*} := \parse \: \mathcal{S}^* \\
    \mathbf{if} \mathcal{S}^* \not\subseteq U\: \pcreturn \: 0 \\
    (\mathsf{ID}^*, \mathsf{AC}^*) := (\{id_j\}_{j \in S^*}, \mathsf{StExt}(\mathsf{st})[\mathcal{S}^*])
    (b, \perp) := \mathsf{Vf}(\mathsf{st}, \mathsf{tx}) \\
    \perp \gets \mathcal{A}\mathcal{O}(\mathsf{tx}, \mathsf{TK}) \\
    (b', \perp) := \mathsf{Vf}(\mathsf{st}, \mathsf{tx}) \\
    \pcreturn \: b \land b'
}
\end{pchstack}
\caption{Availability experiment definition}
\end{figure}

\newpage

\subsection{Construction}

\begin{equation*}
\mathcal{R}(\mathsf{stmnt}, \mathsf{wit}) := \begin{cases} 
    S \subseteq R \\ 
    \mathsf{TagChk}(\mathcal{Z}_{S(i)}) \forall i \in S \\
    \mathsf{SrcChk}(\mathsf{ac}_i, r, \mathsf{sks}_i, \accd_i, \time, \txtype) = 1 \qquad \forall i \in S \\ 
    \mathsf{TgtChk}(\mathsf{ac'}_i, \accd'_i) = 1 \qquad \forall i \in T \\ 
    P(a_S, a'_T) = 1
\end{cases}
\end{equation*}

\begin{equation*}
\mathsf{stmnt} := (P,\mathsf{AC}_R,\mathcal{Z}_{\bar{S}}, \mathsf{AC}_T, \time, \txtype) \\
\end{equation*}
\begin{equation*}
\mathsf{wit} := ((r,\mathsf{sks}_i, \accd_i)_{i\in S}), (\mathsf{mpks}_i, \accd'_i)_{i\in T}) \\
\end{equation*}

\begin{figure}
\begin{minipage}[t]{\textwidth}
\begin{pchstack}[boxed]
\begin{pcvstack}
\pseudocode{
    \Setup(1^\lambda) \\ [0.1\baselineskip ][\hline]
    \mathsf{crs} \gets \Pi.\Setup(1^\lambda) \\
    \mathsf{ck} \gets \Gamma.\mathsf{Gen}(1^\lambda) \\
    \pp_\Delta \gets \Delta.\Setup(1^\lambda) \\
    \pcreturn (\pp, \mathsf{st})
}
\vspace{1em}
\pseudocode{
    \mathsf{TimeExt}(\mathsf{st}) \\[0.1\baselineskip ][\hline]
    (\mathsf{AC}_U, \mathcal{Z}_U, \time) := \parse \mathsf{st} \\
    \pcreturn \time
}
\vspace{1em}
\pseudocode{
    \mathsf{StExt}(\mathsf{st}) \\[0.1\baselineskip ][\hline]
    (\mathsf{AC}_U, \mathcal{Z}_U, \time) := \parse \mathsf{st} \\
    \pcreturn \mathsf{AC}_U
}
\end{pcvstack}
\qquad
\begin{pcvstack}
\pseudocode{
    \mathsf{KGen}(\pp) \\[0.1\baselineskip ][\hline]
    \mathsf{msk} \sample\mathcal{K} \\
    \mathsf{mpk} := \Delta.\mathsf{KGen(msk)} \\
    \pcreturn (\mathsf{mpk}, \mathsf{msk})
}
\vspace{1em}
\pseudocode{
    \mathsf{TxExt}(\mathsf{tx}) \\[0.1\baselineskip ][\hline]
    (P,R,\mathsf{AC}_T, \mathcal{Z}_{\bar{S}}) := \parse \mathsf{tx} \\
    \pcreturn \mathsf{AC}_T
}\vspace{1em}
\pseudocode{
    \mathsf{ExtAccType}(\mathsf{AC}) \\[0.1\baselineskip ][\hline]
    \mathbf{assert} \not\exists \: a,b \in \{ (\mathsf{tout}_i, \mathsf{atype}_i) \}_{i \in \mathsf{AC}} \mid  a \neq b \\
    \pcreturn (\tout, \atype)
}\vspace{1em}
\pseudocode{
    \mathsf{EvalTags}(\mathsf{sks}) \\[0.1\baselineskip ][\hline]
    \pcreturn \{ \Delta.\mathsf{Eval}(\mathsf{sk}_i) \}_{i \in \mathsf{sks}}  \\
}
\end{pcvstack}
\end{pchstack}
\end{minipage}%
\end{figure}
\begin{figure}
\begin{minipage}[t]{\textwidth}
\begin{pchstack}[boxed]

\begin{pcvstack}
\pseudocode{
    \mathsf{Vf}(\mathsf{st},\mathsf{tx}) \\[0.1\baselineskip ][\hline]
    (\mathsf{AC}_U, \mathcal{Z}_U, \time) := \parse \mathsf{st} \\
    \{\mathsf{ac}_i\}_{i \in U} := \parse \mathsf{AC}_U \\
    (P,R,\mathsf{AC}_T, \mathcal{Z}_{\bar{S}}, \tout, \txtype, \pi) := \parse \mathsf{tx} \\
    \mathsf{AC}_R := \{\mathsf{ac}_i\}_{i \in R} \\
    \mathsf{stmnt} := (P,\mathsf{AC}_R,\mathsf{AC}_T,\mathcal{Z}_{\bar{S}}, \txtype, \tout) \\
    \pcif \begin{cases}
        P \in \mathcal{P} \tabularnewline
        R \subseteq U \tabularnewline
        \Pi.\mathsf{Vf}(\mathsf{crs}, \mathsf{stmnt}, \pi) = 1 \tabularnewline
        \mathcal{Z}_{\bar{S}} \cap \mathcal{Z}_{\bar{U}} = \emptyset
    \end{cases} \: \mathbf{then} \\
    \t \pcreturn (1, \mathsf{st}') \\
    \pcelse \pcreturn \: (0, \mathsf{st})
}
\vspace{1em}
\pseudocode{
    \mathsf{KDer}(\mathsf{msks},\tau) \\[0.1\baselineskip ][\hline]
    (r, \delta, \accd) := \parse \tau \\
    \mathsf{sks} := \{ \mathsf{msk}_i +\delta \}_{i\in \mathsf{msks}}\\
    \pcreturn (\mathsf{sks}, r, \accd)
}
\vspace{1em}
\pseudocode{
    \mathsf{TgtChk}(\mathsf{ac}, \accd) \\[0.1\baselineskip ][\hline]
    (\mathsf{pks}, \mathsf{co}) := \parse \mathsf{ac} \\
    (r, \delta, \accd') := \parse \mathsf{tk} \\
    \pcreturn \begin{cases}
        %\accd' \overset{?}{=} \accd \tabularnewline
        \mathsf{co} \overset{?}{=} \Gamma.\mathsf{Com}(\accd,r)
    \end{cases} 
}
\vspace{1em}
\pseudocode{
    \mathsf{SrcChk}(\mathsf{ac}, r, \mathsf{sks}, \accd, \txtype, \mathcal{Z}) \\[0.1\baselineskip ][\hline]
    (\mathsf{ssk}, \mathsf{tsk}, \mathsf{rsk}) := \parse \mathsf{sks} \\
    (\mathsf{pks}, \mathsf{co}) := \parse \mathsf{ac} \\
    (\mathsf{spk}, \mathsf{tpk}, \mathsf{rpk}) := \parse \mathsf{pks} \\
    \mathbf{assert} \: \atype = \txtype \overset{?}{\neq} 0 \\
    \mathbf{assert} \: \mathsf{co} \neq \Gamma.\mathsf{Com}(\accd, r) \\
    \pcif  \txtype = \: 0 \\
    \t \pcreturn \begin{cases} 
    	\mathsf{spk} \overset{?}{=} \Delta.\mathsf{KGen}(\mathsf{ssk}) \tabularnewline
	\Delta.\mathsf{Eval}(\mathsf{ssk}) \overset{?}{=} \mathcal{Z}
    \end{cases} \\
    \pcelse \pcif  \txtype = 1  \\
    \t \pcreturn  \begin{cases}
        \mathsf{tpk} \overset{?}{=} \Delta.\mathsf{KGen}(\mathsf{tsk}) \tabularnewline
        \mathsf{spk} \overset{?}{=} \Delta.\mathsf{KGen}(\mathsf{ssk}) \tabularnewline
	\Delta.\mathsf{Eval}(\mathsf{ssk}), \Delta.\mathsf{Eval}(\mathsf{tsk}) \overset{?}{=} \mathcal{Z}
    \end{cases} \\
    \pcelse \\
    \t \pcreturn \begin{cases} 
	\mathsf{rpk} \overset{?}{=} \Delta.\mathsf{KGen}(\mathsf{rsk}) \tabularnewline
	\Delta.\mathsf{Eval}(\mathsf{ssk}), \Delta.\mathsf{Eval}(\mathsf{rsk}) \overset{?}{=} \mathcal{Z}
    \end{cases} \\
}
\end{pcvstack}
\qquad
\begin{pcvstack}
\pseudocode{
    \mathsf{TimeVf}(\mathsf{st}, \mathsf{tx}) \\[0.1\baselineskip ][\hline]
    (\mathsf{AC}_U, \mathcal{Z}_U, \time) := \parse \mathsf{st} \\
    (P,R,\mathsf{AC}_T, \mathcal{Z}_{\bar{S}}, \tout, \txtype, \pi) := \parse \mathsf{tx} \\
    \pcif \txtype = 0 \: \lor \\
    \t (\txtype = 1 \land  \time \leq \tout) \: \lor\\
    \t (\txtype = 2 \land  \time > \tout) \\
               \t\t \pcreturn \: \mathsf{Vf}(\mathsf{st}, \mathsf{tx}) \\
    \pcreturn (0, \mathsf{st})
}
\vspace{1em}
\pseudocode{
    \mathsf{TxGen}(\mathsf{st}, P, R, \mathcal{S}, \mathcal{T}) \\[0.1\baselineskip ][\hline]
    \{\mathsf{sks}_i, \accd_i\}_{i\in S} := \parse \mathcal{S} \\
    \{\mathsf{mpks}_i, \accd'_i\}_{i\in T} := \parse \mathcal{T} \\
    \{\mathsf{ac}_i\}_{i \in U} := \mathsf{StExt(st)} \\
    \time := \mathsf{TimeExt(st)} \\
    \tout, \atype \gets \mathsf{AccTimeExt}(\{\accd_i\}_{i\in S}) \\
    \txtype := \atype \cdot (1 + \tout \overset{?}{\leq} \time) \\
    \mathbf{for} \; i \in T \: \mathbf{do} \\
    \t r_i \sample \chi \\
    \t \delta_i \sample \mathcal{K} \\
    \t \mathsf{co}_i := \Gamma.\mathsf{Com}(\accd'_i, r_i) \\
    \t (a_i, \tout, b_i) := \parse \accd'_i \\
    \t (\mathsf{smpk}_i, \mathsf{tmpk}_i, \mathsf{rmpk}_i) := \parse \mathsf{mpks}_i \\
    \t \mathsf{spk}_i := \mathsf{smpk}_i + \Delta.\mathsf{Eval}(\delta_i) \\
    \t \mathsf{tk}_i := (r_i, \delta_i, \accd'_i) \\
    \t \pcif b \neq 0 \\
    \t\t \mathsf{tpk}_i := \mathsf{tmpk}_i + \Delta.\mathsf{Eval}(\delta_i) \\
    \t\t \mathsf{rpk}_i := \mathsf{rmpk}_i + \Delta.\mathsf{Eval}(\delta_i) \\
    \t\t \mathsf{pks}_i := (\mathsf{spk}_i, \mathsf{tpk}_i, \mathsf{rpk}_i) \\
    \t \pcelse \\
    \t\t \mathsf{pks}_i := (\mathsf{spk}_i, \perp, \perp) \\
    \t \mathsf{ac}'_i := (\mathsf{pks}_i, \mathsf{co}'_i) \\
    \mathsf{AC}_R := \{\mathsf{ac}_i\}_{i \in R} \\
    \mathsf{AC}_T := \{\mathsf{ac}'_i\}_{i \in T} \\
    \mathcal{Z}_{\bar{S}} := \mathsf{EvalTags}(\mathsf{sks}_i)_{i \in S} \\
    \mathsf{stmnt} := (P,\mathsf{AC}_R,\mathsf{AC}_T,\mathcal{Z}_{\bar{C}}, \tout, \txtype) \\
    \mathsf{wit} := ((r,\mathsf{sks}_i, \accd_i)_{i\in S}), (\mathsf{mpks}_i, \accd'_i)_{i\in T}) \\
    \pi \gets \Pi.\mathsf{Prove}(\mathsf{crs},\mathsf{stmnt},\mathsf{wit}) \\
    \mathsf{tx} := (P,\mathsf{AC}_R,\mathsf{AC}_T,\mathcal{Z}_{\bar{C}}, \tout, \txtype, \pi) \\
    \mathsf{TK} := {\mathsf{tk}_i}_{i \in T} \\
    \pcreturn (\mathsf{tx}, \mathsf{TK})
}
\end{pcvstack}
\end{pchstack}
\end{minipage}%
\end{figure}
\newpage

% TeX root = atomic-swaps.tex

\section{Instantiation and Performance Evaluation}

\subsection{Instantiation}

\paragraph{Commitment}

\paragraph{Tagging Scheme}

\paragraph{2PC}

\begin{figure}[H]
    \begin{pchstack}[center, boxed]
    \pseudocode{
    P_0(\mathsf{ssk^0_{\mathbb{A}}},\mathsf{spk^1_{\mathbb{B}}},\mathsf{tsk_{\mathbb{B}}}) \qquad \qquad P_1(\mathsf{ssk^0_{\mathbb{B}}},\mathsf{spk^1_{\mathbb{A}}},\mathsf{tx}) \\[0.1\baselineskip ][\hline] 
        \<\< \\[-0.4\baselineskip ]
	%\mathsf{spk}^1_{\mathbb{B}} = [\mathsf{ssk^1}_{\mathbb{B}] \\
	%\mathsf{spk}^0_{\mathbb{A}} = [\mathsf{ssk^0}_{\mathbb{A}] \\
    }
    \end{pchstack}
    \caption{Protocol definition of 2PC (RingCCT - public) $\Gamma_{\mathsf{CommitTx}}$}
\end{figure}

\begin{figure}[H]
    \begin{pchstack}[center, boxed]
    \pseudocode{
        P_0(\mathsf{ssk^0_{\mathbb{A}}},\mathsf{rsk_{\mathbb{A}}},\mathsf{tsk_{\mathbb{B}}}) \qquad \qquad P_1(\mathsf{ssk_{\mathbb{B}}},\mathsf{rsk_{\mathbb{B}}},\mathsf{tsk_{\mathbb{A}}}) \\[0.1\baselineskip ][\hline] 
        \<\< \\[-0.4\baselineskip ]
    }
    \end{pchstack}
    \caption{Protocol definition of 2PC (RingCCT - RingCCT) $\Gamma_{\mathsf{CommitTx}}$}
\end{figure}

\begin{todobox}
    Write down explicitly (i.e. in terms of $\GG$ and $\ZZ_q$ arithmetic) for which functionalities do we need 2PCs.     
\end{todobox}


\paragraph{Zero-Knowledge Proofs}

\subsection{Performance Evaluation}

\end{document}

